\documentclass[12pt,english,a4paper,titlepage,cleardoublepage=empty,dottedtoc]{report}



% % % additions by tompollard START -->

% Overwrite \begin{figure}[htbp] with \begin{figure}[H]
% TODO: delete if images position is awkward
\usepackage{float}
\let\origfigure=\figure
\let\endorigfigure=\endfigure
\renewenvironment{figure}[1][]{%
\origfigure[b]
}{%
\endorigfigure
}

% TP: hack to truncate list of figures/tables.
\usepackage{truncate}
\usepackage{caption}
\usepackage{tocloft}
% TP: end hack

% % % <-- END additions by tompollard



\usepackage{lmodern}
\usepackage{setspace}
\setstretch{1.2}
\usepackage{amssymb,amsmath}
\usepackage{ifxetex,ifluatex}
\usepackage{fixltx2e} % provides \textsubscript
\ifnum 0\ifxetex 1\fi\ifluatex 1\fi=0 % if pdftex
  \usepackage[T1]{fontenc}
  \usepackage[utf8]{inputenc}
\else % if luatex or xelatex
  \ifxetex
    \usepackage{mathspec}
  \else
    \usepackage{fontspec}
  \fi
  \defaultfontfeatures{Ligatures=TeX,Scale=MatchLowercase}
\fi
% use upquote if available, for straight quotes in verbatim environments
\IfFileExists{upquote.sty}{\usepackage{upquote}}{}
% use microtype if available
\IfFileExists{microtype.sty}{%
\usepackage{microtype}
\UseMicrotypeSet[protrusion]{basicmath} % disable protrusion for tt fonts
}{}
\usepackage[unicode=true]{hyperref}
\hypersetup{
            pdftitle={Open Specification of a user-controlled Web Service for Personal Data},
            pdfauthor={G. Jahn},
            pdfkeywords={personal data store; personal data as a service; web service architecture; open specification; masters thesis},
            pdfborder={0 0 0},
            breaklinks=true}
\urlstyle{same}  % don't use monospace font for urls
\ifnum 0\ifxetex 1\fi\ifluatex 1\fi=0 % if pdftex
  \usepackage[shorthands=off,main=english]{babel}
\else
  \usepackage{polyglossia}
  \setmainlanguage[]{english}
\fi

% added 20170205 -->
% NOTE: Replace "english" in "captionsenglish" with the language you use
\addto\captionsenglish{
  \renewcommand{\contentsname}
    {Table of Contents}
}
% added 20170205 <--

\usepackage{color}
\usepackage{fancyvrb}
\newcommand{\VerbBar}{|}
\newcommand{\VERB}{\Verb[commandchars=\\\{\}]}
\DefineVerbatimEnvironment{Highlighting}{Verbatim}{commandchars=\\\{\}}
% Add ',fontsize=\small' for more characters per line
\usepackage{framed}
\definecolor{shadecolor}{RGB}{248,248,248}
\newenvironment{Shaded}{\begin{snugshade}}{\end{snugshade}}
\newcommand{\KeywordTok}[1]{\textcolor[rgb]{0.13,0.29,0.53}{\textbf{{#1}}}}
\newcommand{\DataTypeTok}[1]{\textcolor[rgb]{0.13,0.29,0.53}{{#1}}}
\newcommand{\DecValTok}[1]{\textcolor[rgb]{0.00,0.00,0.81}{{#1}}}
\newcommand{\BaseNTok}[1]{\textcolor[rgb]{0.00,0.00,0.81}{{#1}}}
\newcommand{\FloatTok}[1]{\textcolor[rgb]{0.00,0.00,0.81}{{#1}}}
\newcommand{\ConstantTok}[1]{\textcolor[rgb]{0.00,0.00,0.00}{{#1}}}
\newcommand{\CharTok}[1]{\textcolor[rgb]{0.31,0.60,0.02}{{#1}}}
\newcommand{\SpecialCharTok}[1]{\textcolor[rgb]{0.00,0.00,0.00}{{#1}}}
\newcommand{\StringTok}[1]{\textcolor[rgb]{0.31,0.60,0.02}{{#1}}}
\newcommand{\VerbatimStringTok}[1]{\textcolor[rgb]{0.31,0.60,0.02}{{#1}}}
\newcommand{\SpecialStringTok}[1]{\textcolor[rgb]{0.31,0.60,0.02}{{#1}}}
\newcommand{\ImportTok}[1]{{#1}}
\newcommand{\CommentTok}[1]{\textcolor[rgb]{0.56,0.35,0.01}{\textit{{#1}}}}
\newcommand{\DocumentationTok}[1]{\textcolor[rgb]{0.56,0.35,0.01}{\textbf{\textit{{#1}}}}}
\newcommand{\AnnotationTok}[1]{\textcolor[rgb]{0.56,0.35,0.01}{\textbf{\textit{{#1}}}}}
\newcommand{\CommentVarTok}[1]{\textcolor[rgb]{0.56,0.35,0.01}{\textbf{\textit{{#1}}}}}
\newcommand{\OtherTok}[1]{\textcolor[rgb]{0.56,0.35,0.01}{{#1}}}
\newcommand{\FunctionTok}[1]{\textcolor[rgb]{0.00,0.00,0.00}{{#1}}}
\newcommand{\VariableTok}[1]{\textcolor[rgb]{0.00,0.00,0.00}{{#1}}}
\newcommand{\ControlFlowTok}[1]{\textcolor[rgb]{0.13,0.29,0.53}{\textbf{{#1}}}}
\newcommand{\OperatorTok}[1]{\textcolor[rgb]{0.81,0.36,0.00}{\textbf{{#1}}}}
\newcommand{\BuiltInTok}[1]{{#1}}
\newcommand{\ExtensionTok}[1]{{#1}}
\newcommand{\PreprocessorTok}[1]{\textcolor[rgb]{0.56,0.35,0.01}{\textit{{#1}}}}
\newcommand{\AttributeTok}[1]{\textcolor[rgb]{0.77,0.63,0.00}{{#1}}}
\newcommand{\RegionMarkerTok}[1]{{#1}}
\newcommand{\InformationTok}[1]{\textcolor[rgb]{0.56,0.35,0.01}{\textbf{\textit{{#1}}}}}
\newcommand{\WarningTok}[1]{\textcolor[rgb]{0.56,0.35,0.01}{\textbf{\textit{{#1}}}}}
\newcommand{\AlertTok}[1]{\textcolor[rgb]{0.94,0.16,0.16}{{#1}}}
\newcommand{\ErrorTok}[1]{\textcolor[rgb]{0.64,0.00,0.00}{\textbf{{#1}}}}
\newcommand{\NormalTok}[1]{{#1}}
\usepackage{longtable,booktabs}
% Fix footnotes in tables (requires footnote package)
\IfFileExists{footnote.sty}{\usepackage{footnote}\makesavenoteenv{long table}}{}
\usepackage{graphicx,grffile}
\makeatletter
\def\maxwidth{\ifdim\Gin@nat@width>\linewidth\linewidth\else\Gin@nat@width\fi}
\def\maxheight{\ifdim\Gin@nat@height>\textheight\textheight\else\Gin@nat@height\fi}
\makeatother
% Scale images if necessary, so that they will not overflow the page
% margins by default, and it is still possible to overwrite the defaults
% using explicit options in \includegraphics[width, height, ...]{}
\setkeys{Gin}{width=\maxwidth,height=\maxheight,keepaspectratio}
% Make links footnotes instead of hotlinks:
\renewcommand{\href}[2]{#2\footnote{\url{#1}}}
\IfFileExists{parskip.sty}{%
\usepackage{parskip}
}{% else
\setlength{\parindent}{0pt}
\setlength{\parskip}{6pt plus 2pt minus 1pt}
}
\setlength{\emergencystretch}{3em}  % prevent overfull lines
\providecommand{\tightlist}{%
  \setlength{\itemsep}{0pt}\setlength{\parskip}{0pt}}
\setcounter{secnumdepth}{2}

% set default figure placement to htbp
\makeatletter
\def\fps@figure{htbp}
\makeatother


% Headers and page numbering
\usepackage{fancyhdr}
% 20170205 changed from \pagestyle{plain}
\pagestyle{fancy}
% added 20170205 -->
\fancyhf{}
\renewcommand{\headrulewidth}{0pt}
\fancyhead[OL]{\leftmark}
\fancyfoot[C]{\thepage}
% added 20170205 <--

% Table package
\usepackage{ctable}% http://ctan.org/pkg/ctable

% Deal with 'LaTeX Error: Too many unprocessed floats.'
\usepackage{morefloats}
% or use \extrafloats{100}
% add some \clearpage

% % Chapter header
% \usepackage{titlesec, blindtext, color}
% \definecolor{gray75}{gray}{0.75}
% \newcommand{\hsp}{\hspace{20pt}}
% \titleformat{\chapter}[hang]{\Huge\bfseries}{\thechapter\hsp\textcolor{gray75}{|}\hsp}{0pt}{\Huge\bfseries}

% % Fonts and typesetting
% \setmainfont[Scale=1.1]{Helvetica}
% \setsansfont[Scale=1.1]{Verdana}

% FONTS
\usepackage{xunicode}
\usepackage{xltxtra}
\defaultfontfeatures{Mapping=tex-text} % converts LaTeX specials (``quotes'' --- dashes etc.) to unicode
% \setromanfont[Scale=1.01,Ligatures={Common},Numbers={OldStyle}]{Palatino}
% \setromanfont[Scale=1.01,Ligatures={Common},Numbers={OldStyle}]{Adobe Caslon Pro}
%Following line controls size of code chunks
% \setmonofont[Scale=0.9]{Monaco}
%Following line controls size of figure legends
% \setsansfont[Scale=1.2]{Optima Regular}

%Attempt to set math size
%First size must match the text size in the document or command will not work
%\DeclareMathSizes{display size}{text size}{script size}{scriptscript size}.
\DeclareMathSizes{12}{13}{7}{7}

% ---- CUSTOM AMPERSAND
% \newcommand{\amper}{{\fontspec[Scale=.95]{Adobe Caslon Pro}\selectfont\itshape\&}}

% HEADINGS
\usepackage{sectsty}
\usepackage[normalem]{ulem}
\sectionfont{\rmfamily\mdseries\Large}
\subsectionfont{\rmfamily\mdseries\scshape\large}
\subsubsectionfont{\rmfamily\bfseries\upshape\large}
% \sectionfont{\rmfamily\mdseries\Large}
% \subsectionfont{\rmfamily\mdseries\scshape\normalsize}
% \subsubsectionfont{\rmfamily\bfseries\upshape\normalsize}

% Set figure legends and captions to be smaller sized sans serif font
\usepackage[font={footnotesize,sf}]{caption}

\usepackage{siunitx}

% Adjust spacing between lines to 1.5
\usepackage{setspace}
% \onehalfspacing
%\doublespacing
\raggedbottom

% Set margins
\usepackage[top=1.5in,bottom=1.5in,left=1.5in,right=1.4in]{geometry}
% \setlength\parindent{0.4in} % indent at start of paragraphs (set to 0.3?)
\setlength{\parskip}{9pt}

% Add space between pararaphs
% http://texblog.org/2012/11/07/correctly-typesetting-paragraphs-in-latex/
% \usepackage{parskip}
% \setlength{\parskip}{\baselineskip}

% Set colour of links to black so that they don't show up when printed
\usepackage{hyperref}
\hypersetup{colorlinks=false, linkcolor=black}

% added 20170130 -->
% to fix URLs not willing to break in bibs
% src https://tex.stackexchange.com/questions/3033/forcing-linebreaks-in-url
\usepackage{url}
\def\UrlOrds{\do\*\do\-\do\~\do\'\do\"\do\-}%
\makeatletter
\g@addto@macro{\UrlBreaks}{\UrlOrds}
\makeatother
% added 20170130 <--

% Tables
\usepackage{booktabs}
\usepackage{threeparttable}
\usepackage{array}
\newcolumntype{x}[1]{%
>{\centering\arraybackslash}m{#1}}%

% Allow for long captions and float captions on opposite page of figures
% \usepackage[rightFloats, CaptionBefore]{fltpage}

% Don't let floats cross subsections
% \usepackage[section,subsection]{extraplaceins}


% Chapter styling
% src: https://github.com/chiakaivalya/thesis-markdown-pandoc/blob/master/preamble.tex#L26
\usepackage[grey]{quotchap}
\makeatletter
\renewcommand*{\chapnumfont}{%
  \usefont{T1}{\@defaultcnfont}{b}{n}\fontsize{80}{100}\selectfont% Default: 100/130
  \color{chaptergrey}%
}
\makeatother


% added 20170205 -->
% make block quote latin again
\let\origquote\quote
\let\endorigquote\endquote
\renewenvironment{quote}{%
    \origquote
    \itshape
}
{\endorigquote}
% added 20170205 <--

\title{Open Specification of a user-controlled Web Service for Personal Data}
\providecommand{\subtitle}[1]{}
\subtitle{Master Thesis}
\author{G. Jahn}
\date{\today}

\begin{document}
\maketitle
\begin{abstract}
Often data is referred to as \emph{the oil of the 21st century}. But
recovering petroleum or coal is governed by certain rules and
legislation, while vast amounts of personal data are harvested from all
sorts of resources with no working restrictions or asking for
permission. Well air-conditioned data centers are mining around the
clock via dozens of CPUs. Eager to discover even the tiniest
correlations worth interpreting in enormous haystacks that are filled
with data belonging to millions of individuals, who have no knowledge of
those computations. But these very computations, and thus their
developers, almost inevitably and without restraints discriminate
against these so called \emph{data subjects}. Such practices cannot be
accepted because these haystacks contain actual identities of human
beings including their personalities but still getting recklessly
monetized. To address this issue and reduce the possibility of
discrimination, third parties have to be supplied with the least amount
of data in order to stay functional. That is, data owners must be in
charge of deciding which entity has access to what personal data of
theirs. This project aims to specify a personal data service that
empowers its user to regain full control over her personal data. It
facilitates the regulation of data flows and provides detailed
information on where the data goes so that a \emph{data subject} can
become her own data broker. In order to trust such a service, the user
must be able to look inside and see for herself if it behaves in
unexpected ways. Therefore the specification and all implementations are
being open sourced and developed transparently in public, which also
encourages self-hosting.
\end{abstract}

% added 20170204 -->
% moved from cover to separate page
\pagenumbering{Roman}
\addcontentsline{toc}{chapter}{Acknowledgement}
\section*{Acknowledgement}
I'm thanking my Haekelschmein and my internet provider. TODO My
other/better half (significant other) for having my back through the
whole time My parents for letting/allowing me to get this fare.
\newpage
% moved from cover 20170204 <--


{
\setcounter{tocdepth}{1}
% added 20170205 -->
% suppress fancyheader
\pagestyle{plain}
% added 20170205 <--
\tableofcontents
}
% added 20170129 -->
\newpage
\addcontentsline{toc}{chapter}{List of Tables}
% added 20170129 <--
\listoftables
% added 20170129 -->
\newpage
\addcontentsline{toc}{chapter}{List of Figures}
% added 20170129 <--
\listoffigures
% added 20170205 -->
% to fix page numbering
\newpage
% added 20170205 <--
% added 20170205 -->
\pagenumbering{arabic}
% added 20170205 <--
\newpage

\begin{center}
\thispagestyle{empty}
\setcounter{page}{0}
This page intentionally left blank
\end{center}\newpage

\chapter{Introduction}\label{introduction}

\section{Motivation}\label{motivation}

Nowadays, it is difficult to find a business that does not collect data
about something; humans are particular targets of choice for the
\emph{Big Data Movement}
{[}\protect\hyperlink{ref-web_2016_privacy-international-about-big-data}{1}{]}.
Since humans are all individuals, they are distinct from each other.
While subsets of individuals might share a minor set of attributes, the
majority is still very unique to an individual, given that the overall
variety of attributes is complex. That small amount of similarity might
seem to be less important, due to the nature of inflationary occurrence,
but the opposite turns out to be true. These similarities allow to
determine the individuals who are part of a subset and the ones who
aren't. Stereotypical patterns are applied to these subsets and thus to
all related individuals. This enriched information is then used to help
predict outcomes of problems or questions related to these individuals.
In other words, searching for causation where in best the case one might
find correlations. This is also known as \emph{discrimination}, which

\begin{quote}
{[}\ldots{}{]} refers to unfair or unequal treatment of people based on
membership to a category or a minority, without regard to individual
merit.
{[}\protect\hyperlink{ref-paper_2008_discrimination-aware-data-mining}{2}{]}
\end{quote}

When interacting directly with each other, discrimination of human
beings is a serious issue in our society, but also when humans leverage
computers and algorithms to uncover formerly unnoticed information in
order to inform their decision making. For example, when qualifying for
a loan, hiring employees, investigating crimes or renting flats. The
decision to approve or deny is based on computed data about the
individuals in question
{[}\protect\hyperlink{ref-book_2015_ethical-it-innovation_ethical-uses-of-information-and-knowledge}{3}{]},
which is merely discrimination on a much larger scale and with less
effort (almost parenthetically). The described phenomenon is originally
referred to as \emph{Bias in computer systems}
{[}\protect\hyperlink{ref-paper_1996_bias-in-computer-systems}{4}{]}.
What at first seems like machines going rouge on humans is, in fact, the
\emph{cognitive bias}
{[}\protect\hyperlink{ref-wikipedia_2016_cognitive-bias}{5}{]} of human
nature, modeled into machine executable language and built to reveal the
patterns their creators were looking for. The \emph{``Inheritance of
humanness''}
{[}\protect\hyperlink{ref-web_2016_big-data-is-people}{6}{]}, so to say.

In addition to the identity-defining data mentioned before, humans have
the habit to create more and more data on a daily basis, both
pro-actively (e.g by writing a post) and passively (e.g by allowing the
app to access their current location while submitting the post). As a
result, already gigantic databases grow ever larger, waiting to be
harvested, collected, aggregated, analyzed and finally interpreted. The
crux here is, the more data being made available
{[}\protect\hyperlink{ref-video_2015_big-data-and-deep-learning_discrimination}{7}{]}
to \emph{mine} on, the higher the chances to isolate datasets (clusters)
that differ from each other but are coherent within themselves. By
defining those datasets, instead of distinguishing on an individual
level, humans are being reduced to these set-defining characteristics in
order to fit in these clusters.

In order to lower potential discrimination, either the responsible parts
in these machines need to be erased while simultaneously raising
awareness and teaching people about this issue of discrimination, or all
the personal data needs to be prevented from falling into these data
silos in the fist place. Although both approaches are valid and should
be pursued simultaneously, the latter will be addressed in this work.

\section{Purpose \& Outcome}\label{purpose-outcome}

At first glance, it might not be considered harmful to provide one's own
personal data to third parties, at least from an individual's
perspective, because free or improved services are eventually offered in
return. For example, more adequate recommendations and fitting
advertisement, or more helpful therapies and more secure environments.
Gathering and processing data is essentially just mathematics and
computer technologies. How those tools are utilized and what purposed
they serve is within the decision of their developers. However, what
data points are used and how they get processed should be determined by
the data creators. Thereby allowing them an influence on the results of
these processes and thus on decisions made upon them which impact their
lives.

To address the described issue, the initial idea here is to (1) equip
individuals with the ability to control and maintain their entire data
distribution, in order to (2) reduce the amount of \emph{potentially
discriminatory}
{[}\protect\hyperlink{ref-paper_2008_discrimination-aware-data-mining}{2}{]}
attributes that could leak into arbitrary computations. For that, people
need a reliable and trustworthy tool, which helps them to manage all
their \emph{personal data} and provides an interface for third parties
to access their data, but on their own terms. The parties that would be
responsible for such a tool would likely have the most accurate and
reliable one-stop resource to an individuals' \emph{personal data} at
hand, while simultaneously being urged to respect their privacy. This
approach comes along also with some downsides related to security and
potential data loss. Elaborating on these issues and discussing
potential solutions is part of the
\protect\hyperlink{design-discussion}{design process}.

This way of addressing the described dilemma about personal data
analysis is not new (see \protect\hyperlink{related-work}{Related
Work}). Early work done in this field can be dated back to the Mid-2000s
when studies were conducted, for example, about recent developments in
the industry (e.g targeted ads) and the user's concerns about privacy
{[}\protect\hyperlink{ref-study_2004_architecture-for-privacy-sensitive-ubiquitous-computing}{8}{]}.
At that time, the term \emph{Vendor Relationship Management(VRM)} was
first used within the context of user-centric personal data management,
which then also led into the \emph{ProjectVRM}
{[}\protect\hyperlink{ref-web_2010_projectvrm_about}{9}{]} started by
the \emph{Berkman Klein Center for Internet \& Society at Harvard
University}. A great amount of effort has gone into this area of
research since then. While commercial products and business models try
to solve some of the problems related to this. For instance, with
concepts like the \emph{Personal Data Store (PDS)}
{[}\protect\hyperlink{ref-paper_2013_the-personal-data-store-approach-to-personal-data-security_2013}{10}{]}
or an implementation of the \emph{MyData} concept
{[}\protect\hyperlink{ref-whitepaper_2014_mydata-a-nordic-model-for-human-centered-personal-data-management-and-processing}{11}{]}
called \emph{Meeco}
{[}\protect\hyperlink{ref-web_2016_meeco-how-it-works}{12}{]}, which are
all be covered in detail in the following chapter.

The research work done for this thesis constitutes the foundation for an
\emph{Open Specification}, which is a manual for implementing a concept
called \emph{Personal Data as a Service}, henceforth called
\emph{PDaaS}. Examining important topics like the architecture, where
data can be stored, how to obtain data from the exposed API or what
requirements have to be met by a user interface for personal data
management, is part of this work. By the time this document has been
submitted, most of the core issues should have already been addressed
and can thus be outlined in a first draft of the specification. Only
then can the task of actually implementing certain components begin. The
reason for that is, when sensitive subjects, especially things like
people's privacy, are at risk, all aspects in question deserve careful
consideration, so they can be addressed properly. That is, adequate
effort must be put primarily into the theoretical work. However, that
does not mean writing code to test theories and ideas isn't allowed
during the development and specification process. It is encouraged and
might even help to spot flaws or perhaps trigger improvements of the
specification.

To create confidence in this project and in the software built upon it,
it's vital to make all development processes fully transparent and
encourage people to get involved. For this reason all related software
and documents {[}\protect\hyperlink{ref-repo_2016_pdaas-spec}{13}{]} are
open source from day one.

In summary, this document lays the ground work and is intended to be the
initial step in a development process fabricating a tool to manage
personal data. The tool is controlled and administrated by the
individual to whom the personal data belongs. It enables her to get a
more precise understanding of what data is accessed by whom and how this
might affect her privacy.

\hypertarget{terminologies}{\section{Terminologies}\label{terminologies}}

\begin{description}
\tightlist
\item[Web Service:]
A service, that is accessible by electronic devices over the internet.
This makes it almost effortless to use a service which otherwise would
be out of reach. Interactions with a service usually happen through
enriched websites or other web-compatible applications and interfaces.
\item[Open Specification:]
A specification is a formal and very detailed way of describing a
technology, its internals and behaviour from external perspectives. It
provides guidance for implementations to ensure a minimum level of
interoperability. Structured in a formalized document it might become a
\emph{technical standard}. \emph{Open} means here at first nothing but
it's accessible for anyone without restrictions. When it comes to the
intellectual value itself, that might be handled differently, for
example with an enclosed license.
\item[\protect\hypertarget{terminologies--profile-data}{}{Profile
Data}:]
A collection of data points reflecting an individual's inherent
information and other basic predominantly static data points (no sets),
which in conjunction uniquely relate to that individual.
\item[\protect\hypertarget{terminologies--digital-footprint}{}{Digital
Footprint}:]
Refers to data that is related to an individual. It is distinguished
between an active footprint, which involves data and information about
an individual who chose to share them publicly, and a passive footprint,
which includes all data about an individual collected by third parties
without the individual's knowledge.
\item[Personal Data as a Service (PDaaS):]
A web service controlled, owned and maybe even hosted by an individual.
It provides access to the data subject's personal data and offers
maintainability as well as permission management for those data. It can
be seen as her personal agent; sometimes also referred to as \emph{the
system}.
\item[Data Subject:]
An individual who first and foremost is the owner of all of her personal
data; sometimes referred to as \emph{owner}.
\item[\protect\hypertarget{terminologies--operator}{}{Operator}:]
A \emph{data subject} that uses a \emph{PDaaS} to control (and probably
host) her personal data; sometimes referred to as \emph{data controller}
or \emph{data owner}.
\item[\protect\hypertarget{terminologies--consumer}{}{(Data) Consumer}:]
Third party who requests permission or is already allowed to access the
\emph{operator's} \emph{personal data} through her \emph{PDaaS};
sometimes referred to as \emph{(data) collector} or vendor.
\item[\protect\hypertarget{terminologies--data-broker}{}{Data Broker}:]
Third party with commercial interests in collecting, aggregating and
analyzing information/data about humans from any possible resource in
order to combine and enrich the data, to finally license those corpora
to other organisations.
{[}\protect\hyperlink{ref-report_2014_data-brokers}{14}{]}
\item[Permission Request:]
A formalized attempt made by a third party to request permissions in
order to access certain data points on the \emph{PDaaS}. The request has
to include all the data points that are demanded to being accessed as
well as sufficient information about the purpose. It requires the third
party to already being registered as \emph{data consumer}.
\item[Access Request:]
An attempt to actually access data provided by a \emph{PDaaS}. The
request primarily consists of a query, that defines what data points are
tried to be accessed. The access is only permitted if the query matches
against the \emph{permission profiles}.
\item[Permission Profile:]
A set of access rules and configuration tied to a \emph{data consumer}.
It determines how long and what data is accessible by the related
\emph{data consumer}. The profile is the result of a reviewed and
granted \emph{permission request}.
\item[Endpoint:]
An endpoint is defined as part of a URI that uniquely associates to
exactly one single \emph{data consumer}. Usually it's the first part of
a URI (e.g.~domain incl. subdomains), whereas following parts indicating
different resources that might be available within that endpoint. It can
also be viewed as group of resources whose access is restricted.
\end{description}

\hypertarget{scenarios}{\section{Scenarios}\label{scenarios}}

The following use cases portray different situations and possible ways
in which the tool in question might be applicable, and shows in several
ways that it can be helpful to be in charge of its own personal data.
Some are more practical and realistic, like ordering and purchasing a
product on the internet, while others might at the moment not seem to be
very useful, but show a certain potential to become more relevant when
new technologies and business models will occur, who are followed by new
desires for data.

\subsubsection{Ordering a product on the
web}\label{ordering-a-product-on-the-web}

The data subject searches through the web to find a new toaster, because
her old one recently broke. After some clicks and reviews, she found her
soon-to-become latest member of the household's kitchenware. After
putting the model name in a price search engine, hoping to save some
money, the first entry, offering a 23\% discount, caught her attention.
She decides to have a deeper look into the toasters, therefore she is
headed towards the original web shop entry. Finally she comes around and
puts the item onto her card, despite the fact that she never bought
something from that online shop before. Then she proceeds to checkout so
that she can place her order. The shop-interface asks her to either
insert her credentials, proceed without registration or sign-in, or
allow the shop to obtain all required data on its own by either scanning
a QR-Code displayed below or insert a URI to her \emph{Personal Data as
a Service}. She opens up the management panel of her \emph{PDaaS} in a
new browser window and authenticates herself to the system. Afterwards
she creates a new entry in a list of \emph{data consumers} who already
get permitted to access certain characteristics of her personal data. As
a result, she gets prompted with a URI, which she inserts as the shop
interface requests her to do, only after she has convinced herself that
the data exchange with the shop is based on a secure connection (HTTPS).
Moving on to the next step after submitting the URI, the data subject is
ask to decide how she would like to pay. The choices are: credit card,
invoice, online payment or bank transfer. She chooses the last one,
submits her selection and thereby completes the order process. She goes
back to the kitchen. After some time, a push notification appears on her
mobile device lying around. The notification is about a \emph{permission
request} which has just arrived at her \emph{PDaaS}, asking for granting
permissions to the shop-system, she places the order before. The shop
wants to access her full name, address and email, which are required to
proceed with the order. Based on the information given in the request,
she creates a new \emph{permission profile} for the shop. Additionally,
for the profile she can decide between three states of how long the
permission is going to last: \emph{only one time}, \emph{expires on
date} and \emph{granted, until further notice}. Since she has never
ordered at this shop before and probably won't do it again, she decides
to grant access only for this specific occasion. The shop-system gets
then notified about the result of decisions. If the result is positive -
which is the case here - the data can be obtained and the order can be
further processed. As a result, the data subject receives an email,
containing information regarding her order including the shop owner's
bank details, which enables her to pay the due amount. After the
shop-system receives the payment, the toaster is shipped.

In order to get a full impression of how the whole process might would
look like if the data subject would have chosen one of the other payment
methods, the differences are describes below. If the data subject would
have wanted to pay with her credit card, the shop-system would have
asked to also access her credit card as well as its belonging secret.
And when sending the email, the system would have omitted the
information about the shop's bank details. Paying with invoice, would
have been possible only if the \emph{PDaaS} initially had been able to
provided certified profile data, which therefore would have been rated
trustworthy. That again would have reduced the risk taken by shop owner
and would have enabled him to take action in cases of fraud or misuse.
Choosing to involve an online payment service provider as a
\emph{middleman} for processing the payment, would have required the
data subject to have granted the provider certain access to her
\emph{PDaaS} upfront. In that case, the shop-system then would have
asked for her payment provider account identifier, so that the system
could have requested the payment directly form that payment service
provider. This on the other side would have caused the service provider
to consult the \emph{PDaaS}, which would have result in a second
notification asking the data subject for permission to proceed. After
the payment transfer would have been succeeded, the shipment would have
been initiated.

\subsubsection{Interacting with a social
network}\label{interacting-with-a-social-network}

Entering a social network for the first time only requires either a URI
to the data subject's \emph{PDaaS}, which has uniquely been generated
for that purpose, or a QR-Code provided by the social network. The data
subject receives a notification on her mobile device send from her
\emph{PDaaS}, revealing what data that network wants to access and maybe
even why. If her mobile device is currently not at hand, she can also
use the management panel provided by her \emph{PDaaS}, which is
accessible with a web browser on every internet-enabled device. Within
that panel pending permission reviews are indicated. Regardless of
whether the data subject has already reviewed the request, she should
still be able to login. After doing so, she would see all her
information, unless she has not yet granted permissions to the social
network to access her data \emph{until-further-notice}. If this is done,
after waiting a moment and then reloading the browser session, all her
data should then show up. So, every time if someone on that network
tries to access her information, whom she has allowed to see those
information, which is managed by every user only from within the
network, the network pulls the required data from her \emph{PDaaS}, as
long as it is permitted to do so. It is also conceivable, that the
social network does provide a back-channel to the \emph{PDaaS}, so that
all the content she creates within that network including all
interactions with other users can be stored in her \emph{PDaaS}, so that
it could be provided to other \emph{data consumers}. The network itself
only stores references all these content objects. Whether it's for
example an image, a post or comment on a post from somebody else, if the
actual content is needed to be displayed it gets fetched from the
corresponding \emph{PDaaS}.

\subsubsection{Applying for a loan and checking
creditworthiness}\label{applying-for-a-loan-and-checking-creditworthiness}

The data subject would like to buy an apartment. In order to finance
such a acquisition, she needs a funding, which in her case, is based on
a loan. During a conversation in a credit institute of her choice, an
account consultant describes to her what data is required in order to
decide on her creditworthiness. While nodding consensually, she takes
out her smartphone and brings up the management panel of her
\emph{PDaaS}. The consultant flips his screen showing a QR-Code and the
\emph{data subject} scans it. The tool displays some information about
the institute including a reference to this assignment and a list of all
different data points the institute would like to access in order to
calculate her scoring, such as address, monthly income, relationship
status and family, history of banking or other current loans. After some
back and forth and solving some misunderstandings with the help of her
consultant, she decided to just partially allow access to the requested
data and just for this time and purpose. The consultant kindly pointed
out, that these decisions might have an impact on her scoring and thus
on the lending and its terms. While handling some more formalities and
talking about other possible products she might be interested in, the
consultant gets notified by his computer confirming the access
permission. Thereupon the two finishing their meeting and the consultant
informs the data subject about the next steps, which include a note
about contacting her within the next few days when the institute has
come to a conclusion. In case of a positive outcome a new appointment
for handling all the paperwork and signing the contract is needed to be
made. From a technical point of view, two different ways of computing
the score are imaginable. The first one would be, to just transfer the
plain data including expire date and information about how reliable the
data is. However, the actual computations and analytics to obtain the
score, happen within the infrastructure of the credit institute. When
this process has finished, all the personal data that have been
transferred must be erased. An alternative, though, could prevent the
data from leaving the \emph{PDaaS}. Therefor, the institute's request
won't contain a data query. Instead it comes along with some software
and information on how to run it. The \emph{PDaaS} server will provide
an isolated runtime in which the software then can get executed. After
that process is finished, the result is send back to the credit
institute's infrastructure.

\subsubsection{Maintain and provide it's own medical
record}\label{maintain-and-provide-its-own-medical-record}

Some time ago on a hiking trip in a moment of carelessness the data
subject accidently broke her leg. She came into a hospital and went
straight into surgery, where the surgeon where able to fix the injury.
Time went by and the leg healed completely. After she woke up today she
felt some pain coming from the area where her leg was broken. She
decided to call in sick and went straight to a doctor nearby. During her
recovery she has been visiting that doctor regularly. At the reception
desk, she opens up the \emph{PDaaS}'s management panel on her smartphone
and searches through the list of data consumers. After she has found the
entry for this clinic, she flips her phone to show the receptionist the
corresponding QR-Code, which the receptionist starts to scan
immediately. However the receptionist was not able to see any data on
the screen, because the access has already been expired. The data
subject only permitted access for the estimated time of recovery, which
is already over. That's why she got a notification asking about
re-granting some access. While\\
going through the data points the clinic-system has requested, she
notice that her address is incorrect. Last month she moved out into a
bigger apartment just across the street. She must have forgotten to
change that data. She immediately corrects the address right before
saving the \emph{permission profile} for the clinic-system. She also
includes access to all the data originating from that time after her
accident. A moment later the receptionist confirms to now being able to
see all necessary data. The data subject takes a seat in the waiting
room. While passing some time, she decides to take a deeper look into
her list of data consumers. Some of them she couldn't even remember and
for others she was surprised on what data points she has granted access
to them. She starts to restrict certain permissions, if it was
appropriate in her opinion. She even removed some of the entries
entirely. The appointment with her doctor went great. He even had to
review the x-ray images in order to make a adequate differential
diagnosis. After the visit, she makes a quick stop at a pharmacy along
the way to pickup the drugs her doctor had prescribed for her to reduce
the pain. She has to wait in the queue with two other customers being in
front of her. She realizes, that it's the first time she here in this
store. So she prepares a new entry in her list of data consumer,
including all information about her prescriptions. By the time she gets
served, she just let the person behind the register scan her QR-code. In
the next seconds the data subject gets a notification about a
\emph{permission request} from this store, which she quickly reviews and
confirms by making sure that the permissions in that profile are the
ones she prepared minutes ago. A moment later the pharmacist comes back
with her drugs, which she then pays in cash. The transaction is done and
the data subject leaves the store.

\subsubsection{Vehicle data and
mobility}\label{vehicle-data-and-mobility}

Assuming a car itself has no hardware on board to establish a wireless
wide area connection to an outside access node. Only from the inside the
car one is able to connect to it (wired or wireless). After entering a
car, on the data subject' mobile device a notification comes up from the
car asking for permission to connect that mobile device. In addition to
the expiration date, the data subject is provided with two additional
options, which she can en- or disable. First, a wifi network provided to
everyone in the car can be enabled, which utilizes the uplink from the
mobile device to the internet. Secondly, the car is allowed to use the
uplink for opening up connections so it can emit or receive data from
the internet. As a result the device owner gains full control over any
data the car might want to transfer. And this again would allow two
things: (A) permission management for all outgoing data and (B) funnel
all data generated and provided by the car towards the \emph{PDaaS} that
associated with the linked device. It might be feasible as well to deny
certain connections the car tries to open. Data will then be stored only
in the \emph{PDaaS}. If a third party is interested in that data they
have to ask for access permission. That same concept of movement
tracking and vehicle data aggregation could be applied to driving
motorcycles and bicycle as well.

\chapter{Fundamentals}\label{fundamentals}

The following chapter shall provide basic knowledge of concepts like
\emph{Personal Identity}, \emph{Big Data} and \emph{Data Ownership}. It
also explains what \emph{Personal Data} is from a legal standpoint and
covers some of the issues caused by different legislation colliding
through the one global internet. This again requires an elaboration on
how \emph{Personal data} impacts our society as well as the economy.
Overall, this chapter is meant to facilitate a common understanding of
the stated issue and why it could be addressed as described later on.
Furthermore, it is summarized what research has already been done and
what is its current state. Finally, it gives a brief overview of
standards and technologies that might going to be utilized for this
project.

\hypertarget{digital-identity-personal-data-and-ownership}{\section{Digital
Identity, Personal Data and
Ownership}\label{digital-identity-personal-data-and-ownership}}

A \textbf{\protect\hypertarget{def--digital-identity}{}{Digital
Identity}} is viewed as a non-physical abstraction of an entity, such as
an organisation, an individual, a device or even some software. It is
bidirectional associated to its physical counterpart. In the context of
this work, it only refers to human beings. Therefore a Digital Identity
is the representation of an individual in digital systems, consisting of
identity-defining data, such as \emph{personal information}, its own
history and its preferences
{[}\protect\hyperlink{ref-whitepaper_2012_the-value-of-our-digital-identity_definition}{15}{]}.
\emph{Personal information}, in this case, refers to inherent (e.g.~date
of birth) as well as imposed (e.g.~credit card number) characteristics.
The individual to whom those data relates to, is therewith the owner of
that\\
\emph{Digital Identity}. From a technical standpoint, a \emph{Digital
Identity} is essentially a collection of characteristics, attributes and
time series data (e.g.~interaction logs or bank transfer history). Based
on a subset of those attribute values, a unique fingerprint can be
easily generated. Depending on the data point and complexity of its
value, either a unique identifier on its own (e.g.~social security
number) - depending on the context - or only a few are also enough to
generate such fingerprint. Hence it doesn't take a complete set of
attributes including all its values, but rather just a fraction of a
\emph{Digital Identity} in order to determine its rightful owner and
physical counterpart. The \emph{Digital Identity} can be viewed as an
avatar in digital environments or even as the digital part of a
persons's identity. That is, a \emph{Digital Identity} of a living
individual can't just be reduced to some bits and bytes, instead it
should be valued as an appropriate, and maybe even legal, representation
in certain contexts and for a variety of purposes. In some of those
situations it might be required (e.g.~administrative purposes) to ensure
a certain level of authenticity for a \emph{Digital Identity} or for
particular attributes of it. This means, to provide reliable
confirmation that the attribute values are really the ones that belong
to exactly that individual they pretend to belong. An independent third
party, who is trusted by all entities participating in such a construct,
could somehow verify (or vouch) the subject in question. On the other
hand this concept opens up at least on class of attack scenarios. The
risk of identity theft for example increases dramatically, when
assigning such value to a \emph{Digital Identity}, because the attacker
is now longer required to be physically present in order to impersonate
that identity or ``steal'' certain unique identifiers from person.
Instead it's sufficient to gain access to the areas where those
sensitive data is stored. It is noted, that different technical
solutions to these issues do exist and will be discussed later on.

In the context of this project, and all related work,
\textbf{\protect\hypertarget{def--personal-data}{}{Personal Data}} is
defined as the total amount of data that is part of either an
individual's \emph{Digital Identity} or its
\emph{\protect\hyperlink{terminologies--digital-footprint}{Digital
Footprint}}. On the one hand. that includes all the intellectual
property (e.g.~posts, images, videos or comments) ever created, and all
kinds of tracking data, interaction monitoring and metadata, that is
used to manually or automatically enriched content (e.g.geo-location
attached to a tweet as meta information). Moreover, it refers to data
that is captured by someone or something from within the individual's
private living space or property. \emph{Personal data} is basically
understood as every data point reflecting the individual as such and its
personality - partially or as a whole.

The european ``Data Protection Regulations'' on the other hand defines
\emph{\protect\hypertarget{def--personal-data-as-of-legis}{}{Personal
Data}} as follows:

\begin{quote}
`personal data' means any information relating to an identified or
identifiable natural person (`data subject'); an identifiable natural
person is one who can be identified, directly or indirectly, in
particular by reference to an identifier such as a name, an
identification number, location data, an online identifier or to one or
more factors specific to the physical, physiological, genetic, mental,
economic, cultural or social identity of that natural person;
{[}\protect\hyperlink{ref-regulation_2016_eu_general-data-protection-regulation_definition}{16}{]}
\end{quote}

Whereas the U.S.A. has little to no legislation defining \emph{personal
data} and thereby protecting the individual's privacy. There is at least
no explicit federal law addressing such areas
{[}\protect\hyperlink{ref-web_2016_wikipedia_information-privacy-law_us}{17}{]}.
Though, some of the existing sectoral laws contain partially applicable
policies and guidelines
{[}\protect\hyperlink{ref-web_2016_data-protection-laws-in-the-us}{18}{]};
most of them addressing only data related to specific topics
(e.g.~health insurance, financial record or lending). In 2015 the White
House has attempted to fill that gap with the ``Consumer Privacy Bill of
Rights Act'', but to this date it doesn't left the draft state.
According to the critics, the bill lacks of concrete enforceable rules
on which consumers can rely on
{[}\protect\hyperlink{ref-web_2015_white-house-releases-consumer-privacy-bill-draft}{19}{]}.
The draft contains a general definition of \emph{Personal Data}:

\begin{quote}
``Personal data'' means any data that are under the control of a covered
entity, not otherwise generally available to the public through lawful
means, and are linked, or as a practical matter linkable by the covered
entity, to a specific individual, or linked to a device that is
associated with or routinely used by an individual, including but not
limited to {[}\ldots{}{]}
{[}\protect\hyperlink{ref-bill-draft_2015_us_consumer-privacy-bill-of-rights-act_definition}{20}{]}
\end{quote}

At the end a list of concrete data points follows; such as email or
postal address, name, social security number and alike.

Aside from legislation through the government, several third-party
organisations in the U.S. are also allowed to define rules and policies
that can overwrite existing laws, namely the \emph{Federal
Communications Commission (FCC)}, which recently released ``Rules to
Protect Broadband Consumer Privacy'' for ISPs\footnote{Internet Service
  Provider} including a list of categories of sensitive information
{[}\protect\hyperlink{ref-rules_2016_fcc_to-protect-broadband-consumer-privacy_sensitive-types-of-data}{21}{]}.
Thereby the FCC wants \emph{Personally Identifiable Information} (alias
\emph{Personal Data}) to be understood as:

\begin{quote}
{[}\ldots{}{]} any information that is linked or linkable to an
individual. {[}\ldots{}{]} information is ``linked'' or ``linkable'' to
an individual if it can be used on its own, in context, or in
combination to identify an individual or to logically associate with
other information about a specific individual.
{[}\protect\hyperlink{ref-rules_2016_fcc_to-protect-broadband-consumer-privacy_personally-identifiable-information}{22}{]}
\end{quote}

Despite minor difference in details, the FCC has s serious idea of what
includes \emph{personal data} and to whom their belong. Although the
FCC's legal participation might be somewhat debatable regarding
limitation to certain topics, the ``Communications Act'' as a U.S.
federal law qualifies the FCC to regulate and legislate within its
boundaries.

Having a common understanding on what data points belong to a person is
the foundation of defining a set rules on how to handle \emph{Personal
Data} appropriately. Hence, every business, operating within the EU, is
required\footnote{according to article 12-14 of the ``EU General Data
  Protection Regulation 2016/679''} to provide its users with a
\emph{Privacy Policy}, while for example in the U.S. - as mentioned
above - only infrequently and depending on the class of data or context
users, must be informed about how their data get processed and what data
points are involved
{[}\protect\hyperlink{ref-web_2016_privacy-policies-are-mandatory-by-law}{23}{]}.\\
A user typically agrees on a \emph{Privacy Policy} by starting to
interact with the author's business or platform. Thus every
\emph{Privacy Policy} is required to be publicly accessible; for
instance not in a restricted area after logging, but before creating an
account; like in the following example showe, taken from facebook's
current landing page.

\begin{quote}
By clicking Create an account, you agree to our
\href{https://www.facebook.com/legal/terms}{Terms} and that you have
read our \href{https://www.facebook.com/about/privacy}{Data Policy},
including our \href{https://www.facebook.com/policies/cookies/}{Cookie
Use}.
{[}\protect\hyperlink{ref-web_2016_facebooks-landing-page_policy-acknowledgement}{24}{]}
\end{quote}

It can be viewed more like an information notice, which translates and
specifies the prevailing legal situation, rather then a contract, which
the user would be forced to read and accept before revealing her data;
otherwise known from procedures like software installations, where the
user might have to accept terms of use or license agreements. With a
\emph{Privacy Policy} at hand, it's up to each individual to decide, if
the benefits, the service offers, are worth sharing part(s) of her
\emph{personal data}, while at the same time reluctantly tolerating
potential downsides concerning the privacy of that data. When the vendor
considers its policies being accepted by a user, her personal data must
be processed as stated in that policies but most certainly according to
the law. If the policies violate existing law or the vendor effectively
goes against the law with its actual doing, penalties might follow.
Depending on level and impact of the infringement in addition to what
the law itself says, the vendor, while revisiting its wrong-doings in
order to improve, might have to compensate affected individuals, pay a
fine or get revoked its license.

Not only privacy laws, but every legal jurisdiction has it's limitations
- such as its territorial nature - which makes it not exactly an
appropriate tool for addressing existing issues and strengthen
individuals privacy and rights in a global context like the \emph{world
wide web}. Whereas the EU has approved an extensive regulation,
mentioned above, that is supposed to provide privacy protection and
defines the handling of personal data, the U.S. on the other hand has
only subject-specific rules which merely apply to its own citizens. Even
though, the definition of personal data included in the EU regulation is
almost identical to the one introduced for the context of this work, it
does only apply to vendors and individuals who are part of the EU. Even
privacy policies won't help if the vendor is registered in a different
area of jurisdiction than the user is located. For those circumstances
international agreements might be established
{[}\protect\hyperlink{ref-web_2016_international-privacy-standards}{25}{]},
but this approach might still be useless if it either provides no proper
tools for users to enforce their rights or is simply being ignored by
contract partners with or without legal foundation
{[}\protect\hyperlink{ref-web_2017_privacy-shield_faq}{26}{]}
{[}\protect\hyperlink{ref-web_2017_privacy-shield_kritik}{27}{]}.

While the legislation mentioned above is in place,
\emph{\protect\hypertarget{def--ownership}{}{Ownership}} of
\emph{Personal Data} has no legal basis what so ever. The concepts of
intellectual property protection and copyright might intuitively be
applicable, because the data that is defined through the soul existence
of the \emph{data subject} (Digital Identity) and the data that is
created by her, seems to be her \emph{intellectual property} as weill.
Such type of property implies to be a result of a creative process
though, but unfortunately facts, like \emph{personal data} mostly are,
don't show a \emph{threshold of originality}
{[}\protect\hyperlink{ref-paper_2014_who-owns-yours-data}{28}{]}. Thus,
the legal concept of \emph{intellectual property} does not apply.
However, \protect\hypertarget{def--ownership}{}{Ownership} in the
context of this work, is understood as a concept of having exclusive
control over its personal data and how that data get processed at any
given point in time. The exclusive control is emphasizing as (A) the
right to do what ever is desired with its property and (B) by which
rules and mechanisms the ownership can be assigned to someone
{[}\protect\hyperlink{ref-book_1987_private-ownership_definition}{29}{]}.
This might result in a logistical challenge in which the data subject
has to allow data access without loosing the exclusive control over that
data. In any case some effort might be required in order to preserve
ownership as described, caused by the general characteristics that data
has.

The european ``Data Protection Regulations'' mentioned before indicates
only one occurrence of the word \emph{ownership}, and it's not even
related to the context of \emph{personal data} or the \emph{data
subject}. It regulation only stats, that \emph{``Natural persons should
have control of their own personal data''}
{[}\protect\hyperlink{ref-regulation_2016_eu_general-data-protection-regulation_ownership}{30}{]}.
Whereas Commissioner J. Rosenworcel of the FCC wants \emph{``consumers
{[}\ldots{}{]} to {[}\ldots{}{]} take some ownership of what is done
with their personal information.''}
{[}\protect\hyperlink{ref-rules_2016_fcc_to-protect-broadband-consumer-privacy_ownership}{31}{]}
Despite those two exceptions, elaborations on \emph{data ownership} is
almost not existent in current legislation. Instead the question is
typically addressed in \emph{Terms of Service (ToS)} provided by
\emph{data consumers}, which an individual might have to accept in order
to establish a (legal) relationship with its author. The individual
should keep in mind, that \emph{Terms of Services} can change over time;
not necessarily to the users advantage. The contents of a \emph{ToS}
must not violate any applicable or related law, otherwise the terms
might not be legally recognized. Taking for example the following
excerpts from different \emph{Terms of Services}:

\begin{quote}
You own all of the content and information you post on Facebook, and you
can control how it is shared {[}\ldots{}{]}. (under ``2. Sharing Your
Content and Information'', by Facebook
{[}\protect\hyperlink{ref-web_2016_facebook_terms-of-service}{32}{]})
\end{quote}

\begin{quote}
You retain your rights to any Content you submit, post or display on or
through the Services. What's yours is yours --- you own your Content.
(under ``3. Content on the Services'', by Twitter
{[}\protect\hyperlink{ref-web_2016_twitter_terms-of-service}{33}{]})
\end{quote}

\begin{quote}
Some of our Services allow you to upload, submit, store, send or receive
content. You retain ownership of any intellectual property rights that
you hold in that content. In short, what belongs to you stays yours.
(under ``Your Content in our Services'', by Google
{[}\protect\hyperlink{ref-web_2016_google_terms-of-service}{34}{]})
\end{quote}

\begin{quote}
Except for material we may license to you, Apple does not claim
ownership of the materials and/or Content you submit or make available
on the Service (under ``H. Content Submitted or Made Available by You on
the Service'', by Apple
{[}\protect\hyperlink{ref-web_2016_apple-icloud_terms-of-service}{35}{]})
\end{quote}

All those statements are essentially superseded by a subsequent
statement within each \emph{ToS}, stating that the user grants the
author a worldwide license to do almost any imaginable thing with her
data. In case of Apple for example, if the user is \emph{``submitting or
posting {[}\ldots{}{]} Content on areas of the Service that are
accessible by the public or other users with whom {[}the user{]} consent
to share {[}\ldots{}{]} Content''}
{[}\protect\hyperlink{ref-web_2016_apple-icloud_terms-of-service}{35}{]}.
It is worth to be noted though, that every \emph{ToS} only refers to the
content created by the \emph{data subject} instead of all her personal
data. As mentioned above, personal information are no subject of
intellectual property, but playing an important role in data analytics
though. Which is why \emph{privacy policies} are in place, to ensure at
least some enlightenment on the whereabouts of the user's personal data,
even though it doesn't compensate the absent of control. Additionally,
neither the meaning of \emph{ownership}, to which the quoted terms refer
to, is sufficiently outlined, which results in ambiguity, and thus
leaves room for interpretation, nor the proposed definition of
\emph{ownership}, \protect\hypertarget{def--ownership}{}{as described
earlier}, is applicable to these \emph{Terms of services}, since the
\emph{data subject} looses all its control by design. Handing over data
to a consumer effectively disables the exclusive control over the data
and eliminates the ability to assign such control. Hence, the
\emph{ownership} to that data doesn't exist any further. That is, no
(legislation based) way exists to establish a feasible concept of
\emph{ownership}, unless the data consumer has a motivation to promote
the \emph{data subject} to the sole owner of her data as well as honours
these privileges.

Leaving the legislation aside for a moment to move away from the top
view; data consumers might argue, that they have invested a lot in order
to enable themselves to collect, process and store personal data, hence
it belongs to them. Whereas from the data subject's point of view this
might only be acceptable as long as she herself benefits from that
arrangement somehow. Which would be the case if for example the data
subject uses products, services or features, offered by consumers, whose
quality depends on personal data. If the data subject then chooses to
move to a competitor, she would want to bring her personal data with
her. But then again the former data consumer would object that
competitors could in this way benefit from all the investment the
consumer has already made, but without any effort. Not entirely wrong
though, two aspects need to be emphasize here. In order to get high
quality analytics and therefore being able to make accurate decisions to
gain overall improvement, it is (A) vital to put a huge amount of effort
in developing the underlying technologies, rather then primarily
collecting masses of personal data. But this effort is just the
precondition and applies to all of its customers, which again weakens
its argumentation. It is followed by (B) an ongoing and recurring
process of collecting, aggregating and analyzing actively and passively
created data and metadata (e.g.~food deliver history or platform
interactions and tracking). According to the definition of
\emph{\protect\hyperlink{def--personal-data-as-of-legis}{Personal Data}}
through legislation, it appears to be only a fraction of the data,
namely \emph{personal information}, that is involved in these kind of
processes. The larger part, which is defined as the subject's
\emph{\protect\hyperlink{terminologies--digital-footprint}{digital
footprint}}, consists of highly valuable metadata
{[}\protect\hyperlink{ref-web_2013_why-metadata-matters}{36}{]}
{[}\protect\hyperlink{ref-web_2016_why-you-need-metadata-for-big-data-to-success}{37}{]},
but is not covered by law. If the \emph{data subject} ends the
relationship with a collector, at least the personal information should
be erased and every remaining data sufficiently anonymized or even
handed back. Which again can currently be enforced only by legislation,
because the \emph{data subject} has no access to the collector's
infrastructure where the data about her is stored.

In summary, approaching the issues from a legislative angle has showed
to be fraught with problems on a variety of levels. Not least because it
can hardly be proven that vendors behave accordingly. Thus \emph{data
subjects} should not depend on the collector's willingness to respect
\emph{Ownership} of \emph{Personal Data} as stated here. Instead a
technical approach is proposed to embraces the \emph{data subject} as
the origin of her personal data and to proactively regaining control,
rather then just relying.

\section{Personal Data in the context of the Big Data
Movement}\label{personal-data-in-the-context-of-the-big-data-movement}

The term \textbf{\protect\hypertarget{def--big-data}{}{Big Data}}
initially has been referred to as a \emph{huge amount of data}
containing more or less structured datasets
{[}\protect\hyperlink{ref-web_2016_oxford_definition_big-data}{38}{]},
whose size, over time, has started to limit its usability, because it
exceeded the capabilities of state-of-the-art standalone computer
systems and storage-capacity. But instead of reducing the overall size,
the issues are addressed by utilizing distributed storing and parallel
computing. Aside from challenges in logistic and resource management
when for example information retrieval needs to get automated on a large
scale
{[}\protect\hyperlink{ref-web_2016_wikipedia_definition_big-data}{39}{]},
this strategy still doesn't answer the question of how to extract useful
information from such deep ``data lakes''. What questions need to be
asked to get answers whose usefulness has yet to be known of? To
discover knowledge in order to back decision-making processes,
technologies from the fields of \emph{data mining}, \emph{artificial
intelligence} and \emph{machine learning} (e.g.~neural networks) have
been adapted. All in conjunction this is nowadays known as \emph{Big
Data (Analytics)}. Additionally it is a collective term for the practise
or approach as here described, as well as the philosophy of massively
collecting data while tending to neglect people's privacy.

Big Data analytics serve the prior purpose of extracting useful
information, whose result depends on the question initially being asked
as well as what datasets the corpus contains. General steps involved in
such a knowledge discovery process, can be outlined as follows
{[}\protect\hyperlink{ref-chapter_2007_the-knowledge-discovery-process}{40}{]}
{[}\protect\hyperlink{ref-paper_2009_a-data-mining-knowledge-discovery-process-model}{41}{]},

\begin{enumerate}
\def\labelenumi{\arabic{enumi}.}
\tightlist
\item
  find and understand problems; formalize question(s) which the results
  have to answer
\item
  design data models accordingly and test/correlate them against sample
  data
\item
  collect and prepare data
\item
  process data (data mining)
\item
  analyse and interpret results
\item
  use discovered knowledge (e.g.~make appropriate business decisions)
\end{enumerate}

In general, the majority of businesses are typically required to have
customer relationships. Such relations are based on the transfer of
valued goods (e.g.~services, products, etc.) in exchange for
compensation (e.g.~money). In order to process such a transfer, the
vendor requires certain information about the involved customer. Since
all entities related to this concept, including the customers, are
considered to be human beings, such information most likely includes
\emph{\protect\hyperlink{digital-identity-personal-data-and-ownership}{Personal
Data}}. A business normally is eager to grow, and if it has commercial
interests as well it aims for profit maximization. So the business needs
to overall improve, but therefor it requires knowledge on what are its
flaws in addition to where and how it can improve. To gain such
knowledge, analytics based on Big Data approaches are part of various
business strategies. But this also means, that a lot of Personal Data
gets collected as part of those analytics, since that data is part of
many business processes.

As a result of humans being primarily responsible for every money flow
in this globalized environment, they also decide on the success of
business. This basically means, every process analysis with an
underlying commercial dependency somehow involves personal data
{[}\protect\hyperlink{ref-web_2016_facebook-utilizes-98-data-points}{42}{]}
{[}\protect\hyperlink{ref-web_2016_big-data-types-of-data-used-in-analytics}{43}{]}
{[}\protect\hyperlink{ref-book-chapter_1999_Principles-of-knowledge-discovery-in-databases_introduction-to-data-mining}{44}{]}
{[}\protect\hyperlink{ref-web_2013_big-data-collection-collides-with-privacy-concerns}{45}{]},
whether this data is mandatory in that process or additionally obtained
to for example measure and analyse customer behaviour. Common data
points involved in big data analytics start with gender, age, residency
or income, goes on with time series events like changing current
geo-location or web search history and goes all the way up to health
data and self-created content like posts, images or videos. Depending on
the data point though, that data might not be that easy to collect. In
general, most businesses obtain data from within their own platforms.
Some data might even be in the customer's rang of control (e.g.~customer
or profile data), but most of the data is created during direct (content
creation, inputs) or indirect (transactions, meta information)
interaction with the business. The sensitivity level of involved
personal data is determined by how big is the benefit for the customer
in comparison to what is the vendor's demand from the customer's
commitment (e.g. required inputs, or usage requires access to location
information)

From a technical perspective, collecting indirectly created data is as
simple as integrating logging or debugging statements in the program
logic. Since most vendors and organisations nowadays have a business
that is partially happening through the internet or is even completely
based on it, most scenarios of transactions utilizing server-client
architectures. Furthermore, the \emph{always-on} philosophy evolved to
an imperative and implicit state of devices. Standalone software,
installed on a personal computer, calls the vendor's infrastructure that
is located in the cloud on a regularly basis, just to make sure that its
user behaves properly, while en passant the vendor is provided with
detailed user statistics. The web-browser already invokes a common
narrative that requests happening here and then - still preventable
though But when it comes to native mobile applications it's almost
impossible {[}\protect\hyperlink{ref-web_2016_answers-io}{46}{]} to
notice such behaviour or even prevent them from exposing potentially
sensitive information. Those developments in architecture design have
enabled a system-wide collection of potentially useful information on a
large scale
{[}\protect\hyperlink{ref-web_2016_big-data-enthusiasts-should-not-ignore}{47}{]}.
Logging transactions triggered by the user on the client and forwarding
the resulting logs to the back end infrastructure, or keeping track of
all sorts of transactions directly in the back end; all these collected
chucks of data are then being enriched with meta information before
running together in a designated place where they finally get stored and
probably never removed again, waiting to get analysed some day.

Within \emph{Big Data Communities} sometimes the \emph{big} seems to get
misinterpret as, regardless of what the problem is that needs to be
solved, gathering as much data points as possible is a valid approach,
even if it's just precautionary - in the future, those exact data might
become valuable. Such mindset is reflected by the often-cited concept of
the three \emph{Vs} (Volume, Velocity, Variety)
{[}\protect\hyperlink{ref-report_2001_3d-data-management-controlling-data-volume-velocity-and-variety}{48}{]}.
It is not entirely wrong though, because it's the nature of pattern and
correlation discovery to provide increasing quality results
{[}\protect\hyperlink{ref-paper_2015_big-data-for-development-a-review-of-promises-and-challenges:more-data}{49}{]},
when the overall data corpus gets enriched with more and precise
datasets. But typically when new technologies emerge and maybe even get
hyped, questioning downsides and possible negative mid- or long-term
impacts are typically unlikely to be high priority. The focus instead is
to experiment and try to reach and maybe even breach boundaries while
beginning to evolve. Non-technical aspects such as privacy and security
awareness don't come in naturally, instead the adoption rate has to
increase, whereupon more and divers research is happen, which then might
uncover such issues. Only then they can be addressed properly and on
different levels - technical, political as well as social. Hence, the
\emph{Big Data Community} itself is able to evolve, too.

Big Data and Knowledge Discovery is a balancing act between respecting
the user's privacy and having enough data at hand so that initial
questioning can be satisfied by the computed results. Aside from having
specific domain knowledge of the used technologies, people who are work
in these fields, need to be aware of any downsides or pitfalls and also
have to be sensible on the ramifications of their approaches and doings.
Such improvements are already happening, not only originating from the
community's forward thinkers
{[}\protect\hyperlink{ref-web_2016_the-state-of-big-data}{50}{]}, but
also advocated by governments and consumer rights organisations, as
stated in the previous section (see
\emph{\protect\hyperlink{digital-identity-personal-data-and-ownership}{legislation}}.
Even leading Tech-Companies start trying to do better
{[}\protect\hyperlink{ref-web_2016_apple_customer-letter}{51}{]}
{[}\protect\hyperlink{ref-web_2016_what-is-differential-privacy}{52}{]}
{[}\protect\hyperlink{ref-web_2016_eff_whatsapp-rolls-out-emd-to-end-encryption}{53}{]}.

In various science and research areas it's also very common to gather
and store enormous amounts of data. In these contexts the data and its
analytical results are used to for example run complex simulations
(e.g.~weather, population, diseases, hardware, physics), monitor and
analyse complex proceedings (e.g.~nature, infrastructure, behaviour),
explore the unknown (e.g.~universe), or even to imitate a famous painter
{[}\protect\hyperlink{ref-web_2016_research-experiment_ai-rembrandt}{54}{]}.
But unlike in the academic sector, the commercial interest within the
private sector is much larger, therefore Personal Data are in great
demand, because forecasting customer behaviour rather then the global
warming is much more valuable in that context.

It is the logical conclusion to distribute and scale horizontal when the
data demand exceeds latest hardware capacity and capability, but it's no
justification for thoughtless data collecting. Eventually, not the
amount of data does count, but how to handle such amounts in order to
not loose its usefulness; what data then flow into those data lakes is
up to the questioner, which might not act in the interest of everyone.
Therefore \emph{data subject} need to regain control and actively
participate in formulate these questions.

\hypertarget{personal-data-as-a-product}{\section{Personal Data as a
Product}\label{personal-data-as-a-product}}

The \emph{Big Data} paradigm itself, as mentioned before, just provides
a structured and technical-based method to uncover non-obvious or
non-visible information from self-made data silos, in order to assist in
making (right) (business) decisions. Though, when asking data collectors
about their actual motivation, most likely the answer would be something
along the lines of typical PR-phrasing like \emph{``we want to have a
better understanding of our customers''}. In the long run, this means to
increase revenue, but in short term to do what exactly? Maybe to predict
what might be the next thing people are supposed to buy, or what things
they would probably like to consume but most certainly not yet know of?

In order to try comprehending such perspective, lets take a look at some
examples.

\begin{enumerate}
\def\labelenumi{(\Alph{enumi})}
\item
  An advertising service utilizes tracking data for targeted
  advertising. The more information the service has on an individual,
  the more accurate decisions it is able to make about what ads are the
  ones that the individual most likely will click on and disclose with a
  successful purchase. As a result the placed advertisement becomes more
  valuable to the advertiser, because of the high precision. This again
  causes the service to increase the charges for serving and presenting
  the ad, because the overall quality of its service - the product - has
  improved.
\item
  Content recommendation engines of large streaming provider's,
  regardless of the content, serve as another example. This feature is
  also underpinned by extensive data aggregation of customer information
  (user profiling), such as consumption histories (e.g.~watch list),
  favoured content, friend's consumption or any kind of trackable
  platform interaction. This means, the more information the service has
  on the user, the more precise are the assumption on current mood,
  taste and interests, which leads to more suitable recommendations. At
  the end the user feels well taken care of and therefor values the
  service.
\item
  Another example is \emph{Google Traffic}
  {[}\protect\hyperlink{ref-web_2007_introducing-google-traffic}{55}{]}
  {[}\protect\hyperlink{ref-web_2016_wikipedia_google-traffic}{56}{]}, a
  service that is integrated as a feature in \emph{Google Maps}, a
  web-based mapping service from Google. \emph{Google Traffic}
  visualises real-time traffic conditions, while using \emph{Maps} as a
  navigation assistant in order to provide the user with a selection of
  possible paths. These paths are enriched with a duration, which takes
  the traffic conditions into account. The data, required to offer such
  information, is supplied by mobile devices, constantly sending its
  current position including a timestamp to Google's infrastructure.
  This however, is only possible, because Google's services are widely
  used in addition to the significant market share of mobile devices
  {[}\protect\hyperlink{ref-graphic_2016_global-mobile-os-market-share}{57}{]}
  that are driven by Android - a mobile operating system developed by
  Google that deeply integrates with its services. The same assumptions
  can be made as stated in the previous examples. The more geo-location
  data simultaneously is obtained, the more precise the information
  about traffic conditions are. Since this scenario demands real-time
  information, it adds the \emph{time} component as an additional level
  of complexity to the problem.
\end{enumerate}

Whereas the impact on the society of the first group of examples might
be questionable, yet it seems to be in doubt if proper applications even
exist, but adjusting the perspective reveals additional categories of
scenarios, for example

\begin{enumerate}
\def\labelenumi{(\Alph{enumi})}
\setcounter{enumi}{3}
\item
  Planing and managing human resources for special occasions with big
  croweds, such as huge events or emergency situations where attendees
  might get in danger and need some help
  {[}\protect\hyperlink{ref-estimating-the-locations-of-emergency-events-from-twitter-streams_2014}{58}{]}
\item
  Predicting infrastructure workloads (e.g.~power grid)
  {[}\protect\hyperlink{ref-paper_2015_improving-power-grid-monitoring-data-quality-an-efficient-machine-learning-framework-for-missing-data-prediction}{59}{]}
\item
  Making more accurate diagnostics to improve patient's therapy
  {[}\protect\hyperlink{ref-the-practice-of-predictive-analytics-in-healthcare_2013}{60}{]}
  (
\end{enumerate}

\begin{enumerate}
\def\labelenumi{\Alph{enumi})}
\setcounter{enumi}{6}
\tightlist
\item
  Finding patterns in climate changes, which otherwise would not have
  been detected
  {[}\protect\hyperlink{ref-data-collection-for-climate-changes_2014}{61}{]}
\end{enumerate}

Even though all described examples require a large corpus of data each
and utilize Knowledge Discovery, some of them might not necessarily
depend on Personal Data, whereas for other scenarios they are
indispensable and yet others only implicitly rely on data collected from
individuals. As noted in the
\href{personal-data-in-the-context-of-the-big-data-movement}{previous
section on Big Data}, it depends on the purpose, which can be defined
i.a. through a existing \emph{business model}. But at least in those
examples it seems to be common motivation to primarily improve and
enhance the collector's product in order to satisfy its customers - yet
again that depends on what is considered as product and who are the
customers, generally indicated by the money flow.

Generalizing businesses based on its affiliation to an industry sector
is hardly an for utilizing Personal Data, but empirically a rough
attribution often suffices in order to get a picture of possible
business models. With that said, the following observation can be made.
When putting a Top 10 List of industries utilizing \emph{Big Data}
{[}\protect\hyperlink{ref-graphic_2015_applications-of-big-data-in-10-industry-verticals}{62}{]}
and a visualization showing categories of personal data targeted by data
collectors
{[}\protect\hyperlink{ref-graphic_2012_personal-data-ecosystem}{63}{]}
side by side, it appears to be that at least seven\footnote{Banking and
  Securities; Communication, Media \& Entertainment; Healthcare
  Providers; Government; Insurance; Retail \& Wholesale Trade; Energy \&
  Utilities} of these industries can be identified as data collectors,
whereas less then a half\footnote{Banking and Securities; Communication,
  Media \& Entertainment; Insurance; Energy \& Utilities} are
participating in being a
\protect\hyperlink{terminologies--data-broker}{Data Broker}, but almost
all of them are suspected to target people's personal data, whether
obtained by themselves or acquired from \emph{Data Broker} (for more
examples, see
{[}\protect\hyperlink{ref-video_2016_corporate-surveillance-digital-tracking-big-data-privacy}{64}{]}).
Therefore, it's save to say, that \emph{Personal Data} is considered
either as the actual product, especially from a Dater Broker's point of
view, or indirectly due to its essential part in \emph{Big Data}
approaches. The former generates direct revenue by selling these data
and the latter might affect a business in a positive matter.

To conclude, the possession of more and precise information on
individuals leads to increased revenue for the possessor. Though,using
personal data to improve a product becomes not necessarily an issue,
unless the data owner is not the customer. Taking a closer look at the
business model often reveals the role of personal data. Thus Personal
Data becomes the currency and its owner becomes the product, whereas the
possessor becomes controller and profiteer. This rather unsatisfying
situation is going to be addressed further on. For example by shifting
the individual's role to offering its own data to those who have
previously collected and sold them.

\hypertarget{related-work}{\section{Related Work}\label{related-work}}

\emph{NOTICE: All research, projects, studies and work mentioned in this
section represents only a fraction of what's already been done in this
field and should be therefore seen as an excerpt containing selected,
most related and relevant approaches.}

The idea of a digital vault, controlled and maintained by the data
subject, is not new. Holding her most sensitive and valuable collections
of bits and bytes, protected from all the data brokers, collectors and
authorities, while interacting with the digital and physical world,
opening and closing its door from time to time to either put something
important to her inside or releasing an information important for
someone else. While in the mid and late 2000s the growth of computer
performance and capacity has crossed its zenith (see Moore's Law
{[}\protect\hyperlink{ref-paper_1965_moors-law}{65}{]}), at that same
time the internet has started to become a key part in many people's
lives and in the society as a whole. Facilitated by these circumstances,
\emph{cloud computing} has been on the rise ever since, causing the
shift towards distributed processing and patterns alike, thereby making
it possible to rethink solutions from the past and try to go new ways,
namely a breakthrough 2007 in \emph{neuronal networks} cutesy of G.
Hinton
{[}\protect\hyperlink{ref-podcast_2015_cre-neuronale-netze}{66}{]}. As a
result, fields like \emph{data mining}, \emph{machine learning},
\emph{artificial intelligence} and most recently combined under the
collective term called \emph{Big Data}, have gained a wide range of
attention as tools for knowledge discovery. In almost any industry a
greater amount of resources is invested in these areas
{[}\protect\hyperlink{ref-web_2016_industries-intention-to-invest-in-big-data}{67}{]}.

The initial motivation for this project can be understand as a
counter-movement away from all the data silos in the cloud, starting to
focus again on privacy, the individual and its digital alter ego.

From simple middleware-solutions, on to full-fledged software-based
platforms, through to embedded hardware devices, a great variety of
approaches have started to appear in the mid 2000s to this day. A side
effect, though, was, over time various research teams and projects have
invented and coined different terms, which at the end all refer to the
same, or at least similar, concept. The following list shows some of
them \emph{(alphabetical order)}:

\begin{itemize}
\tightlist
\item
  Databox
\item
  Identity Manager
\item
  Personal \ldots{}

  \begin{itemize}
  \tightlist
  \item
    Agent
  \item
    Container
  \item
    Data Store/Service/Stream (PDS)
  \item
    Data Vault
  \item
    Information Hub
  \item
    Information Management System (PIMS)
  \end{itemize}
\item
  Vendor Relationship Management (VRM)
\end{itemize}

One of the first occurring research projects was \emph{ProjectVRM},
which originated from \emph{Berkman Center for Internet \& Society} at
\emph{Harvard University}. As its name suggests, it was inspired by the
idea of turning the concepts of a \emph{Customer Relationship
Management} (CRM) upside down. This puts the vendor's customers back in
charge of their data priorly managed by vendors. It also solves the
problem of unintended data redundancy - from the customers perspective.
Over time the project has growing to the largest and most influential
one in this research field. It has transformed into an umbrella and hub
for all kinds of projects and research related to that topic
{[}\protect\hyperlink{ref-web_2016_projectvrm_development-work}{68}{]},
whether it's frameworks or standards, services offering (e.g.~privacy
protection), reference implementations, applications, software or
hardware components. \emph{VRM} became more and more a synonym for a set
of principles
{[}\protect\hyperlink{ref-web_2016_projectvrm_principles}{69}{]}, for
example \emph{``Customers must have control of data they generate and
gather. {[}They{]} must be able to assert their own terms of
engagement.''} These principles can be found in various ways across many
work done within this research area.

Another work of research worth mentioning, because of the foundational
work it has been done, is the european funded project called
\emph{Trusted Architecture for Securely Shared Service} (TAS3)
{[}\protect\hyperlink{ref-web_2011_tas3-project}{70}{]}. This project
has led to an open source reference implementation called
\emph{ZXID}.\footnote{more information on the project, the code and the
  author, Sampo Kellomäki, can be found under \emph{zxid.org}} The major
goal has been to develop an architecture, that generalizes various
approaches towards a non context-agnostic solution fitting into more
sophisticated and dynamic scenarios, while still respecting commercial
businesses (vendors) as well as users (customers), but at the same time
facilitating a high level of user-centric security and privacy i.a. by a
developed policy framework. Due to these requirements the architecture
ended up being rather complex
{[}\protect\hyperlink{ref-graphic_2011_architecture_components-of-organization-domain}{71}{]}.
\emph{ZXID} as its implementation involves several standards, like SAML
2.0\footnote{Security Assertion Markup Language 2.0} and
XACML,\footnote{eXtensible Access Control Markup Language} has just
three third-party dependencies, \emph{OpenSSL}, \emph{cURL (libcurl)}
and \emph{zlib}, and as of now it supports Java, PHP and Perl. The
project has lasted for a period of 4 years. After it has been finished
in 2011, the research work has continued i.a. by the \emph{Liberty
Alliance Project}. It is now part of the \emph{Kantara Initiative}
{[}\protect\hyperlink{ref-web_kantara-initiative}{72}{]}, including all
documents and results. These results are references occasionally,
recently by the IEEE
{[}\protect\hyperlink{ref-paper_2014_personal-data-store-approach}{73}{]}.

A research project, which is probably the closest to what this work aims
to create, bears the name \emph{openPDS}
{[}\protect\hyperlink{ref-paper_2012_openpds_on-trusted-use-of-large-scale-personal-data}{74}{]}
and is done by \emph{Humans Dynamics Lab}
{[}\protect\hyperlink{ref-web_mit_openpds-safeanswers-project-page}{75}{]},
which again is part of \emph{MIT Media Laboratories}. Despite the usual
concepts of a \emph{PDS}, it introduces multi-platform components and
user interfaces including a mobile devices, and also separates the
persistence layer physically. This approach enables place- and
time-independent administrative access for the data subject. Moreover,
with their idea of \emph{SafeAnswers}
{[}\protect\hyperlink{ref-paper_2014_openpds_protecting-privacy-of-meta-data-through-safeanswers}{76}{]},
the team goes even a step further. The concept behind that idea is based
around \emph{remote code execution}, briefly described in
\protect\hyperlink{header-applying-for-a-loan-and-checking-creditworthiness}{one
of the user stories in the first chapter}. It abstracts the concept of a
data request to a more human-understandable level, a simple question.
This question consists of two representations: (A) a human-readable
question of a third party, and (B) a code-based representation of that
question, which gets executed in a sandbox on the data subjects's
\emph{PDS} system with the required data as arguments. The data, used as
arguments, is implicitly defined trough (A). The output of that
execution represents both, answer and response.

Aside from all the research projects done within the academic context,
applications with a commercial interest have also started to occur in a
variety of sectors. Microsoft's HealthVault
{[}\protect\hyperlink{ref-web_microsoft_healthvault}{77}{]}, for
example, which aims to replace the patient's paper-based medical records
and combine them in one digital version. This results in a
patient-centered medical data and document archive, helping doctors to
make a more accurate decisions on medical treatment, because they have
more knowledge obtained for example from a personal medical history.

\emph{Meeco} {[}\protect\hyperlink{ref-web_meeco_how-it-works}{78}{]}
{[}\protect\hyperlink{ref-slides_2015_meeco-case-study}{79}{]}, based on
the MyData-Project
{[}whitepaper\_2014\_mydata-a-nordic-model-for-human-centered-personal-data-management-and-processing{]},
essentially just replaces platform-agnostic advertisement service
providers with a closed environment and serves ads on its own. Though,
within this environment data subjects are provided with more control
over what information they reveal, but this approach doesn't take the so
eagerly demanded next step to get rid of the advertisement market as
revenue stream and instead find a suitable business model that focuses
on the data subject, not surrounding them with just another walled
garden.

A recently announced project, sponsored by Germany's \emph{Federal
Ministry of Education and Research}, but developed and maintained
primarily by \emph{Fraunhofer-Gesellschaft} in cooperation with several
private companies like \emph{PricewaterhouseCoopers AG},
\emph{Volkswagen AG}, \emph{thyssenkrupp AG} or \emph{REWE Systems
GmbH}, is the so called \emph{Industrial Data Space}
{[}\protect\hyperlink{ref-web_industrial-data-space}{80}{]}. The project
unifies both, research and commercial interests and runs over a time
period of three years until the third quarter of 2018. It aims
\emph{``{[}\ldots{}{]} to facilitate the secure exchange and easy
linkage of data in business ecosystems''}, where at the same time
\emph{``{[}\ldots{}{]} ensuring digital sovereignty of data owners''}
{[}\protect\hyperlink{ref-whitepaper_2016_industrial-data-space}{81}{]}.
It will be interesting to see how these two, yet rather distinct
objectives, come together in the future. Based on the white paper, the
project's focus mainly seems to lie in enabling and standardizing the
way companies collect, exchange and aggregate data with each other
across process chains to ensure high interoperability and accessibility.

hereafter it can be found a selective list of further research projects,
work and commercial products around the topic of \emph{personal data}:

\textbf{Research}

\begin{itemize}
\tightlist
\item
  Higgins {[}https://www.eclipse.org/higgins/{]}
\item
  Hub-of-All-Things {[}http://hubofallthings.com/what-is-the-hat/{]}
\item
  ownyourinfo {[}http://www.ownyourinfo.com{]}
\item
  PAGORA {[}http://www.paoga.com{]}
\item
  PRIME/PrimeLife {[}https://www.prime-project.eu, http://primelife.
  ercim.eu/{]}
\item
  databox.me (reference implementation of the
  \emph{\href{https://github.com/solid/solid}{Solid framework}})
\item
  Polis (greek research project from 2008)
  {[}http://polis.ee.duth.gr/Polis{]}
\end{itemize}

\textbf{Organisations}

\begin{itemize}
\tightlist
\item
  Open Identity Exchange {[}http://openidentityexchange.org/resources
  /white-papers/{]}
\item
  Qiy Foundation {[}https://www.qiyfoundation.org/{]}
\end{itemize}

\textbf{Commercial Products}

\begin{itemize}
\tightlist
\item
  MyData {[}https://mydatafi.wordpress.com/{]}
\item
  RESPECT network {[}https://www.respectnetwork.com/{]}
\item
  aWise AEGIS {[}http://www.ewise.com/aegis{]}
\end{itemize}

\hypertarget{standards-specifications-and-related-technologies}{\section{Standards,
Specifications and related
Technologies}\label{standards-specifications-and-related-technologies}}

The overall attempt for this project is to involve as much standards as
possible, because it increases the chances for interoperability and
thereby it lowers the effort, that might be required to integrate with
third parties or other APIs. Hereinafter, some of the possible
technologies will be described briefly, and outlined what purposes they
might be going service in addition to why they might be a reasonable
choice.

\textbf{\protect\hypertarget{def--http}{}{HTTP}}
{[}\protect\hyperlink{ref-web_spec_http1}{82}{]}, the well known
stateless `transport layer' for, among others, the \emph{World Wide
Web}. It most likely will fulfill the same purpose in the context of
this work, because it implements a server-client pattern in its very
core. Whether internal components, locally on the same host or as part
of a distributed system, talk to each other or data consumers interact
with the system, this protocol transfers the data that needs to be
exchanged. Features that were introduced with Version 2
{[}\protect\hyperlink{ref-web_spec_http2}{83}{]} of the protocol are yet
to be known of their relevance for use cases within this project.\\
Unlike \emph{HTTP}, \textbf{WebSockes}
{[}\protect\hyperlink{ref-web_spec_websockets}{84}{]} provide the
concept of ongoing bidirectional connections on top of TCP, though
connection establishing utilizes the same principles known for HTTP(S).
This combination still allows TLS while benefiting from high efficient
real-time data exchange or from supporting remotely pending process
responses, while at the same time avoiding HTTP's long-polling
abilities. It is conceivable to use \emph{WebSockets} for communicate
between components or even with external parties.

\textbf{JSON}\footnote{The JavaScript Object Notation (JSON) Data
  Interchange Format; ECMA Standard\\
  {[}\protect\hyperlink{ref-web_spec_json}{85}{]} and Internet
  Engineering Task Force RFC 7159
  {[}\protect\hyperlink{ref-web_rfc_json}{86}{]}} is an alternative data
serialization format to XML, heavily used in web contexts to transfer
data via \emph{HTTP}, whose syntax is inspired by the JavaScript
object-literal notation. Like XML, its structuring mechanisms allow i.a.
type preservation and nesting, which enable to represent more complex
data structures including relations.

The open standard \textbf{\protect\hypertarget{def--oauth}{}{OAuth}}
defines a process flow for authorizing third parties to access
externally hosted resources, such as the user's profile image on a
social media platform. The authorisation validation is done by the help
of a previously generated token. However, generating and supplying such
token can be initiated in a variety of ways depending on the underlying
architecture and design, for example with the user entering her
credentials (\texttt{grant\_type=authorization\_code}). This design
seems to \emph{OAuth} is being misleadingly - but intentionally -
integrated as an authentication service
{[}\protect\hyperlink{ref-web_2012_problem-with-oauth-for-authentication}{87}{]}
rather then a authorization service; whether as an alternative or as an
addition to existing in-house solutions. Therewith the application
authors pass the responsibility on to the OAuth-supporting data
providers. OAuth \emph{version 1.0a}
{[}\protect\hyperlink{ref-web_spec_oauth-1a}{88}{]}, which is rather
considered a protocol, provides confidentiality by encrypting data
before it gets transferred and integrity of transferred data by using
signatures. Whereas \emph{Version 2.0}
{[}\protect\hyperlink{ref-web_spec_oauth-2}{89}{]}, labeled as a
framework, requires \emph{TLS} and thus passes on the responsibility for
confidentiality to the transport layer below. It also supports
additional process flows for scenarios involving specific platforms,
such as \emph{``web applications, desktop applications, mobile phones,
and living room devices''}
{[}\protect\hyperlink{ref-web_2016_oauth-2}{90}{]}.

With \textbf{OpenID} on the other side, the authenticity of a requesting
user gets verified by design. An in-depth description of the whole
process can be found in the protocol's same-titled specifications
{[}\protect\hyperlink{ref-web_spec_openid-spec-index}{91}{]}. With
decentralisation in mind, the protocols's nature encourages to design a
distributed application architecture, similar to the idea behind a
\emph{microservice}, but without owning all services involved -
\emph{decentralized authentication as a service} so to speak. An
application owner doesn't have to write or implement its own user
authentication and management system, instead it is sufficient to just
integrate those parts that are need to support signing in with
\emph{OpenID}, which is typically a client interacting with the Identity
Provider. Equally the user is not required to register an account for
every new app, instead she can use her \emph{OpenID}, already created
with another identity provider, to authenticate with the application.
The extension \emph{OpenID Attribute Exchange} allows to import
additional profile data, similar to what OAuth is meant to be used for.
\emph{OpenID Connect}
{[}\protect\hyperlink{ref-web_spec_openid-connect-1}{92}{]} is the third
iteration of the OpenID technology. While for example facebook uses
OAuth also for authentication (known as pseudo-authentication
{[}\protect\hyperlink{ref-web_2017_wikipedia_openid-vs-pseudo-oauth}{93}{]})
instead of just authorising entities, \emph{OpenID connect} on the other
hand, provides authentication in an additional layer build upon
\emph{OAuth2.0} and \emph{JWT}. Previous versions of OpenID have
provided the concept of extension in order add functionality such as
accessing profile information. This is ability is no part of the core
facilitated by OAuth, so that a user's identity can share certain data
with third parties via REST interface.

If it's required for certain components, while being part of a
distributed software, to not maintain any state, either the architecture
need to be changed so that the state at is no longer needed in that
component, or embed the state into the process communication so that it
is passed from one component to the other. This is a common use case for
a \textbf{\protect\hypertarget{def--jwt}{}{JSON Web Token}} \emph{(JWT)}
{[}\protect\hyperlink{ref-web_spec_json-web-token}{94}{]}. A \emph{JWT},
as it's name implies, is, syntactically speaking, formatted as
\emph{JSON}, but URI-safe encoded into \emph{Base64}, before it gets
transferred. The token itself contains the state. Here is where the use
of \emph{HTTP} comes in handy, because the token can be stored within
the HTTP header and therefore can be passed through all communication
points, where then data can be extracted and thereby verified. Such a
token typically consists of three parts: (A) information about itself,
(B) a payload, which can be arbitrary data such as user or state
information, and (C) a signature; all separated with a period.
Additional standards define encryption \emph{(JWE\footnote{JSON Web
  Encryption, Internet Engineering Task Force RFC 7516
  {[}\protect\hyperlink{ref-web_spec_json-web-encryption}{95}{]}})} to
ensure confidentiality, and signatures \emph{(JWS\footnote{JSON Web
  Signature, Internet Engineering Task Force RFC 7515
  {[}\protect\hyperlink{ref-web_spec_json-web-signature}{96}{]}})} to
preserve integrity of it's contents. Using a \emph{JWT} for
authentication purposes is described as \emph{stateless authentication},
because the verifying entity doesn't need to be aware of session IDs nor
any other information about a session. So, instead of the backend
interface being burdened to check a state (e.g.
\texttt{isLoggedIn(sessionId)} or \texttt{isAuthorized(sessionId)}) on
every incoming request, in order to verify permissions, which required
to maintain a state in the first place, it just needs decrypt the token
and proceed according to the contained information.

When transferring data over a potential non-private channel, several
features might be desired, which eventually provide an overall trust to
that data. One important aspect might me, that no one else expect sender
and receiver are able to know and see what the actual data is.
\textbf{Symmetrical Cryptography} is used to achieve this. It states
that the sender encrypts the data with the help of a key and the
receiver decrypts the data with that same key. That is, sender and
receiver, both need to know that particular key, but everyone who is not
allowed to access that information must not be in possession of that
key. To agree on a key without compromising the key during that process,
both entities either need to switch to a private medium (e.g meet
physically and exchange) or have to use a procedure, in which at any
point in time the entire key is not exposed to others than sender and
receiver. This procedure is called \textbf{Diffie-Hellman-Key-Exchange}
{[}\protect\hyperlink{ref-paper_1976_d-h-key-exchange}{97}{]} and is
based on mathematical laws of modulo operations for when prime numbers
are involved. It is designed with the goal to agree on a \emph{secret}
while at the same time using a non-private channel. The data, exchanged
during the process, alone cannot be used to exploit the secret. Such
behaviour is similar to the concepts of
\textbf{\protect\hypertarget{def--asym-crypto}{}{Asymmetrical
Cryptography}} \emph{(or public-key cryptography)}
{[}\protect\hyperlink{ref-book_2014_chapter-9-1-public-key-crypto}{98}{]},
which is underpinned by a \emph{key-pair}; one key is \emph{public} and
the other one is \emph{private}. Depending on which of keys is used to
\emph{encrypt} the data, only the other one can be used for
\emph{decrypting} the cipher. If then this technology gets combined with
the concept of digital signatures (encrypted fingerprints from data),
together it would provide integrity and authentication.

Putting \emph{HTTP} on top of the \emph{Transport Layer Security (TLS)}
{[}\protect\hyperlink{ref-web_spec_tls}{99}{]} results in
\textbf{{[}HTTPS{]}\{\#\}}. TLS provides encryption during data
transport, which reduces the vulnerability to \emph{man-in-the-middle}
attacks and thus ensures not only confidentiality but data integrity
too. \emph{Asymmetric cryptography} is the foundation for the connection
establishment, hence \emph{TLS} also allows to verify integrity of the
entity on the the connection's counterside, and, depending on the
integration, it could even be used for authentication purposes. Though,
relying on those cryptographic concepts requires additional
infrastructure. Such an infrastructure is known as \emph{Public Key
Infrastructure (PKI)}
{[}\protect\hyperlink{ref-book_2014_chapter-14-5-pki}{100}{]}. It
manages and provides keys and certificates for a dedicated scope of
entities in a directory, including related information to the owners of
these key and certificates. A Certificate Authority (or \emph{CA}), as
part of that infrastructure, issues, maintains and revokes digital
certificates. The infrastructure that is needed to provide secure HTTP
connections for the internet is one of those \emph{PKI}s - a public one
and probably one of the largest. It is based on the widely used
IETF\footnote{Internet Engineering Task Force; non-profit organisation
  that develops and releases standards mainly related to the Internet
  protocol suite} standard \emph{X.509}
{[}\protect\hyperlink{ref-web_spec_x509}{101}{]}.

\textbf{REST(ful)}\footnote{\emph{Representational State Transfer},
  introduces by Roy Fielding in his doctoral dissertation
  {[}\protect\hyperlink{ref-web_spec_rest}{102}{]}} is a common set of
principles to design web resources and its interaction. It primarily
defines server-client communication, in a more generic and thereby
interoperable way. Aside from hierarchically structured URIs, which can
reflect semantic relations or hierarchical order between data points, it
involves a group of rudimentary vocabulary\footnote{HTTP Methods or
  Verbs {[}\protect\hyperlink{ref-web_spec_http-methods}{103}{]}
  (e.g.~GET, OPTIONS, PUT, DELETE)} for HTTP requests to provide basic
CRUD-operations\footnote{Create, Read, Update, Delete} across
distributed systems. The entire request needs to contain everything that
is required to be processed on the REST-server, e.g.~state information
and possibly authentication parameter. A restful API typically has the
purpose of exposing certain features provided through the platform or
service to third parties in order to synergistically integrate with
them. But utilizing these principals for all internal server-client
interaction is also very common. This concept can also be understood as
a proxy to the actual business logic in the back end.

The \emph{QL} in \textbf{\protect\hypertarget{def--graphql}{}{GraphQL}}
{[}\protect\hyperlink{ref-web_spec_graphql}{104}{]} stands for
\emph{query language}. Developed by Facebook Inc., its goal is to
abstract multiple data sources into a unified API or resource, so that
different storage technologies are seamlessly queryable without using
it's native \emph{QL}. The result is provided in JSON format, which
naturally supports graph-like data structures. This is utilized in
GraphQL and implicitly embraced through its purpose of abstraction. Data
points that might be somehow related but stored in different locations,
can be obtained so that both are end up in the same object through which
they are related or indirectly linked to each other. The shape of a
query is later mirrored by the result. GraphQL is not only an
abstraction towards a more generic query language. It also separates
almost any operation and the flow control from the the \emph{QL} and
moves it into a second layer. The so called \emph{GraphQL} server is
responsible for resolving and executing queries.

The term \textbf{\protect\hypertarget{def--semantic-web}{}{Semantic
Web}} bundles a conglomerate of standards released by the W3C.\footnote{World
  Wide Web Consortium; international community that develops standards
  for the web} It is based around an idea called \emph{web of linked
data}, which aims to \emph{``enable people to create data stores on the
Web, build vocabularies, and write rules for handling data''}
{[}web\_2016\_w3c\_semantic-web-activity{]}. The standards address
syntax, schemas, formats, access control and integrations for several
scopes and contexts. Among others, the following three technologies are
essential for the \emph{Semantic Web}. RDF\footnote{Resource Description
  Framework {[}\protect\hyperlink{ref-web_w3c-tr_rdf}{105}{]}} basically
defines the syntax. OWL\footnote{Web Ontology Language
  {[}\protect\hyperlink{ref-web_w3c-tr_owl}{106}{]}} provides the
guidelines on how the semantics and schemas (vocabulary) should be
defined and with \protect\hypertarget{def--sparql}{}{SPARQL}
{[}\protect\hyperlink{ref-web_w3c-tr_sparql}{107}{]}, the query
language, data can be retrieved. One can not help it, but picture the
web as a database - queryable data with URIs that are embedded in
arbitrary websites. Imagine a person's email address, which is available
under a specific domain (preferably owned by that person) - or to be
more precise, a URI \emph{(WebID)
{[}\protect\hyperlink{ref-web_w3c-draft_webid}{108}{]}} - and provided
in a certain syntax \emph{(RDF)}, tagged with the semantic \emph{(OWL)}
of a email address, all together embedded in an imprint of a website.
This information can then be queried \emph{(SPARQL)}, if the unique
identifier of that person \emph{(URI)} is known. While defining the
standards, a main focus was to design a syntax that is at the same time
valid markup. The vision behind this: embracing the concept of a single
source of truth (per entity) and embedding or linking data points rather
then creating new instances of the same data that might only\\
valid at that point in time - in short, preventing redundant work and
outdated data. Related to the \emph{Semantic Web} is the a project
called \textbf{Solid}.\footnote{social linked data
  {[}\protect\hyperlink{ref-web_spec_solid}{109}{]}} Backed by the
\emph{WebAccessControl}
{[}\protect\hyperlink{ref-web_2016_wiki_webaccesscontrol}{110}{]} system
and based on the \emph{Linked Data} principles including the standards
just mentioned, the project focuses on decentralization and personal
data management while developing a specification around this. A
reference implementation called \emph{databox}
{[}\protect\hyperlink{ref-web_2016_demo_databox}{111}{]} follows the
specification process.

The concept of an application (or software)
\textbf{\protect\hypertarget{def--container}{}{container}} is about
encapsulating runtime environments by introducing an additional layer of
abstraction. A container bundles just those software dependencies
(e.g.~binaries) that are absolutely necessary so that the enclosed
program is able to run properly. The actual separation from the host
system is done, aside from other technologies, with the help of two
features provided by the Linux kernel. \emph{Cgroups},\footnote{control
  groups {[}\protect\hyperlink{ref-web_2015_cgroup-doc}{112}{]}} which
defines or restricts how much of the existing resources can be used by a
group of processes (e.g.~CPU, memory or network). Whereas
\emph{namespaces}
{[}\protect\hyperlink{ref-web_2016_kernel-namespace}{113}{]} defines or
restricts what parts of the system can be accessed or seen by a process
(e.g.~filesystem, user, other processes). The idea of encapsulating
programs from the operating system-level is not new. Technologies, such
as \emph{libvirt}, \emph{systemd-nspawn}, \emph{jails}, or
\emph{hypervisors} (e.g.~VMware, KVM, virtualbox) have been used for
years, but some of them were usually too cumbersome and never reached a
great level of convenience, so that only people with a certain expertise
have been able to handle systems build upon virtualization, whereas
people with other backgrounds couldn't and weren't that much interested.
Until \emph{Docker} and \emph{rkt} have emerged. After some years of
separated work, both authors, and others, recently joined forces in the
\emph{Open Container Initiative}
{[}\protect\hyperlink{ref-web_2016_open-container-initiative}{114}{]},
which promises to harmonize the diverged landscape and start building
common ground to ensure a higher interoperability. That in turn started
to raise a demand for sophisticated orchestration. It also marks the
initial draft of specifications for runtime
{[}\protect\hyperlink{ref-web_oci-spec_runtime}{115}{]} and image
{[}\protect\hyperlink{ref-web_oci-spec_image}{116}{]} definition for
\emph{application container}, which are still under heavy development.
This concept of \emph{containerization} also has the side-effect of
making the application platform-agnostic, because it allows a certain
set of software to run on a system which might otherwise not support
that software (e.g.~mobile devices); it just requires the runtime to be
supported.

In the past years different countries around the world started to
introduce \emph{information technology} to the day-to-day processes,
interactions and communications between public services and their
citizens. Processes like changing residence information or filing tax
report, are all summarized since then under the name
\emph{E-government}.\footnote{Electronic government} One of those
developments is the so called \textbf{Electronic ID
Card}\{\#def--eid-card\}, hereinafter called \emph{eID card}. Equipped
with storage, logic and interfaces for wireless communication, those
\emph{eID cards} can be used to store certain information and digital
keys, or to authenticate the owner electronically to a third party
without being physically present. Such an \emph{eID card} was introduced
also in Germany in 2010. The so called \emph{nPA}\footnote{in german so
  called \emph{elektronische Personalausweis (nPA)}} has been an
important step towards an operational \emph{e-government}. Aside from
minor flaws
{[}\protect\hyperlink{ref-web_2013_npa-sicherheitsdefizit}{117}{]} and
disadvantages
{[}\protect\hyperlink{ref-web_2014_test-qes-support-in-npa}{118}{]} an
\emph{eID card} might come along with, the question here is, how can
such technology be usefully integrated in this project and does it
somewhat plausible to do so. As an official document, the card has one
major advantage over inherent, self-configured or generated secrets like
passwords, fingerprints or TANs.\footnote{Transaction authentication
  number} It is \emph{signed} by design, which means, by creating this
document and handing it over to the related citizen, the third party (or
\emph{`authority'}) - in this case the government - has verified the
authenticity of that individual.

When communicating via email it's already common to encrypted the
transport channel, but using \emph{asymmetric cryptography} for
encrypting emails end-to-end is rather unusual. The equivalent to a
\emph{PKI} would be basically a \emph{public key server} that follows a
concept called \emph{web of trust}. In which all entities (user; senders
and recipients) are signing each other's public keys. The more users
have signed a public key, the higher the level of trust is on that key
and therefore on the entity who the signed key-pair belongs to. Public
keys are simply uploaded by its owners to those key servers mentioned
before. If someone want to write an email to others, all public keys
relating the recipients must be obtained from these public server, so
that the email can then be encrypted with those keys and are therefore
only readable by the owner of the keys's private part.

Related to that topic, another technology emerging as part of the
\emph{e-government} development, is the german
\textbf{\protect\hypertarget{def--de-mail}{}{De-Mail}}
{[}\protect\hyperlink{ref-web_2017_about-de-mail}{119}{]}. It's an
eMail-Service that is meant to provide infrastructure and mechanisms to
exchange legally binding electronic documents. One would expect a
\emph{public key cryptography}-based implementation from sender all the
way over to the recipient
{[}\protect\hyperlink{ref-statement_2013_de-mail}{120}{]}; maybe even
with taking advantage of the \emph{nPA's} capability to create
\emph{QES}, which refers to the ability of using the \emph{nPA} to sign
arbitrary data. Instead, the creators of the corresponding law decided
that it's sufficient to prove the author's identity based on its
credentials when handing over the email to the server, and that it's
reasonable to let the provider signs the document on the email server,
and finally that this described implementation results in a legally
binding document by definition of that law. The different levels of
mistakes made in conception and legislation are outlined in a statement
from CCC\footnote{Chaos Computer Club e.V.}-members
{[}\protect\hyperlink{ref-statement_2013_de-mail}{120}{]}, who have been
consulted as official experts during the development of that law.

\hypertarget{core-principles}{\chapter{Core
Principles}\label{core-principles}}

Right from the start a set of principles have built the cornerstones and
orientation marks of the idea behind the \emph{PDaaS}. Those, who are
meant to be reflected also by the upcoming \emph{Open Specification},
will be explained further within the following sections.

\section{Data Ownership}\label{data-ownership}

Depending on the perspective, the question about ownership of certain
data might not that trivial to answer. As stated in a
\protect\hyperlink{digital-identity-personal-data-and-ownership}{previous
section}, ownership requires a certain amount of originality to become
intellectual property, which is not the case for personal data - at
least for all the non-creative content. Thus there is no legal ground
for an individual to license those data that obviously belongs to her.
Switching the perspective from the \emph{data subject} to the data
\emph{consumer}; for them, several laws exist addressing conditions and
rules regarding data acquisition, processing and usage. Leaving aside
the absence of any legislation regarding data ownership, it cannot be
denied, that is seems unnatural not being the owner of all the data that
reflects a person's identity and the person as an individual. So instead
of defining those rules meant to protect data subjects but actually
demands data consumers to comply with, the proposal here is to put the
entity, to whom the data is related to, in control of defining, who can
access her data and what are accessors allowed to do with it. This would
make the \emph{data subject}, per definition (see
\protect\hyperlink{def-ownership}{Ownership}) and effectively to the
owner of those data. Although, it is to be noted, that the legislation
for data consumers mentioned before, remains a highly important, since
this project is not able to cover every single use case, that might
occur.

Promoted from the data subject to the data owner and therefore being the
center of the \emph{PDaaS}, the \emph{operator} of the \emph{PDaaS}
gains abilities to have as much control as possible over all the data
related to her, in order to determine in a very precise way what data of
hers can be accessed by third parties at any point in time, and also to
literally carry all her personal data with her.

\section{Identity Verification}\label{identity-verification}

When an instance of such a system \emph{(PDaaS)} is going to be the
digital counterpart of an individuals identity or her \emph{``personal
agent''}
{[}\protect\hyperlink{ref-book_2015_ethical-it-innovation}{121}{]}, then
everyone who relies on the information that agent provides, must also be
able to trust the source from where that data originates and vice versa;
the \emph{operator} too need to verify the authenticity of the
requesting source; regardless if it's the initial registration or
further \emph{access attempts}. Based on these mechanisms, the system
can also provide an authentication services that third parties might
utilize to outsource the authentication of its owner, including enabling
additional security factor steps.

\section{Reliable Data}\label{reliable-data}

Being able to verify the authenticity of a communication partner only
means to be half-way through. Data consumers also need to trust the data
itself, which is attributed by the following properties. (A)
\emph{integrity} - which means the recipient can verify that the
received data is still the same, or if someone has tampered with the
obtained data. (B) \emph{authenticity} - it is somehow ensured, or the
recipient must be certain, that the received data actual belongs to the
individual, meaning the data truly reflects attributes of the
individual, from whom the data comes from. A negative result of that
check should not cause a termination of the process, but instead should
warn data consumers unverified authenticity, so that they themselves can
decide if and how to proceed.

\section{Authorisation}\label{authorisation}

Controlling it's own data might probably be the most important ability
of such a system, because the data owner gets enabled to grant
permission to any entity who wants to obtain certain information in a
potentially automated way about her. She can authorise as precise as
desired how long and what data (sets, points or fields) is accessible by
a single entity. Thereby, the data owner is able to change the
\emph{access permissions} for any entity at any point in time, for
example motivated by a noticed incident.

\hypertarget{supervised-data-access}{\section{Supervised Data
Access}\label{supervised-data-access}}

Rules and constraints might be one way to handle personal data demands
of third parties. But a plain \emph{query-and-respond-data} approach
could be replaced by a more goal-driven concept, that prevents data from
leaving the system. It allows to execute a small program within a
locally defined environment, computing only a fraction of a larger
computation initiated by the \emph{data consumer} beforehand; similar to
a distributed Map-Reduce concept
{[}\protect\hyperlink{ref-paper_2004_distributed-mapreduce}{122}{]}. The
opposite but also conceivable approach would be to provides either some
software to the \emph{data consumer}, which is required in order to
access the contents of a response, or a runtime environment that queries
the system by itself. In general, it is not very likely that data
consumers, who already got granted certain access, would renounce their
privileges. Thus it is vital that the data owner is the one in charge of
canceling access permissions or applying appropriate changes to them.
Supervising methods provide an appropriate ways to make data available
to those who are eager to consume them.

\section{Containerization}\label{containerization}

Abstracting an operating system by moving the bare minimum of required
parts into virtualizations results in an environment setup that can,
depending on the configuration, fully encapsulate its internals from the
host environment. This approach yields to some valuable features, such
as

\begin{enumerate}
\def\labelenumi{(\Alph{enumi})}
\tightlist
\item
  Effortless portability, which reduces the requirements on environment
  and hardware
\item
  Higher flexibility in placing components, through which advantages can
  be made out of characteristics that other devices might bring with
\item
  Isolation and reduction of shared spaces and scopes, which for example
  can prevent side effects\\
  All these in conjunction lead also to an overall security improvement,
  or at least it enables new patterns to improve those aspects.
  Furthermore, it opens up for more versatile and diverse scenarios,
  like storing data decentralized, scaling during unexpected workloads
  or getting used as a medical record due to higher security
  precautions. The convenience of precisely assign environment resources
  might also become relevant for cases where device's hardware
  specification might be somewhat low. Building a system upon a
  container-based philosophy and enclosing components in their own
  environment offers a variety of design and architectural possibilities
  without the necessity of increasing the overall system complexity.
\end{enumerate}

\section{Open Development}\label{open-development}

When developing an \emph{Open Specification} it only comes natural to
build upon open technologies, which are recognized as \emph{open
standards} and \emph{open source}; \emph{open} in the sense of
\emph{unrestricted accessible by everybody}, not to be confused with
free - as in \emph{freedom} - software. Advocating such a philosophy
allows not only to develop implementations in a collaborative way, but
also enables to work fully transparent on the specification itself. Such
an open environment makes it possible for anyone who is interested, to
participate or even to contribute to the project. Thus, to lower the
barrier, usable and meaningful documentation is vital. Such an openness
ensures the possibility of people looking into the source code and
getting a picture of what the program actually does and how it works.
Even source code reviews therefore become feasible and any security flaw
thus uncovered can be fixed immediately. Furthermore, this approach
allows data subjects to setup their own infrastructure and host such a
system by themselves, which gains even more control over the data and
increases the level of trust, instead of using a \emph{SaaS}\footnote{Software
  as a Service} solution that is host by possibly untrusted entity. It
also encourages any kind of adjustments or customization to the software
in order to serve operator's own needs. Enabling an open development
enables users and contributors working together and collectively improve
the project in a variety of ways.

\hypertarget{requirements}{\chapter{Requirements}\label{requirements}}

Derived from the \protect\hyperlink{core-principles}{Core Principles},
the subsequent requirements shall be served as a list of features on the
one hand, to get an idea of how the open specification and thus the
resulting software might look like, and on the other hand to give an
overview about priorities (can/could, may/might, should, must/have to).
Other chapters may contain references to specific requirements listed
below.

\subsubsection{Architecture \& Design:}\label{architecture-design}

\textbf{\emph{\protect\hypertarget{sa01}{}{S.A.01}} - Accessibility \&
Compatibility}\\
Since the internet is one of the most widely used infrastructure for
data transfer and communication, it is assumed that all common platforms
support underlying technologies, such as HTTP and TLS. Thus the emerging
system must implement a service based on web technologies, that provides
supervised access to personal data.

\textbf{\emph{\protect\hypertarget{sa02}{}{S.A.02}} - Portability}\\
All major components should be designed and therefore communicate with
each other in a way so that individual components are able to get
relocated while the system has to remain fully functional. It must be
possible to build a distributed system, that may require to place
certain components into different environments/devices/platforms.

\textbf{\emph{\protect\hypertarget{sa03}{}{S.A.03}} - Roles}\\
The system has to define two types of roles (see also
\protect\hyperlink{terminologies}{Terminologies}). The first one is the
\protect\hyperlink{terminologies--operator}{operator}, who is in control
of the system and, depending on the architecture, must be at least one
individual but could be more. The operator maintains all the data that
gets provided trough the system, and decides which third party gets
access to what data. The second type is a
\protect\hyperlink{terminologies--consumer}{consumers}. These consumers
are external third parties that desire to access certain data about or
from the operator. While the consumer has to interact with the system
only via plain HTTP requests, the operator must be provided with
graphical user interfaces for possibly multiple platforms.

\textbf{\emph{\protect\hypertarget{sa04}{}{S.A.04}} - Authenticity}\\
Since they have to rely on the data, both entities - everyone who
belongs to one of the \emph{\protect\hyperlink{sa03}{roles}} - have to
be able able to verify the authenticity of the opposite' identity as
well as of the received data, in order to be certain that this data is
real and belongs to the sender. It should be possible to opt out to that
level of reliability if it's not necessary, or to opt-in selectively for
certain types of data. However, if one of the parties demands the other
one to provide such level, but the other doesn't, then the access
attempt has to fail.

\textbf{\emph{\protect\hypertarget{sa05}{}{S.A.05}} - Availability}\\
When third parties are requesting data, it's very likely that those
procedures are triggered automatically or at least machine-supported.
Hence those requests can arrive at the \emph{PDaaS} at any point in
time. Therefore the \emph{PDaaS}, or at least parts of it, should be
available all the time. Even if the request can't be processed
completely, the system is still able to inform the \emph{data subject}
about that event; somewhat like an answering machine. This also enhances
the \emph{PDaaS} as a serious and reliable data source. It also relates
to the topic of \emph{failure safety and redundancy}.

\textbf{\emph{\protect\hypertarget{sa06}{}{S.A.06}} - Communication}\\
In order to elaborate on \protect\hyperlink{sa01}{S.A.01} and
\protect\hyperlink{sa02}{S.A.02}, to ensure internal interoperability,
all communication between components has to happen on HTTPS, if they
don't run in the same environment.

\subsubsection{Persistence:}\label{persistence}

\textbf{\emph{\protect\hypertarget{sp01}{}{S.P.01}} - Data Outflow}\\
Data may leave the system only if it's absolutely necessary and no other
option exists to retain the goal of the relating process. But if data
still has to get transferred, no other than the data consumer must be
able to access that data. Confidentiality has to be preserved at all
cost.

\textbf{\emph{\protect\hypertarget{sp02}{}{S.P.02}} - Data
Relationship}\\
Data structures and data models must show high flexibility and may not
consist of strong relations and serration. If the syntax of the data
representation also provides structure elements then it should be
utilized to also embed semantics.

\textbf{\emph{\protect\hypertarget{sp03}{}{S.P.03}} - Schema and
Structure}\\
The operator can create new data types (based on a schema) in order to
extend the capabilities of the data API. Structures and schemas can
change over time (\protect\hyperlink{sp04}{S.P.04}). Every dataset and
data point has to relate to a corresponding type, whether it's a simple
type (string, integer, boolean, etc.) or a structured composition based
on a schema.

\textbf{\emph{\protect\hypertarget{sp04}{}{S.P.04}} - Write}\\
Primarily the operator is the one who has permissions to add, change or
remove data. This is done either by using the appropriate forms provided
by a graphical user interface or with the help of other import
mechanisms. The latter could be enabled by (A) supporting the upload of
files that containing supported formats, (B) data APIs restricted to the
operator or (C) defining an external resource reachable via http (e.g.
\emph{RESTful URI}) in order to (semi-)automate further ongoing data
imports from multiple data resources (e.g.~IoT, browser plugin).
Additionally, it might be possible in the future to allow \emph{data
consumers} making some data to flow back into the operator's system,
after she is certain that this data is valid and useful for her.

\textbf{\emph{\protect\hypertarget{sp05}{}{S.P.05}} - Data Redundancy}\\
Providing and managing data is the core task here. Hence the system
needs to make backups or at least provide mechanisms and tools for the
\emph{operator} to do that. Different strategies are conceivable but
have to respect related requirements (\protect\hyperlink{sp01}{S.P.01},
\protect\hyperlink{sa03}{S.A.03}) and specific environment conditions
though. The least feasible solution would be a manual backup only
allowed by the \emph{operator}.

\textbf{\emph{\protect\hypertarget{sp06}{}{S.P.06}} - Data Precision}\\
As stated in the previous chapter about
\emph{\protect\hyperlink{personal-data-as-a-product}{Data as a
Product}}, the level of precision of data has a significant impact on
the \emph{data subject's} privacy. Therefore the \emph{data subjects}
must be provided with the ability to configure how precise certain data
should be when it is provided to \emph{data consumers} somehow. Those
adjustments must have no impact on the actual stored data. That is,
adjustments have to be made after the data is fetched but before it is
further processed.

\subsubsection{Interfaces:}\label{interfaces}

\textbf{\emph{\protect\hypertarget{si01}{}{S.I.01}} - Documentation}\\
The interfaces from all components have to be documented and well
specified so that the components themselves could be individually
replaced without any impact to the rest of the system. This also
involves comprehensive information on how to communicate and what
endpoints are provided, including required arguments and result
structure.

\textbf{\emph{\protect\hypertarget{si02}{}{S.I.02}} - External Data
Query}\\
Data consumer can request a schema, in order to know how the response
data will actually look like, since certain parts of the data structure
might change over time (see \protect\hyperlink{sp03}{S.P.03},
\protect\hyperlink{sp04}{S.P.04}). In order to check if the access
request is permitted, the system first parses and validates the query,
and eventually proceeds to execution. When querying data from the
system, the \emph{data consumer} might be required to provide a schema,
which should force him to be as precise as possible about what data
exactly needs to be accessed. In addition to that, the consuming entity
must provide some \emph{meaningful} text, describing the purpose of the
requested data. He should not be allowed to place wildcard selectors for
sets of data into the query. Instead he must always define a more
specific filter or a maximum number of items, if the query retrieves
more then one element.

\textbf{\emph{\protect\hypertarget{si03}{}{S.I.03}} - Formats}\\
When components communicate between each other or interactions with the
system from the outside take place, all data send back and forth should
be serialized/structured in a JSON or JSON-like structure.

\subsubsection{Graphical User
Interface:}\label{graphical-user-interface}

\textbf{\emph{\protect\hypertarget{pviu01}{}{P.VIU.01}} - Responsive
user interface}\\
The graphical user interface has to be responsive to the available
space, in order to acknowledge the divers market of available screen
sizes.

\textbf{\emph{\protect\hypertarget{pviu02}{}{P.VIU.02}} - Platform
support}\\
All graphical user interfaces must be implemented at least based on web
technologies, which are then provided by a server and available on any
system that comes with a modern browser. To enable additional features
and deeper integration with the surrounding environment, it is
recommended - at least for mobile devices - to build user interfaces
upon natively supported technologies, such as \emph{Swift} and
\emph{Java}. The operator would benefit from capabilities such as
\emph{push notifications} and storing data on that device.

\textbf{\emph{\protect\hypertarget{pviu03}{}{P.VIU.03}} - Permission
Profiles}\\
The operator should be capable of filtering, sorting and searching
through a list of \emph{permission profiles}; for a better
administration experience and to easily find certain entries in a
continuously growing list of profiles.

\textbf{\emph{\protect\hypertarget{pviu04}{}{P.VIU.04}} - Access
History}~ The operator must be provided with a list of all past
\emph{permission and access requests}, in order to monitor who is
accessing what data and when, and thus being capable of evaluating and
eventually stopping certain access and data usage. Such tool should have
filter, search and sort capabilities. It is build upon and therefore
requires the \protect\hyperlink{pb01}{access logging} functionality.

\subsubsection{Interactions:}\label{interactions}

\textbf{\emph{\protect\hypertarget{pi01}{}{P.I.01}} - Effort}\\
Common interactions for use cases, such as changing \emph{profile data},
importing datasets or managing \emph{permission requests} have to
require as little effort as possible. This means short UI response time
on the one hand and as less single input and interaction steps as
possible to complete a task. Given the circumstances (e.g.~distributed
architecture, involvement of mobile devices), the \emph{permission
request review} and \emph{permission profile creation} might become a
special challenge to hide certain complexity.

\textbf{\emph{\protect\hypertarget{pi02}{}{P.I.02}} - Design}\\
Graphical user interfaces must be designed and structured in such a way
that is is highly intuitive for the user to operate. Thus, it is
important for example to use meaningful icons and appropriate labels.
This also refers to a flat and not crammed menu navigation. Context
related interaction elements should be positioned within the area
related to that context.

\textbf{\emph{\protect\hypertarget{pi03}{}{P.I.03}} - Notifications}\\
The user should be notified about every interaction with the system
originated by a third party immediately after its occurrence, but she
has to be notified at least about every \emph{permission request}. This
behaviour should be configurable; depending on the \emph{permission
type} and on every \emph{permission profile}. Regardless of the
configuration, notifications themselves must show up decently and
pending user interactions must be indicated in the user interface.
Notifications do not necessarily require a reaction by the operator.

\textbf{\emph{\protect\hypertarget{pi04}{}{P.I.04}} - Permission Request
\& Review}\\
A process involving data transaction must be always initiated by the
data subjects. So, before a \emph{data consumer} is able to access data,
first the \emph{operator} needs to \emph{invite} that specific third
party by either telling him (per URI) where to register or asking him to
provide all information required for a registration process upfront
encoded in QR-Code. Both solutions need a secure channel for transport,
which refers to \emph{TLS}. The latter has to be ensured by the applying
third party alone, whereas the first option requires the \emph{PDaaS} to
provide a TLS-supporting endpoint as well. After a successful
registration, the consumer can submit a \emph{request permission}, which
has to include information about the \emph{consumer}, what data he wants
to access, for what purpose and how long or how often the data needs to
be accessed. The operator then reviews these information and creates an
\emph{permission profile} based on that information. A verz important
attribute in such a profile has to be the definition of when this
permission expires. The operator should be able to decide between three
\emph{permission types}:

\begin{itemize}
\tightlist
\item
  \emph{one-time-only}
\item
  \emph{expires-on-date}
\item
  \emph{until-further-notice} After creating the profile, a response
  must be send to the \emph{data consumer}, which should contain the
  review result and the determined permission type.
\end{itemize}

\textbf{\emph{\protect\hypertarget{pi05}{}{P.I.05}} - Templating}\\
The operator should be able to create templates for \emph{permission
profiles} and \emph{permission rules} in order to (A) apply a set of
configurations in advance before a \emph{permission request} arrives and

\begin{enumerate}
\def\labelenumi{(\Alph{enumi})}
\setcounter{enumi}{1}
\tightlist
\item
  reduce recurring redundant configurations and interactions.
\end{enumerate}

\subsubsection{Behaviour:}\label{behaviour}

\textbf{\emph{\protect\hypertarget{pb01}{}{P.B.01}} - Access Logging}\\
All interactions and changes in the persistence layer should be logged.
At data request must be logged. Such log is the foundation of the
\emph{access history}, so that the user is able to keep track of
occurring accesses and look up past ones.

\textbf{\emph{\protect\hypertarget{pb02}{}{P.B.02}} - Real time}\\
Real time communication might be essential for time-critical data
transaction. Hence graphical user interfaces should be communicating
with the server through an ongoing connection to enable real time
support. Assuming the following example scenario: permission request got
reviewed on mobile device, but notification indicator in the desktop
browser still reflects `pending'. If just one of the user interfaces has
no real-time capabilities but all the others do, the user interface
might get into an undefined state presenting the user with wrong
information, which will decrease user experience dramatically. This
mean, either all user interfaces have to provide real-time feedback or
non.

\hypertarget{design-discussion}{\chapter{Design
Discussion}\label{design-discussion}}

The following chapter documents the processes of a variety of design
decision makings, examines possible issues emerging alongside and
discuses different solutions obtained from several perspectives in order
to evaluate their advantages and disadvantages. Probably not every issue
gets its deserved room, but major aspects will be addressed. In short,
the majority of the project's conceptual work is done below.

The end of every subchapter includes a section containing a summary of
conclusions based on the prior discussions related to that topic of that
subchapter.

\hypertarget{authentication}{\section{Authentication}\label{authentication}}

First of all, the system has to support two
\protect\hyperlink{sa03}{roles}. Either one or non of these roles is
assigned to an entity, hence entities that are trying to authenticate to
the system might have different intentions depending on the
characteristics of those roles. The \emph{operator} for example wants to
review \emph{permission requests} in real time, so accessing the system
from different devices is a common scenario. When inheriting the
\emph{operator role} an entity gains further capabilities to interact
with the system, such as data manipulation. Whereas a \emph{data
consumer} always uses just one origin and processes requests
sequentially. Those very distinct groups of scenarios would make it
possible to apply different authentication mechanisms that not
necessarily have a lot in common.

With respect to the requirements (\protect\hyperlink{sa01}{S.A.01}), the
most appropriate way to communicate with the \emph{PDaaS} over the
internet would be by using \emph{\protect\hyperlink{def--http}{HTTP}}.
Thus, to preserve confidentiality on all in- and outgoing data
(\protect\hyperlink{sp01}{S.P.01}) the most convenient solution in this
case is to use \emph{HTTP} on top of \emph{TLS}. \emph{TLS} relies i.a.
on \protect\hyperlink{def--asym-crypto}{asymmetric cryptography}. During
the connection establishment the initial handshake requires a
certificate, issued and signed by a CA, which has to be provided by the
server. This ensures at the same time a reasonable level of identity
authentication, almost effortless. If the certificate is not installed,
it can be installed manually on the client. If the certificate is not
trusted (e.g.~it is self-signed), it can either be ignored or the
process fails to establish a connection, depending on the server
configurations. The identity verification in TLS works in both
directions, which means not only the client has to verify the server's
identity by checking the certificate. If the server insists on, the
client has to provide a certificate as well, which then the server tries
to verify. Only if the outcome is positive, the connection establishing
succeeds. According to the specification
{[}\protect\hyperlink{ref-web_spec_tls-12_client-auth}{123}{]} it is
still optional though.

\emph{HTTP} as a comprehensive and flexible protocol enables to use
several technologies for server-client authentication purposes. Some of
them are build-in, others can simply be implemented on top of it. Within
the scope of this work, those technologies are categorized in the
following types: (A) stateful and (B) stateless authentication. The
first one (A) includes vor example \emph{Basic access Authentication}
(or \emph{Basic Auth}) and authentications based on \emph{Cookies}.
Whereas the \emph{two-way authentication} in TLS mentioned above and
\protect\hyperlink{def--jwt}{authentication based on web-token} are
associated with the latter (B). \emph{Basic Auth} is natively provided
by the \emph{http-agent} and requires in its original form
\emph{(user:password)} some sort of state on the server; at least when
the system has to provide multitenancy. If instead just a general access
restriction for certain requests would suffice, no state is required.
One of the most common implementations of user-specific states is a
\emph{session} on the server, that contains one or more values
representing the state and a unique identifier, by which an entity can
be associated with. A client has to provide that session ID in order to
be provided with all session-related data hold by the server. This is
typically done in a HTTP header, whether as \emph{Basic Auth} value, a
\emph{Cookie}, which is domain-specific, or in some other custom header.
Since the \emph{two-way authentication} (or \emph{mutual authentication}
{[}\protect\hyperlink{ref-web_2017_wikipedia_mutual-auth}{124}{]}) is
performed based on files containing keys and certificates, which are
typically not very fluctuant in its contents or state, this procedure is
categorized as stateless. Order or origin of incoming requests have no
impact on the result of the actual authentication process. The same
applies to TLS features such as \emph{Session {[}ID, Ticket{]}
Resumption}
{[}\protect\hyperlink{ref-book_2013_networking-101_tls-session-resumption}{125}{]},
therefore they are left aside, because they serve the sole purpose of
performance optimization. Similar to the \emph{Session Ticket
Resumption}
{[}\protect\hyperlink{ref-web_spec_tls-session-ticket-resumption}{126}{]}
a web token, namely the \protect\hyperlink{def--jwt}{JSON Web Token},
also moves the state towards the client, but that's about all they have
in common. A \emph{JWT} carries everything with it that's worth knowing,
including possible states, and if necessary the token is symmetrical
encrypted by the server. That is, only the server is able to see the
actual data contained by JWT and therefore reacting accordingly.

Keeping track of one or multiple states on the server and maintain the
synchronization of involved data between server and client is expensive
and by fare trivial, because this pattern requires the server to be
aware of all current states (sessions) and has to have them accessible
at all time. The additional resources required for such an approach were
referred to as being expensive. It also means, that requests, ultimately
resulting in responses that contain contents, might depend on preceding
requests and their incoming order. Furthermore those session data has to
be in a consistent state in order to ge safely stored from time to time.
Otherwise, if the server fails to run at some point, data only existing
in the memory would be gone with no chance to get recovered. To a
stateless authentication non of those aspects applies. Certificates and
keys as well as web tokens are both carry the information with them that
might be necessary. Thus, considering those disadvantages, \emph{public
key cryptography} and web tokens are the preferred technologies for all
authentication processes.

Except for \emph{two-way authentication} all authentication technologies
mentioned above require an initial step from the client to obtain some
sort of token that is used to authenticate all subsequent requests. This
step is commonly known as \emph{login} or \emph{sign in} and requires
the authorizing entity to provide some credentials consisting of at
least two parts. One part, that uniquely relates to the entity but
doesn't have to be private, and another part that only the entity knows
or has. Typically that's a username or email address and a password or
some other secret bit sequence (e.g.~stored on and provided by a USB
stick). An \emph{\protect\hyperlink{def--eid-card}{eID card}} could
possibly be used as secret (or unique object) as well. Suitable use
cases therefor are (A) to let the \emph{operator} login to the
\emph{PDaaS} management tool or

\begin{enumerate}
\def\labelenumi{(\Alph{enumi})}
\setcounter{enumi}{1}
\tightlist
\item
  to approve or authorize \emph{access requests}. How the actual login
  process (A) would look like partially depends on the \emph{eID card}'s
  implementation, but in general this use case would make sense. When
  considering the german implementation \emph{(nPA)} for example,
  accessing the management tool via desktop requires also a card reader,
  preferably with an integrated hardware keypad. Whereas authenticating
  to the tool with a mobile device could be achieved with the card's
  RFID-capabilities, as long as the used device is able to communicate
  with the RFID-chip. Both scenarios (A+B) required the \emph{nPA} to
  have the \emph{eID} feature enabled. If a service wants to provide
  \emph{nPA}-based online authentication \emph{(eID-Service)}, which is
  defined as a non-sovereign \emph{(``nicht hoheitlich'')} feature, it
  has to comply with several requirements
  {[}\protect\hyperlink{ref-web_bsi-spec_eid}{127}{]} starting with
  applying for a permission to send a certificate signing request to a
  BerCA.\footnote{(german) Berechtigungszertifikate-Anbieter} This
  request is send from an \emph{eID-Server}
  {[}\protect\hyperlink{ref-web_2017_npa-eid-server}{128}{]} in order to
  get a public key signed, which priorly has been generated on a
  dedicated and certified hardware. This hardware is requested by the
  officials as part of a \emph{eID-Server}. The key pair - re-generated
  and re-signed every three days - is needed to establish a connection
  to the \emph{nPA}, which is then used to authenticate the owner of
  that \emph{eID card}. The described procedure appears to be highly
  expensive (regarding effort, hardware costs etc.), especially because
  every single \emph{operator} would needs to go through the whole
  process in order to support this authentication method; not mentioning
  the uncertainty of the official's decision on the permission
  application. Another approach could be to integrate an external
  authentication provider supporting the \emph{nPA}, which would not
  only add an additional dependency, but also weaken the system. All two
  scenarios are fairly similar, insofar as they would use the same
  mechanism to initially authenticate to the system, but with different
  intentions.
\end{enumerate}

Because of its simplicity the concept of web tokens are fairly
straightforward to implement into the \emph{PDaaS}. But since web tokens
ensure integrity and optional confidentiality only of their own carriage
but not for the entire HTTP payload, both requirements need to be
addressed separately. Serving HTTP over \emph{TLS} solves that issue
though. For connections that use a web token, it should be sufficient to
rely on the public PKI that drives the internet with \emph{HTTP} over
\emph{TLS}. All information required for the actual authenticate are
provided by the token itself. Though, the situation is different if
\emph{two-way authentication} is used instead. Therefor, the system has
to provide its own \emph{PKI} including a Certificate Authority that
issues certificates for \emph{data consumers}, because not only the
\emph{endpoints} on the \emph{PDaaS} (server) need to be certified, all
\emph{data consumers} (clients) need to present a certificate as well.
Only the \emph{PDaaS} verifies and thus determines (supervised by the
\emph{operator}) who is authorized to get access to the system. Hence
the \emph{PKI} needs to be selfcontained and private in order to
function independently so that only invited parties can get involved.
Referring to the statement mentioned above, \emph{data consumer} have as
well to be able to verify the identity of the \emph{PDaaS}, in order to
prevent man-in-the-middle attacks. Addressing this issue basically
means, \emph{data consumers} have to verify the certificate presented by
the \emph{PDaaS}. This can be done in two ways. One is, a certificate
has been installed on the \emph{PDaaS} that is certified and therefore
trusted by a trustworthy public CA, as mentioned above. Then
\emph{consumers} use the CA's certificate to verify the \emph{PDaaS}
certificate. The other way is, to let the \emph{PDaaS} create and sign a
public key by itself. Before \emph{consumers} then get presented with
the self-signed certificate of the \emph{PDaaS} during the initiation of
the TLS connection, they already have to be aware of that certificate.
That is, \emph{consumers} need to be provided with that certificate on a
private channel upfront. Otherwise would the process be again vulnerable
to man-in-the-middle-attacks.

In summary, if a public-key-based connection, performing a \emph{two-way
authentication}, establishes successfully, it implies that the identity
originating the request is valid and the integrity of the containing
data is given. Whereas on a token-based authentication every incoming
request has to carry the token so that the system can verify and
associate the request with an account. Furthermore data it not
automatically encrypted and thus integrity is not preserved.

An advantage of token-based authentication over TLS-based \emph{two-way
authentication} is that the token can be used on multiple clients at the
same time. Or an account, a token is associated with, can actually have
more than one token. Whereas during the asymmetric cryptography-based
\emph{two-way authentication} the client's private key is required,
which results in 1:1 relation. So if it's likely that a \emph{operator}
has several clients, regardless for what purposes, then the same private
key has to be on those clients. Though, a private key typically does not
leave its current system or at least does not exist in multiple systems
at the same time in order to prevent exposure, which any action of
duplication implies. To reduce those risks, it's common practice to
generate a private key at that location where it is going to be used.

OpenID, a open standard for decentralized user authentication, also uses
subdomains as unique identifiers to associate with entities that need to
be authenticated, similar to the approach proposed in this work. But
since it originates in a very distinct scenario it is also very related
and therefore restricted to that. Trying to adopt the standard might
result in various adjustments to this project leading to an
implementation that shares not much compliance, which is not the
intention of a standard.

The technology \emph{\protect\hyperlink{def--de-mail}{De-Mail}} tries to
ensure authenticity of an author's identity by embedding a legal
foundation into email-based communication. But instead of providing
technically valid authenticity through end-to-end encryption so that a
recipient can truly rely on that information, it only goes as far as
legal definition and legislation reaches. Thus it has no relevance to
this work, other then the concept of letting a server sign outgoing
data, which might be the only solution to avoid an overhead in user
interaction caused by recurring events.

Computations based on asymmetric cryptography usually is slower then the
ones based on symmetric cryptography
{[}\protect\hyperlink{ref-book_2014_chapter-10-5-asym-random-number-gen}{129}{]},
but since there are no timing constrains when interacting with the
\emph{PDaaS}, regardless of whether it's external communication with
\emph{data consumers} or internal between components, parameters for
cryptographic procedures can chosen as costly as the system resources
allow them to be, thus the level of security can be increased.

\emph{\textbf{Conclusions:}} Based on the several requirements and
distinct advantages of the two authentication mechanisms, it is
preferred to use asymmetric cryptography in combination with
\emph{HTTPS} for the communication between the system and \emph{data
consumers}, where the system provides its own \emph{PKI}. Whereas a
token-based authentication on top of \emph{HTTPS} and public CAs should
be suitable for communication between the system and the
\emph{operator}, preferable based on
\emph{\protect\hyperlink{def--jwt}{JSON Web Tokens}}, because the
session state is preserved within the token rather then having the
system itself keeping track of it. Though, it is worth mentioning, that
a JSON Web Token implementation is feasible as well to fully replace the
approach of \emph{two-way authentication} and private \emph{PKI}. The
disadvantage here would be, whether \emph{data consumers} are able to
authenticate themselves or not, a HTTPS connection will establish in any
case. At the same point, authenticating the \emph{operator} is as well
doable on the TLS layer; but this approach is restricted only to trusted
environments like native mobile applications, because browser-based
applications are not considered trusted and they missing of certain
capabilities. Addressing the requirement of \emph{consumers} to verify
whether the certificate, presented by the \emph{PDaaS}, can be trusted
or not, both solutions, providing self-signed certificate on a secure
channel upfront or using certificates certified by publicly trusted
entities are legitimate. Although, the latter requires a service or an
automatism that provides a new signed certificate whenever a new
\emph{data consumer} registers, such dependencies should be kept to an
absolute minimum. To hardening an authentication procedure often one or
more factors are added, for example an \emph{eID card} or one-time
password. This adds complexity to the procedure and thus increases the
effort that is needed to perform a successful attack. But equally it
also increases the effort to support those factors in the first place.
Using multi-factor authentication is generally valued and will be
briefly noted as an optional security enhancement for the \emph{operator
role}. However detailed discussions regarding this topic are left to
follow-up work on the specification.

\hypertarget{data-reliability}{\section{Data
Reliability}\label{data-reliability}}

Within the section \protect\hyperlink{authentication}{about
authentication} it was discussed how to preserve data integrity -
referring to possible man-in-the-middle attacks and alike. Furthermore
it was described how to authenticate the different user roles so that
their identities are ensured, though, authenticity of the actual data a
\emph{PDaaS} provides has yet to be provided. Data authenticity here
refers to authentic and reliable (\protect\hyperlink{sa04}{S.A.04})
data, which means (A) the data really represent the entity that is
associated to the originating \emph{PDaaS} and is thus owned by that
entity, and (B) the data is true at that moment when the data, attempted
to be accessed, is queried from within the system. Since the
\emph{operator} can change the data at any point in time, this property
requires a process where a trustworthy third party has to be able
somehow to verify the reliability of the data in question. That process
on the other hand, is in direct contrast to the discussion about the
\protect\hyperlink{authentication}{authentication system} and why it
should be designed to be selfcontained. But if providing information on
data reliability is not required, it won't be an issue anymore. The
information can be defined in a response as an optional property. Within
the request the \emph{data consumer} has to indicate whether the
response should contain information about its reliability or not.
Depending on what data is requested, the \emph{PDaaS} then decides to
test the reliability. Based on the procedures that are available, the
data reliability gets verified somehow.

But how does this reliability check exactly should look like? It comes
down to two general steps. The first one is matching the actual data
involved in that request against a reference dataset. The second step is
optional, although important for the \emph{data consumer} in order to
evaluate the sufficiency of the provided level of reliability. It
involves the party, that also runs the first step, to confirm the result
of that audit. The result of that evaluation then gets included in the
response.

The following proposed methods vary in the provided level of reliability
as well as in the amount of effort to support them and in the possible
impact to its surrounding system. Not all data points are necessary to
test for reliability. Profile data for example are more likely to be
tested, whereas consumption lists or location histories are more or less
hard to verify, because currently there is no reliable way to verify the
origin of those datasets, besides from being not a primary focus here.

\begin{enumerate}
\def\labelenumi{(\arabic{enumi})}
\item
  \textbf{Local Verification by matching}\\
  The probably simplest and at the same time least reliable method is to
  just look at the existing data tha is stored in the database and
  matches them against those data that is used to create a response.
\item
  \textbf{Local Verification and signing}\\
  An electronic ID card can serve as an authentication token for the
  \emph{operator}, but it can also be utilized to verify the reliability
  of certain data. Using the german implementation \emph{(nPA)} as an
  example, the \emph{eID} feature would provide access to the owner's
  basic profile data, which thus can be used to match against those data
  points that are both, hold by that \emph{nPA} and going to be used to
  create the response, originating in the \emph{PDaaS} persistence
  layer. If the result of that matching procedure is positive, the
  related data then gets signed with a \emph{QES} courtesy of the
  \emph{data subject's} \emph{nPA}. That signature also gets included in
  the response, so that the recipient can verify the reliability of the
  data.
\item
  \textbf{Remote Verification and signing}\\
  Another method involves a third party who also has the same data that
  needs to be tested. The idea is to hand the data in question over to
  that party, who then tries to match against all those data points
  available in that context. The party also has the ability to sign
  data, which is what's happening, if the matching procedure has a
  positive outcome. It is required to sign the whole response or at
  least a replicable dataset that contains the data that were initially
  required. The party then hands everything back to the \emph{PDaaS} for
  further processing.
\item
  \textbf{Recurring Certification}\\
  The following method describes a modification of (3). This method
  involves no matching procedure. The external third party, verifies
  whether the data in question are correct and whether they relate to
  the \emph{data subject} by either literally looking at the data or by
  automatically processing a matching against their databases. If that
  party is satisfied, a certificate will be issued. This certificate
  contains an expiration date, which implies the consequence of going
  through this process again in the future, much like an issuing process
  of a common \emph{Certificate Authority}. This certificate is then
  served as part of a response, which enabled the \emph{data consumer}
  to verify the data's reliability on its own. This is done by hashing
  the data in question, decrypting the hash embedded in the certificate
  and matching one against the other. If they are equal, the data has
  not changed since the party's review and is therefore reliable.
  Though, if data has changed in the \emph{PDaaS} and data points are
  affected, that are also included in this verification process, then a
  new certificate needs to be issued, because the containing hash is now
  invalid.
\end{enumerate}

Only one method per request should be used to verify data reliability,
because every method can imply a different level of confidence. As
described above the response send back to the \emph{data consumer} has
to indicate the used method. Based on that the \emph{data consumer} is
then able evaluate that level und can act accordingly (e.g.~verify a
signature).

Expanding those verification procedures is reasonable, but to keep it
simple for now this aspect won't receive further attention, since the
current requirements are sufficiently met. It will left to future work,
though.

\emph{\textbf{Conclusions:}} The signing procedure as part of local
verification method involve private key and certificate stored on the
operator's \emph{eID card}. Every time the \emph{PDaaS} verifies data
reliability that method has to be went through. Thus the \emph{operator}
is forced to interact wit the \emph{PDaaS}. Otherwise the operator's
private key needs to be stored somewhere within the \emph{PDaaS}. No
matter where or how, that would potentially expose a highly confidential
part of a cryptographic procedure. Not only would this reduces the
overall security level of the system, it also makes every task this
method is involved in vulnerable to certain attacks. Aside from that,
it's highly unlikely that an \emph{eID card} would allow to extract it's
containing private keys. This all results in increased inconvenience
which is inevitable for this proposed method. The \emph{Local
Verification and signing} method has the same dependencies too mentioned
in the discussion about requirements for using the (german) \emph{eID
card} as an authentication token. And since it was rejected because of
those dependencies and because of the inconvenience mentioned before,
that verification method eventually is not going to be supported in the
specification.

The \emph{Remote Verification and signing (3)} method would require the
external party to be an official authority, because no other entity has
(A) the data in question (primarily profile data), which makes them (B)
legally binding; and they are commonly trusted. The same goes for The
\emph{Recurring Certification (4)}, but while the \emph{Remote
Verification and signing} method introduces a very strong dependency to
that external party, the \emph{Recurring Certification}\\
offers a simple loosely linked dependency. Whose design would make it
even possible to obtain such a certificate manually but automate it on
the other side as well. Nevertheless, both provide a trustworthy
certification.

Finally the first method, \emph{Local Verification by matching (1)},
which just performs a matching of two datasets against each other. Those
datasets are obtained from the same \emph{PDaaS} storage, but at
different times; right before the request finally gets proceeded,
though. The primary purpose of addressing the issue of data reliability
rests on the \emph{data consumer's} concern of accessing data that is
intentionally tampered with (e.g.~by the \emph{data subject}), since
integrity during transport is already ensured. In this light, and even
if the whole system would be compromised, which in that case might need
more urgent addressing then ensuring data reliability, it won't mitigate
the fact, that the \emph{operator} is the only one able to change data.
Hence it provides the lowest level of reliability.

Certain fields of application of a \emph{PDaaS} as a data resource might
already impose some constraints about the level of reliability and maybe
even how it can be provide. Violation of relevant legislation or other
rules might prevent the \emph{PDaaS} from being put in use. Others
instead - depending on their guidelines and business model - don't rely
on a certain level of confidence. In general, \emph{data consumers} are
expected to already have a basic confidence in a \emph{PDaaS} and in the
data originating there. Regardless of that, providing an indication on
the authenticity of data is valued as a first and important step towards
a fully working feature. All of the proposed verification methods have
some downsides, Though, the \emph{Recurring Certification} method would
be the least invasive and therewith an adequate choice.

A primary goal for the \emph{PDaaS} is to preserve all data owned by the
\emph{data subject} and giving her control over where the data might go;
not providing sufficient proof for the data authenticity.\\
Though, it is still important, to provide \emph{data consumers} with an
information about the level of reliability, but it is up to them how to
rate that information and how to act on that.

\section{Access Management}\label{access-management}

In the subsequent section it will be discussed, how several processes
around the topic of \emph{data consumers accessing data on the PDaaS}
can be modeled, what consequences certain variations might have and what
issues need to be addressed.

Below a general design is proposed of how \emph{data consumers} get
authorized and thereby are able to access the \emph{data subject's}
personal data, and how
\protect\hyperlink{standards-specifications-and-related-technologies}{previous
mentioned technologies} can be assembled in order to meet the specified
\protect\hyperlink{requirements}{requirements}.

\subsection*{Part One: consumer
registration}\label{part-one-consumer-registration}
\addcontentsline{toc}{subsection}{Part One: consumer registration}

\begin{enumerate}
\def\labelenumi{\arabic{enumi})}
\setcounter{enumi}{-1}
\item
  The \emph{operator} creates a new unique URI in the system
\item
  \textbf{Prepare registration}; the \emph{operator} has to tell the
  third party were and how to register as a \emph{data consumer} by
  handing over a URI uniquely associated to the current registration
  process. \emph{Several things need to be noted here. First, the
  operator `pulls' consumers into the system. This is the only way for a
  consumer to establish a relation. If consumers were able initiate this
  process on their own without the operator's involvement, it would be
  much harder for the system to detect spam or fraudulent requests.
  Second, handing over that URI must be done over a secure channel.}
\item
  \textbf{Send registration attempt}; the third party then makes the
  actual attempt to register as a \emph{data consumer} by providing
  required information. Those information have to be include some kind
  of feedback channel (e.g.~URI) so that the system can get back to that
  third party. They also may contain a first \emph{permission request}.
\end{enumerate}

\emph{NOTICE: the two initial steps can be skipped if the third party
would rather present the information mentioned in step 2), as a QR-Code
so that the }operator* can obtain it and thereby is able to proceed.
This approach would short cut and hence simplify the process.*

\begin{enumerate}
\def\labelenumi{\arabic{enumi})}
\setcounter{enumi}{2}
\item
  \textbf{Review registration attempt}; the operator gets notified about
  a new registration attempt, which she then has to review and decide
  whether to accept it or not.
\item
  \textbf{Create permission profile}; if a \emph{permission request} is
  enclosed in the registration, it has to be reviewed as well. If it's
  not, step follows immediately. If permissions are granted a new
  \emph{permission profile} is created. Optionally, it could also be
  created if the permissions were refused. It's just meant for the
  \emph{operator} to keep track of her decisions.
\item
  \textbf{Respond to third party}; regardless of the decision, the third
  party get's then informed via feedback channel about th decisions and
  is also provided with further details required to obtain actual data.
\end{enumerate}

\subsection*{Part Two: obtain data}\label{part-two-obtain-data}
\addcontentsline{toc}{subsection}{Part Two: obtain data}

\begin{enumerate}
\def\labelenumi{\arabic{enumi})}
\setcounter{enumi}{-1}
\item
  A successful registration as a \emph{data consumer} is required
\item
  \textbf{Send request}; the \emph{data consumer} sends \emph{access
  request} to the system, containing a all information about what data
  is needed, how to process the data and what the response should
  contain.
\item
  \textbf{Parse and check request}; after the system has received an
  \emph{access request}, first it\\
  authenticates the \emph{data consumer} and checks related
  \emph{permission profiles}. According to the defined \emph{access
  rules}, the system decides how to proceed. Either it pauses, because
  it needs further attention from the \emph{operator}, or it can start
  to process and create the response.
\item
  \textbf{Compute response}; how the computation would look like mainly
  depends on the contents of the \emph{access request} and also on what
  a \emph{permission profile} determines (see `types of access requests'
  below).
\item
  \textbf{Respond to consumer}; handover the computed response back to
  the requester by proceeding with one of the two following options.
  Either the system responses with a current status of the process and
  where the \emph{consumer} will/can find the demanded data, or the
  \emph{consumer} includes a callback URI, which the system has to
  invoke with demanded response.
\end{enumerate}

With respect to the requirements (\protect\hyperlink{sp01}{S.P.01}),
personal data should not leak into the outside. To tackle this issue,
the following three types of \emph{access requests} are defined,
starting with the most sufficient solution:

\begin{enumerate}
\def\labelenumi{\alph{enumi})}
\tightlist
\item
  \textbf{Supervised Code Execution}; \emph{access requests}
  additionally come with an executable program - binary or source code -
  potentially including information about provisioning. After the
  required data is retrieved from the storage, the program gets invoked
  with the data locally on the system but within a completely separated
  environment \emph{(sandbox)}. The result of that invocation gets
  returned to the system.
\item
  \textbf{Data DRM\footnote{\emph{Digital Rights Management} - set of
    technologies, that are used to control access to data or content
    that is restricted in certain ways (e.g.~content provided by video
    streaming)}}; after data is retrieved from the storage it gets
  encrypted. The cipher is included in the response. Upfront, \emph{data
  consumers} are equipped with a small program, that can connect to the
  \emph{PDaaS} and has to wrap the \emph{consumer's} own software that
  is planned to proceed the requested data. Now when \emph{consumers}
  receive the response, the program needs to get invoked with the
  cipher, so that, by priorly fetching the key from the \emph{PDaaS},
  the cipher can be decrypted from within the invocation. Thus the data
  is made available to the wrapped software and only during runtime.
  After the invocation has finished the program needs to propagate the
  results that are returned by the software back to the outer
  environment.
\item
  \textbf{Plain Forwarding {[}default{]}}; retrieve data from the
  storage, quick-checking the result and forwarding it directly into
  response.
\end{enumerate}

So, the data won't leave, unless the \emph{PDaaS} doesn't support any of
the proposed request types or the \emph{data consumer} provides no
alternative, so that the fallback type must be applied. If that's the
case, the overall confidentiality of all personal data is still
preserved, because all communications from and to the \emph{PDaaS} are
generally happening over HTTPS anyway, so that the data is encrypted
during the transport.

The concept of authorizing a data consumer to get the ability of
accessing personal data is fairly trivial. During (or after) the
\emph{registration} consumers have to provide detailed information about
their intentions, so that the operator is confident about their
permissions when reviewing them. The created \emph{permission profile}
reflects the result of that review. Such a permission profile defines
what data points are requested and eventually permitted to access, how
often they can be accessed and how long those permissions last. The
latter is defined as \emph{permission type} and is either one of the
following:

\begin{description}
\tightlist
\item[\emph{one-time-only}]
access permissions are hereby granted for just a single \emph{access
request} (with tolerating certain errors regarding the communication
layer)
\item[\emph{expires-on-date}]
access permissions are hereby granted until the defined point in time
has arrived
\item[\emph{until-further-notice}]
access permissions are hereby granted until the \emph{permission type}
has changed or the \emph{permission profile} has been deleted
\end{description}

\emph{NOTICE: The default permission type should be configurable. The
operator can change all permission profiles at any point in time.}

Among other information, an access request contains the \emph{data
query} that shows very precisely what data points are affected by that
request. So if an \emph{access request} arrives at the \emph{PDaaS},
assuming the \emph{data consumer} has been authenticated sufficiently,
the systems (0) searches for a permission profile that correspond to the
\emph{data consumer} and the requested data points. If it fails to find
one, the access request gets refused. But if it does, then it checks (1)
if the permission type suffices at that moment and (2) if the query only
contains data points that are enabled in the profile as well. Here the
order does matter, because it is imaginable that the operation behind
(1) is less complex then operation (2). So, at the end running (1)
before (2) can result in a lower response-time, if operation (1) already
results negative. If all operations have a positive result, access is
granted.

As stated in the section about \protect\hyperlink{data-reliability}{data
reliability}, the \emph{data subject} is able to add, change or remove
all her data or even the \emph{permission profiles} at any point in
time. This raises the question of how to solve the situation were
\emph{access requests} are being processed, while those changes are
happening and might affect the result of those requests. The first and
simplest approach would be to not address this issue at all, but that
would be unreasonable, because providing data to the \emph{consumer}
normally means for the \emph{data subject} to get something in return or
to somehow benefit from that. So that approach is no option. Using a
failure of reliability verifications as a mechanism to re-request data
won't work either in that case, because it would be based on a wrong
assumption, since that failure can have multiple causes, not only the
issue here in question. A stateless solution seems to be not fitting due
to the time-related dependency. So the only currently perceivable way is
to keep track of all momentarily processing or pending \emph{access
requests} to detect those who are affected by the changes so that each
of them can be aborted and processed again. Here it is important to
determine the right moment, when all changes are done, otherwise the
system might end up restarting those computations repeatedly within a
short amount of time. The described issue relates to both,
\emph{personal data} and \emph{permission profiles}, because either can
impact the response that is returned to the \emph{data consumer}.
Furthermore, it needs to be ensured that only after \emph{permission
requests} are being reviewed and \emph{permission profiles} are being
created, the \emph{data consumers} receive their credentials or a
notification to get started.

It is up to \emph{data consumers} to decide which data they are
requesting to access, but how do they know what data can be requested?
The only option is to expose information about data availability
(\protect\hyperlink{si02}{S.I.02}), which can be done in a variety of
ways. First, those information can be made publicly available via URI,
providing a Machine-readable format, so that information can be
processed automatically by consumers. It is also feasible to restrict
that access to registered \emph{consumers} only, in order to prevent
those information from being crawled. They could be valuable as meta
data and therefore used in undesired processing that could raise privacy
concerns. It is imaginable to let the \emph{operators} restrict the
access to such availability information on an individual consumer-basis
or system-wide, and furthermore to set default configurations for this
behaviour. Depending on those configurations request might fail, thus
requester need to be provided with meaningful errors. Http error codes
{[}\protect\hyperlink{ref-web_spec_http-error-codes}{130}{]} might be a
sufficient fit for that purpose.

An already standardized way to implement authorization is the
\protect\hyperlink{def--oauth}{OAuth} Specification. And since the TLS
layer is already in place to handle authentication, the choice would be
to use version 2 of the standard, because it relies on HTTPS. Only two
of the four \emph{grant types} provided by OAuth would match with the
process design introduced before. The types are \texttt{password} and
\texttt{client\_credentials}, which basically require identifier(s) and
secret or credentials to directly obtain a \texttt{token}. The other two
types define additional steps and interactions involving consumer and
operator before getting the \texttt{token}, because these other two
types are intended to be used for processes involving untrusted
environments (e.g.~browser, third party apps). Aside from not fitting
into such scenarios it would also make the proposed process undesirably
more complicated. Although the proposal includes user interactions like
selecting and confirming requested permissions. According to the
documentations
{[}\protect\hyperlink{ref-web_spec_oauth-1a_client-reg}{131}{]}
{[}\protect\hyperlink{ref-web_spec_oauth-2_client-reg}{132}{]}, both
OAuth versions (1.0a and 2) require the client (here \emph{data
consumer}) to register to the authorization server upfront (to obtain a
\texttt{client\_id}), before initializing the authorization process.
However, as stated before, the concept of the \emph{data subject}
`pulling' a \emph{data consumer} towards the \emph{PDaaS} is preferred
over letting \emph{data consumers} try to `push' themselves towards the
system. The reason to prevent undesired registration attempt, is,
because they all have to be reviewed by the \emph{data subject}.
Furthermore, it is not within the scope of the OAuth Specification to
define how this should be accomplished. Thus, such step needs to be
added in addition to an entire OAuth-Flow, which might cause otherwise
avoidable overhead in user interactions. Moreover, the proposed design
does not include that specific registration process either. Instead,
this process is not needed at all, because according proposal, client
identification happens implicitly as a result of how the resource owner
(operator) obtains the registration request from the client (see
\emph{Part One: consumer registration, step 0 to 2}). Further
investigations show that the semantic of an \texttt{access\_token} from
the perspective of an resource server consists of (A) authentication
{[}does this token exist?{]} and (B) authorization {[}Is this token
valid and what does it permit?{]}. Those aspects are in part already
provided by the proposed way of using the TLS layer. Because every data
consumer has its own endpoint to connect with the \emph{PDaaS} and the
certificate used by the \emph{consumer} is singed by a signature only
used for that endpoint. This means, the consumer is already
authenticated, when the TLS connection has successfully established. And
since \emph{permission profiles} relate to a specific endpoint, it would
make providing an \texttt{access\_token} to become obsolete. To
summarize, implementing OAuth would introduce several mechanisms that
otherwise can be provided by the combination of \emph{two-way
authentication} in TLS, dedicated endpoints and certification.

\emph{\textbf{Conclusions:}} In the preceding text, various solutions
were developed, based on which, the following three solutions are being
at disposal:

\begin{enumerate}
\def\labelenumi{\alph{enumi})}
\tightlist
\item
  OAuth 1.0a and HTTP
\item
  OAuth 2 and HTTPS (public Certification and PKI)
\item
  HTTP over TLS with \emph{two-way authentication}, private PKI,
  sub-domains as dedicated endpoints
\end{enumerate}

The solutions a) and b) require an extra step in which data consumers
need to register themselves at the \emph{PDaaS}. This already must be
done on a secure channel to prevent man-in-the-middle attacks.
Furthermore in that step does option a) obtain a symmetric key for
creating signatures used to ensue confidentiality and integrity in
subsequent steps. All those cryptographic procedures need to be adopted
when implementing the here proposed specification and also when
interacting with those implementations. While this can cause much more
harm, it is proposed to leave as much of these sensitive parts as
possible to existing implementations who have already proven themselves.
Thus HTTPS is mandatory, which makes b) more suitable over a), because
it's also more flexible and easier to implement. Whereas solution c)
moves the complete authentication procedure to a different layer. It
hence results in separating authentication and authorization from each
other, leaving no remains of relation. This opens the authorization
design up to for example other implementations that might be more
suitable for certain \emph{data types}. In addition, it would only
require little effort to support the case where multiple \emph{data
consumers} share the same \emph{endpoint} and thereby the same
\emph{permission profiles}. Combining b) and c) would result in
significant redundancy, since both solutions have much overlap in their
provided features, even though b) aims to be a framework for
authorization. The process description in the beginning of this section
is used as the foundation of \emph{access management} in the
\emph{PDaaS}. Implementing OAuth based on this design would leave
nothing from the framework, but a simple request returning an identifier
associated with permissions. And even these identifiers are obsolete
when combining TLS with dedicated \emph{consumer}-specific endpoints, as
c) states. So, there is not much benefit in using OAuth, other then
developers might be somewhat familiar with the API. This can be
addressed by a detailed specification for this project, hence c) is
preferred over b). At the end, the only suitable use case from this work
would consists of just a request that obtains a token after
authenticating with the provided credentials. And since OAuth only
provides a framework for how to authorize third parties to access
external resources, but leaves the procedure of how to actually verify
those access attempts up to its implementers
{[}\protect\hyperlink{ref-web_spec_oauth-1a_access-verification}{133}{]}
{[}\protect\hyperlink{ref-web_spec_oauth-2_access-verification}{134}{]}.
In the context of this project OAuth does not comply with the rest of
the design aspects.

How the first steps of a registration are look like, is up to the
\emph{consumer}, even though the option involving a QR-Code might result
in a nicer user experience from the perspective of a data subject. In
any case, a secure channel is vital.

When accessing personal data, at the same time preventing those data
from leakage is almost impossible, which originates in the nature of
digital data being able to get effortlessly copied. Nevertheless it is
possible to make it much more difficult, so that it becomes inefficient
to bypass those mechanisms. At the same time it requires some effort to
establish, run and maintain the infrastructure needed for those
mechanisms. In case of the \emph{Data DRM} proposal that effort is not
proportionate, because it requires additional infrastructure, interfaces
and cryptographic procedures and introduces therefore new attack
scenarios. For now the only approach being considered, is the
\emph{Supervised Code Execution}, aside from defaulting to simple
forwarding. When implementing this approach, two directions might need
to be considered. Alongside the executable program, \emph{data
consumers} either provide all dependencies so that everything is bundled
up, or don't provide any dependency at all. The latter is preferred,
because it reduces the amount of potentially malicious, flawed or
needless components, so that the data subject, supported by her
\emph{PDaaS}, has more supervising capabilities and thus more control
over her personal data. Since the overall goal here is to prevent the
data subject from loosing control over her data, it might also be
conceivable, that certain categories of personal data, representing a
higher level of sensitivity, also require a least sufficient
\emph{request type}. If the data consumer does not comply, access will
be refused. Also, depending on which category the personal data relates
to, the \emph{PDaaS} might be able to somehow anonymize certain types of
data, if it is even capable of doing so, because the \emph{consumer} at
least supposedly knows what individual is behind the \emph{PDaaS} it is
currently interacting with. The field of \emph{data anonymization} is a
large research area on its own, which recently started to gain a lot of
traction due to emerging privacy concerns about \emph{Big Data}. Thus it
will be left for future work.

\hypertarget{data}{\section{Data}\label{data}}

The core task of a \emph{PDaaS} is providing data, \emph{personal data},
which in conjunction is the digital manifestation of an individual, a
person. One party creates the data, another one obtains and processes
it. Thus, both need to agree, or at least need to know, how that data
looks like, how is it structured and what are their semantics. The
following section is intended to discuss different technologies, used to
create queries that obtain those data points that are desired. Further
on, it describes some basic data types and schemas, that might be useful
in the context of \emph{personal data} and also for previously
introduces \protect\hyperlink{scenarios}{scenarios}.

First of all, to address the need of portability, which has to be
satisfied by those components, who are storing and providing
\emph{personal data}, it is essential to abstract the actual storage
from how it gets accessed. This makes it possible to relocate those
storage onto other platforms and environments. Thereby the
\emph{personal data storage} itself becomes platform-agnostic from an
outside perspective, in other words portable. In order to reduce
possible issues related to unsupported communication protocols it might
be reasonable to enforce HTTP - over TLS, if they don't share the same
environment - even if the storage therefor requires an additional driver
or proxy layer, like for example a mobile app.

Possible technologies are for example
\emph{\protect\hyperlink{def--graphql}{GraphQL}} or the
\emph{\protect\hyperlink{def--sparql}{SPARQL}}, which is part of the
\emph{\protect\hyperlink{def--semantic-web}{Semantic Web Suite}}. Both
are query languages underpinned by the concept of a graph. This means,
relations between data points are embedded within the data structure
itself. That meant, in terms of a graph, relations are \emph{edges} and
data points are \emph{nodes}. In consequence the structure of a query
itself reappears in its result, which means the originator of that query
knows exactly what to expect for the response. Therefore it's not
necessary to provide any additional information about how to handle and
interpret the responded data. The code examples
(\protect\hyperlink{code-01_sparql-query}{Code 01} and
\protect\hyperlink{code-02_sparql-query-results}{Code 02},
\protect\hyperlink{code-03_graphql-query}{Code 03} and
\protect\hyperlink{code-04_graphql-query-result}{Code 04}) give a first
impression of how it might look like, when a \emph{consumer} obtains the
name of the \emph{data subject} and a bank account of hers, that
supports online payment.

Without going into much details here, the syntax of the SPARQL query
(\protect\hyperlink{code-01_sparql-query}{Code 01}) already shows its
nature of decentralization. This aspect at the same time introduces
additional external dependencies. Because the query language itself has
no concept of schemas or any kind of semantic, it needs to be made aware
of them. SPARQL queries typically return XML which then can be rendered
into (HTML) tables. JSON and RDF are also supported. The reason for
performing two queries in the example instead of just one, is, because
otherwise the result would have returned multiple `rows' with redundant
data, if more then one bank account would have supported online payment;
varyingly in those three columns data contain data about bank accounts,
but being identical in the fields related to the profile information,
though.

Whereas the GraphQL query syntax
(\protect\hyperlink{code-03_graphql-query}{Code 03}) compared with its
result (\protect\hyperlink{code-04_graphql-query-result}{Code 04}) shows
of a remarkable resemblance. By defining \texttt{paymentMethod} as an
argument, the resolver for \texttt{bankAccounts} then implements an
instruction to match the value of that argument
(\texttt{\textquotesingle{}online-service\textquotesingle{}}) against
the whole set. GraphQL's server then knows from which resources the data
in question need to be pulled and how they need to be aggregated. While
SPARQL has a full-featured query language syntax including all sorts of
controls (e.g. aggregation, operators, nested queries etc.), GraphQL's
syntax instead is more rudimental, because all its functions and logic
has been abandoned from the language itself and put into a server part.
With this concept of separation it is straightforward to validate
queries, because it essentially means matching against types. Both query
languages share a comprehensive understanding of a type-system, that
encourages to create all kinds of data types. While in GraphQL common
schemas still need to be created, SPARQL already provides a reasonable
amount of vocabulary
{[}\protect\hyperlink{ref-web_w3c-tr_rdf-schemas}{135}{]}. However, when
comparing the results of both languages, some distinctions appear. While
in GraphQL the characteristics of graph-structured data are remain,
SPARQL's output is missing a certain level of depth. The reason for that
originates in the design of the query language and its syntax. SPARQL is
able to notice implicit relations between data points, though its query
language is not capable of grabbing and presenting them. Thus the result
(\protect\hyperlink{code-02_sparql-query-results}{Code 02}) only
consists of two dimensions.

It is crucial for the \emph{PDaaS} to provide the \emph{data subject}
with abilities to create her own data types and schemas
(\protect\hyperlink{sp03}{S.P.03}). Thereby she is enabled to serve data
points according to her own needs and terms. In order to interact with
their customers or users, \emph{data consumers} might as well develop
and provide schemas for their requests. This can help \emph{data
subjects} to speedup the process of permission granting and to easier
understand what data points are affected. Data types and schemas are the
key to validate incoming and outgoing data. If data violates the
underlying schema or no appropriate schema exists, the data transfer
fails. Other missing data types could be developed by a community,
because not every \emph{data subject} might be capable of modeling her
own data types. Thus everyone can benefit from that effort taken by a
few. As a result, the ones that are widely used might then become de
facto standards. Moreover, it's also possible that several data types
will emerge, that are based on common standards, for example
\emph{medical record}
{[}\protect\hyperlink{ref-web_spec_data-schemas_ehr}{136}{]},
\emph{point of interest}
{[}\protect\hyperlink{ref-web_spec_data-schemas_poi}{137}{]} or
\emph{bank transfer}
{[}\protect\hyperlink{ref-web_spec_data-schemas_bank-transfer}{138}{]}.
With that approach those data types can be viewed as something like a
plugin or add-on to the \emph{PDaaS} ecosystem.

In order to avoid confusion about the differences between types and
schemas and to simplify their relations, the following two definitions
are henceforth being used. A (data) \emph{type} is the superior term;
hence refers to both of them.

\begin{verbatim}
*Float* or *Nil (null)*
\end{verbatim}

\begin{verbatim}
*primitives*, but can consist of other structs as well
\end{verbatim}

Based on these two concepts almost any imaginable data type can be
modeled. A selection of such types can be found in the list of
\protect\hyperlink{list-01_suggested-structs}{suggested structs (List
01)}, whereas an extract of (sub-)categories that might be useful in a
\emph{PDaaS} are specified in a list of
\protect\hyperlink{list-02_data-categories}{data categories (List 02)}.
Additional examples for \emph{structs} are a
\protect\hyperlink{code-05_struct_profile}{\emph{data subject's} profile
(Code 05)}, a \protect\hyperlink{code-06_struct_contact}{contact (Code
06)} and bare \protect\hyperlink{code-07_struct_position}{position
information (Code 07)}. All those examples and lists are only to be
understood as a starting point that should cover basic scenarios as well
as to give a first impression of what data types a \emph{PDaaS} could
provide.

\textbf{\protect\hypertarget{list-01_suggested-structs}{}{List 01:
Suggestions for useful structs}}

\begin{itemize}
\tightlist
\item
  Address
\item
  Contact
\item
  Location

  \begin{itemize}
  \tightlist
  \item
    Country
  \item
    City
  \item
    Position
  \end{itemize}
\item
  Media

  \begin{itemize}
  \tightlist
  \item
    Audio
  \item
    Video
  \item
    Photo
  \end{itemize}
\item
  Organisation
\item
  Date
\item
  TimeRange
\item
  Language
\item
  Diseases
\end{itemize}

\textbf{\protect\hypertarget{list-02_data-categories}{}{List 02:
relevant (sub-)categories of data}}

\begin{itemize}
\tightlist
\item
  Finance

  \begin{itemize}
  \tightlist
  \item
    Income
  \item
    Bank transfers
  \end{itemize}
\item
  Shopping history
\item
  Product
\item
  Things/Objects
\item
  Media consumption

  \begin{itemize}
  \tightlist
  \item
    Music playlist
  \item
    Watchlist
  \end{itemize}
\item
  Favorites/Interests

  \begin{itemize}
  \tightlist
  \item
    Music genres
  \item
    Songs
  \item
    Movies
  \item
    Books
  \item
    Travel destinations
  \item
    Topics
  \end{itemize}
\item
  Curriculum vitae (CV)

  \begin{itemize}
  \tightlist
  \item
    Educational level
  \item
    Visited schools
  \end{itemize}
\item
  Visited \ldots{}

  \begin{itemize}
  \tightlist
  \item
    points of interest
  \item
    countries
  \item
    websites/URLs (browser history)
  \end{itemize}
\item
  Units (measurements)
\item
  Organisations

  \begin{itemize}
  \tightlist
  \item
    Company
  \item
    Bank
  \item
    \ldots{}
  \end{itemize}
\item
  Medical/Health Record

  \begin{itemize}
  \tightlist
  \item
    Diseases
  \item
    Treatments
  \item
    Visits to the doctor
  \item
    Medication
  \end{itemize}
\end{itemize}

The available \emph{primitives} mainly depend on those who are supported
by the query language itself. In this case, all \emph{primitives}
mentioned above are supported by \emph{SPARQL}
{[}\protect\hyperlink{ref-web_spec_xml_types}{139}{]} and \emph{GraphQL}
{[}\protect\hyperlink{ref-web_spec_graphql_types}{140}{]}. When choosing
a database system it has to be ensured that either the system already
supports the required \emph{primitives} or they can be emulated somehow
with a least amount of drawbacks. When modelling relations between data
points one can use for example keys (or identifiers) to make reference,
or additional syntactical tools like \emph{lists} (or arrays) and maps
(or objects). Those tools facilitate readability so that relations are
almost intuitively observable, hence they should be enforced. Whereas
another known concept in data modelling, called \emph{inheritance},
isn't required, but could help to reason about certain \emph{structs}
and their representations. It might add complexity, though.

Aside form the subject's personal data other information and data must
be persisted as well. This includes for example:

\begin{itemize}
\tightlist
\item
  Application data

  \begin{itemize}
  \tightlist
  \item
    Templates (\protect\hyperlink{pi05}{P.I.05})
  \item
    Permission profiles (incl. versioning)
  \item
    Consumer information
  \item
    Meta data
  \item
    Notifications
  \item
    States
  \item
    Tokes
  \item
    Access logs
  \end{itemize}
\item
  Files

  \begin{itemize}
  \tightlist
  \item
    Cryptographic keys
  \item
    Executable program
  \item
    Container images
  \item
    Configurations
  \item
    User interfaces
  \item
    Documentations
  \end{itemize}
\end{itemize}

The list revels that not only a database system is needed to satisfy the
requirements, but the environments filesystem might need to be utilized
as well. This leads to the question of what requirements a database
system has to satisfy. But first of all it is pivotal to distinguish
between the needs of a \emph{personal data storage (PDS)} and a general
\emph{persistence layer (PL)} for the system's backend.

\begin{longtable}[]{@{}lcc@{}}
\caption{selection of characteristics that a database system has to
feature in order to be suitable for either of the defined purposes
\label{tbl:dbs-features}}\tabularnewline
\toprule
Characteristic & Personal Data Storage & Persistence
Layer\tabularnewline
\midrule
\endfirsthead
\toprule
Characteristic & Personal Data Storage & Persistence
Layer\tabularnewline
\midrule
\endhead
portable & - & -\tabularnewline
advanced user \& permission management & - & \textbf{X}\tabularnewline
document-oriented & \textbf{X} & \textbf{X}\tabularnewline
support common primitives & \textbf{X} & \textbf{X}\tabularnewline
replication & - & \textbf{X}\tabularnewline
efficient binary storage and serialization & \textbf{X} &
\textbf{X}\tabularnewline
high performance & - & \textbf{X}\tabularnewline
operations and transactions & \textbf{X} & \textbf{X}\tabularnewline
background optimization & - & \textbf{X}\tabularnewline
version control & - & -\tabularnewline
\bottomrule
\end{longtable}

Although, most of the characteristics (in Table \ref{tbl:dbs-features})
are self explanatory, certain aspects need to be commented on. Fist,
portability, an important requirement
(\protect\hyperlink{sa02}{S.A.02}), that is oddly not marked in Table
\ref{tbl:dbs-features}. This is because of the priorly introduced
concept of abstracting the \emph{personal data storage} with an
additional query language. Therefore the access to the \emph{PDS}
becomes platform-agnostic. Whereas the database system storing that data
can be implemented with respect to the requirements while considering
the environment constraints at the same time. Basic permission
management should suffice the \emph{PDS}, since it's not accessed in
multiple ways. It only relates to the query language abstraction. Data
and especially its structure is expected to be highly fluctuant
(\protect\hyperlink{sp02}{S.P.02}), thus advantages of relational
databases (e.g.~schema-oriented and -optimized) would instead harm the
performance and flexibility, because they are not primarily designed to
handle schema changes. Database systems, whose storage engine is build
upon a document-oriented approach, would be a better choice. Replication
can be utilized for horizontal scaling, federation or backups
(\protect\hyperlink{sp05}{S.P.05}). The focus here is on the latter,
because without \emph{PL} the \emph{PDaaS} wont be able to function. In
case of irreversible data loss, the whole system state is gone, which
then has to be reconfigured and reproduced from the ground up. Such
effort can be spared by introducing a reliable backup strategy. With the
\emph{PDS} on the other hand replication is not necessary, but to ensure
no data loss still needs to be addressed. Therefor every database system
that might be used for the \emph{PDS} must provide a mechanism to backup
or at least to export the data, which can be triggered and obtained
through the operator's management tool. Another approach imaginable
could be to not only store the actual data written to the \emph{PDS} but
also to save all queries in a chronological order that have somehow
changed the data. Therefore the current state can be restored just by
running those queries against the \emph{PDS}. It is reasonable to store
the queries of the abstracted query language not the ones the query
language is transformed into. If a mobile device is part of the
\emph{PDaaS}, another approach for the operator could be to perform
regular device backups. Those strategies are all just initial thoughts
which might be sufficient only as a starting point. Other solutions are
imaginable. Though, elaborating on those is beyond the scope of this
work. Depending on the technologies that are being used, it might be
necessary from a conceptional perspective to split the \emph{PL} into
two parts. One part is a database system and the other is represented by
the environment's filesystem. It might be no alternative when it comes
to key files, which are typically accessed as files, or configuration
files for certain technologies. In any case, both \emph{PL} and
\emph{PDS}, have to be able to store files of any kind, which is
required for instants to support the secenario of medical records. File
size restriction should be mandatory though, because the \emph{PDaaS}
has no intention to replace existing \emph{file hosting} solutions.

Being able to undo changes of certain data points or to review the
change-history of those data can be very useful; not only when those
changes were persisted mistakenly. This behaviour might not be necessary
for every data, especially when it comes to application configuration or
logs. Also, not every \emph{operator} might require these features.
Therefore, and because database systems with no alternative might not be
able to provide this capability, it's not required by either \emph{PDS}
nor \emph{PL}. If a history is not natively supported but still desired,
it has to be considered if for example high frequently backups would
already suffice or if a implementation on the application-level is
required.

Before serving data it needs to be at first inserted into the
\emph{PDS}. This can be done in three different ways:

\begin{enumerate}
\def\labelenumi{\alph{enumi})}
\tightlist
\item
  The \emph{data subject} uses forms provided by the graphical user
  interface to insert data about her, for example her
  \protect\hyperlink{code-05_struct_profile}{profile information (Code
  05)}. This data is then submitted into the \emph{PDaaS} which takes
  care of storing it.
\item
  The \emph{data subject} is in possession of file(s) or string(s)
  containing a data format that is supported by the system. The
  graphical user interface provides a mechanism to either upload the
  file(s) or insert the string(s). Thereby the data is then send into
  the system. If this raw data is not self-explaining to the system the
  \emph{data subject} has to provide more information on the context of
  those data.
\item
  Third party software, for example a browser plugin, is used to provide
  the \emph{PDaaS} with data; in this case a browser history. This
  software uses a restricted API provided by the \emph{PDaaS} to let
  data flow into the system.
\end{enumerate}

These three concepts, especially b) and c) are required to get inspected
for malicious content and extensively validated against existing
\emph{structs}. Only if these checks do not fail, the submitted data can
be stored. For scenario b) the \emph{data subject} needs to be asked to
review the imported data to make sure everything is as expected. When
enabling third party software to submit data, appropriate authentication
and permission mechanisms must be in place. That software is classed
like all other \emph{operator} front ends, but without permissions to
obtain data. According to the
\protect\hyperlink{authentication}{discussion on authentication}, these
mechanisms thus can be implemented through \emph{JWT}.

\emph{\textbf{Conclusions:}} In order to gain flexibility in choosing
technology and location for the \emph{personal data storage}, the
logical consequence is to abstract the interface from the database
system. Introducing a separate query language is proposed as a
reasonable approach. It can be chosen between two suggested query
languages, \emph{GraphQL} or \emph{SPARQL}. Both provide the necessary
features required to integrate them with a distributed system;
\emph{SPARQL} with its concepts of URIs as identifiers and resources,
and \emph{GraphQL} with its separation of query definition and
execution. This also effects the process of query validation, which is
much harder to do for \emph{SPARQL}, because its syntax is more flexible
and allows some shorthands, therefore the possibilities are way to many.
In general \emph{SPARQL's} syntax is harder to reason about compared to
\emph{GraphQL}. And even though the result of both languages is
formatted in JSON, only \emph{GraphQL} preserves all the relations in
the output, which are already embedded in the query syntax. In result,
\emph{GraphQL} (and its implementations) is the query language of choice
for this project.

When it comes to creating new structs engaging a user community can
compensate the lack of certain types. Examples for a potential starting
point of \emph{PDaaS}-supported data types were showed before. Data
Modelling in general is a large research field to its own. With regards
to the \emph{PDaaS} it needs much more attention, though it's beyond the
scope of this work. The basic approaches within this section should only
be viewed as an introduction that gives an outlook of how it's imagined.

\section{Architecture}\label{architecture}

By taking all requirements as well as previous sections, their and
discussions into account, this section has the purpose of figuring out
how all the different concepts and conclusions from this chapter can fit
together in an overall system architecture that is organized in either a
distributed or a monolithic manner. The outcome of this section should
not impact results or conclusions from other topics related to the
behave of the system's interfaces from a user point of view.

The foundation of this project is a server-client Architecture, which is
chosen for (A) providing availability (\protect\hyperlink{sa05}{S.A.05})
and (B) separating concerns
{[}\protect\hyperlink{ref-web_2016_wikipedia_separation-of-concerns}{141}{]}.
Such a distributed system provides various locations to place these
concerns, which are in fact different environments with different
properties. Those combinations of locations and environments are herein
after called \emph{platforms}. To further describe these
\emph{platforms}, characteristics such as architectural layer and access
possibilities to its internals are taken into account. The resulting
three \emph{platform} types are shown in Table
\ref{tbl:platforms-characteristics}.

\begin{longtable}[]{@{}lcclll@{}}
\caption{All platform types where components of the \emph{PDaaS}
architecture can be placed
\label{tbl:platforms-characteristics}}\tabularnewline
\toprule
Type & trusted & private & controlled by & Layer &
Purpose\tabularnewline
\midrule
\endfirsthead
\toprule
Type & trusted & private & controlled by & Layer &
Purpose\tabularnewline
\midrule
\endhead
Server & yes & yes & data & back & - business logic\tabularnewline
\(\ \) & \(\ \) & \(\ \) & subject & end & - third-party
interfaces\tabularnewline
\(\ \) & \(\ \) & \(\ \) & \(\ \) & \(\ \) & - data
storage\tabularnewline
\(\ \) & \(\ \) & \(\ \) & \(\ \) & \(\ \) & \(\ \)\tabularnewline
Desktop & no & no & data & front & - based on web\tabularnewline
\(\ \) & \(\ \) & \(\ \) & subject & end & \(\ \)
technologies\tabularnewline
\(\ \) & \(\ \) & \(\ \) & \(\ \) & \(\ \) & \(\ \)\tabularnewline
Mobile & no & cond. & data & front & - typically mobiles\tabularnewline
\(\ \) & \(\ \) & \(\ \) & subject & end & \(\ \) devices\tabularnewline
\(\ \) & \(\ \) & \(\ \) & \(\ \) & \(\ \) & - based on
host-specific\tabularnewline
\(\ \) & \(\ \) & \(\ \) & \(\ \) & \(\ \) & \(\ \) native
technologies\tabularnewline
\(\ \) & \(\ \) & \(\ \) & \(\ \) & \(\ \) & - data
storage\tabularnewline
\bottomrule
\end{longtable}

The next step is to determine all the components, that are required in
order to cover most of the defined use cases. The conglomeration below
highlights all major components, including information in which
platforms they could be positioned, in addition to further details about
their major task(s), underlying technologies and relation(s) to each
other. ~\\
\textbf{Web server}

\emph{Platform:} Server

\emph{Tasks:}

\begin{itemize}
\tightlist
\item
  serve web-based user interface(s)
\item
  handle all in- \& outgoing traffic (outmost layer)
\item
  revers proxying certain traffic to different components
\item
  en- \& decrypt HTTPS traffic, thus authenticate \emph{consumers}
\item
  load balancing (if necessary)
\item
  desktop notification
\item
  spam protection
\end{itemize}

\emph{Relations:}

\begin{itemize}
\tightlist
\item
  operator, consumers, third parties
\item
  any front end platform (Desktop, Mobile)
\end{itemize}

\emph{Technologies:}

\begin{itemize}
\tightlist
\item
  HTTP
\item
  TLS
\item
  WebSockets
\end{itemize}

~\\
\textbf{Permission Manager}

\emph{Platform:} Server

\emph{Tasks:}

\begin{itemize}
\tightlist
\item
  creating \emph{permission profiles}
\item
  permission validation
\item
  examine data queries
\item
  queue \emph{consumer} requests
\end{itemize}

\emph{Relations:}

\begin{itemize}
\tightlist
\item
  Storage Connector
\item
  Notification Infrastructure
\item
  Persistence Layer
\item
  Tracker
\item
  Code Execution Environment
\end{itemize}

\emph{Technologies:}

\begin{itemize}
\tightlist
\item
  any modern language/framework capable of parallel computing
\end{itemize}

~\\
\textbf{PKI}

\emph{Platform:} Server

\emph{Tasks:}

\begin{itemize}
\tightlist
\item
  CA
\item
  manage keys and certificates per \emph{endpoint}
\item
  obtain trusted certificates from public CAs
\end{itemize}

\emph{Relations:}

\begin{itemize}
\tightlist
\item
  Web Server
\end{itemize}

\emph{Technologies:}

\begin{itemize}
\tightlist
\item
  X.509
\item
  ACME {[}\protect\hyperlink{ref-web_spec_acme}{142}{]} (Let's Encrypt)
\end{itemize}

~\\
\textbf{Storage Connector}

\emph{Platform:} Server

\emph{Tasks:}

\begin{itemize}
\tightlist
\item
  abstracts to system agnostic Query Language
\item
  queries personal data, regardless of where it's located
\end{itemize}

\emph{Relations:}

\begin{itemize}
\tightlist
\item
  Personal Data Storage
\end{itemize}

\emph{Technologies:}

\begin{itemize}
\tightlist
\item
  driver for used database
\end{itemize}

~\\
\textbf{Operator API}

\emph{Platform:} Server

\emph{Tasks:}

\begin{itemize}
\tightlist
\item
  authenticates \emph{operator}
\item
  writes personal data through Storage Connector
\item
  provides relevant data, such as history
\item
  system configuration
\item
  automated data inflow
\end{itemize}

\emph{Relations:}

\begin{itemize}
\tightlist
\item
  Storage Connector
\item
  Notification Infrastructure
\item
  PKI
\item
  Tracker
\item
  Permission Manager
\end{itemize}

\emph{Technologies:}

\begin{itemize}
\tightlist
\item
  JWT
\end{itemize}

~\\
\textbf{Code Execution Environment}

\emph{Platform:} Server

\emph{Tasks:}

\begin{itemize}
\tightlist
\item
  isolated runtime (sandbox) for computations/programs provided by
  \emph{consumers}
\item
  restrict interaction with outer environment to absolute minimum
  (e.g.~no shared filesystem or network)
\item
  one-time use
\item
  monitor sandbox during computation
\item
  examine and test the provided software
\end{itemize}

\emph{Relations:}

\begin{itemize}
\tightlist
\item
  Permission Manager
\end{itemize}

\emph{Technologies:}

\begin{itemize}
\tightlist
\item
  Virtualization
\item
  Container (OCI)
\end{itemize}

~\\
\textbf{Tracker}

\emph{Platform:} Server

\emph{Tasks:}

\begin{itemize}
\tightlist
\item
  log all changes made with \emph{Storage Connector}
\item
  tracks states for ongoing consumer requests
\item
  log all \emph{access requests}
\end{itemize}

\emph{Relations:}

\begin{itemize}
\tightlist
\item
  Persistence Layer
\end{itemize}

\emph{Technologies:}

\begin{itemize}
\tightlist
\item
  any modern language/framework capable of parallel computing
\end{itemize}

~\\
\textbf{Personal Data Storage}

\emph{Platform:} Server, Mobile

\emph{Tasks:}

\begin{itemize}
\tightlist
\item
  stores the \emph{operator's} personal data
\end{itemize}

\emph{Relations:}

\begin{itemize}
\tightlist
\item
  Storage Connector
\end{itemize}

\emph{Technologies:}

\begin{itemize}
\tightlist
\item
  non relational database
\item
  depending on host environment
\end{itemize}

~\\
\textbf{Persistence Layer}

\emph{Platform:} Server

\emph{Tasks:}

\begin{itemize}
\tightlist
\item
  stores Permission Profiles, History, Tokens, Configurations and other
  application data
\item
  cache runtime data and information
\item
  holds keys
\end{itemize}

\emph{Relations:}

\begin{itemize}
\tightlist
\item
  Operator API
\item
  Permission Manager
\end{itemize}

\emph{Technologies:}

\begin{itemize}
\tightlist
\item
  non relational database
\item
  Filesystem
\end{itemize}

~\\
\textbf{Notification Infrastructure}

\emph{Platform:} Server

\emph{Tasks:}

\begin{itemize}
\tightlist
\item
  notifies about everything that needs \emph{operator's} approval
  (e.g.~new registrations, new \emph{permission requests})
\end{itemize}

\emph{Relations:}

\begin{itemize}
\tightlist
\item
  Web server
\end{itemize}

\emph{Technologies:}

\begin{itemize}
\tightlist
\item
  WebSockets for web UIs via local web server
\item
  mobile device manufacturer's Push Notification server for mobile apps
\end{itemize}

~\\
\textbf{User Interface}

\emph{Platform:} Desktop, Mobile

\emph{Tasks:}

\begin{itemize}
\tightlist
\item
  access restricted to \emph{operator} only
\item
  access \& permission management
\item
  data management (editor, types \& import)
\item
  history and log viewer
\item
  system monitoring
\end{itemize}

\emph{Relations:}

\begin{itemize}
\tightlist
\item
  Web server
\end{itemize}

\emph{Technologies:}

\begin{itemize}
\tightlist
\item
  HTML, CSS, Javascript
\item
  Java
\item
  Swift, Objective-C
\end{itemize}

~\\
After outlining all different components while keeping the aspect of
portability (\protect\hyperlink{sa02}{S.A.02}) in mind, it becomes
apparent which arrangements make sense and what variations might be
possible. As a result, two, more or less, distinct designs are proposed.
One is a rather monolithic approach and the other involve more platform
types and outlines a certain flexibility.

\begin{figure}
\centering
\includegraphics{./assets/figures/pdaas_component-composition_monolithic.png}
\caption{PDaaS Architecture, monolithic
composition\label{fig:composition-monolithic}}
\end{figure}

\begin{figure}
\centering
\includegraphics{./assets/figures/pdaas_component-composition_distributed.png}
\caption{PDaaS Architecture, distributed
composition\label{fig:composition-distributed}}
\end{figure}

The main difference between the two compositions is the non-existence of
the mobile platform in the monolithic approach (Figure
\ref{fig:composition-monolithic}). Although \emph{monolithic} only
refers to the components arrangement on a \emph{server} platform,
originally consisting of in a single process that contains all
components and is thus is responsible for every task. It is also
imaginable that all server components not necessarily have to be placed
into one server environment, but being distributed over several virtual
machines or containers, so that they can scale and run more
independently. This can improve \emph{redundancy} as well.

In theory, a possible version of the arrangement would be to move all
components to either the desktop or the mobile platform. This comes
along with some downsides and major issues though, that are anything but
trivial to solve. Nevertheless, not only to ensure nearly 100\% uptime
and discovery in a landscape where NAT\footnote{Network Address
  Translation; practice of routing traffic between and through distinct
  networks address spaces by remapping IPs from those different networks
  onto each other (e.g.~by utilizing ports)} and dynamic IPs are still
common practice, mobile platforms as well as on the desktop, all
components but the user interface need to be implemented natively. From
a \emph{operator's} perspective that would mean, to have all components
at hand and therefore full control over the \emph{PDaaS}. It still would
still raise security concerns, though.

Aside from providing the \emph{operator} with a non-stationary and
instantly accessible interface to her \emph{PDaaS}, involving a
\emph{mobile platform} primarily has the purpose of enabling the
\emph{data subject} to carry all her sensitive data along. This is
considered a major advantage over the monolithic approach, were all the
personal data is located in the \emph{`cloud'}. Depending on the
perspective, it can either be seen as a \emph{singe source of truth} or
a \emph{single point of failure}. Regardless of that, it introduces the
demand of a backup or some redundancy concept, which has briefly been
touched on in the discussion about database system requirements within
the \protect\hyperlink{data}{section on \emph{data}}. A mobile platform
as part of the system makes it more easier for the data subject to
establish a security concept, in which the relation between
\emph{personal data storage} and the rest of the system is much more
liberated, so that all access attempts only happen under full
supervision. It is debatable whether to place the \emph{permission
profiles} in the \emph{persistence layer} among all other domain-related
information, put it into the \emph{personal data storage} too, or define
it as an own storage component, in order to be flexible in its placing.

Authenticating \emph{consumers} is performed based on TLS by the web
server and its configured subdomains including their individual keys and
certificates provided by the \emph{PKI}. The \emph{operator}
authentication is either done by the \emph{Operator API} or by the
\emph{web server}, depending on the \emph{web server's} capabilities.
Though, it makes more sense, to entrust the \emph{web server} with that
task, because it's the outmost component and it would prevent
unauthorized and potential malicious requests from getting further into
the system. And since a native front end on a mobile platform is
considered \emph{private} as well, it is reasonable in that case to
change the \emph{operator} authentication from JWT-based to TLS-based
\emph{two-way authentication}, which would otherwise be inconvenient
when using web-based front ends.

If components are placed only on the server and require to communicate
between each other, but are separated into independent processes, then
some inter-process communication need to be established (e.g.~sockets).
It is also conceivable that inter-communication between server
components could just be unidirectional. Approaches like changing
configuration files by writing to the filesystem can therefor be
feasible in some cases. Components that can vary in terms of their
platform, have to communicate to other components via \emph{HTTPS}.

The architecture implicitly distinguishes between two different groups
of endpoints. These endpoints that are made available by the \emph{web
server}, which reverse-proxys incoming connections to role-related
(operator* or data consumer) components. Starting from that, this
separation can be driven further by simply encapsulating those
components into services, that either are related to one of the roles or
used by both. This basically results in the \emph{web server}
communicating with the two role-grouped services in a bidirectional
manner. The group of endpoints for \emph{data consumers} mainly consists
of those through witch \emph{access requests} and \emph{permission
requests} are coming in and the public one, that is used for when
consumers apply for registration. The other one is a small group of
endpoints required for all tools the \emph{operator} might need; from
data API or notification to authentication and web-based user interface.

Considering the rapid growth of emerging website and applications, which
all require user registration, users are getting tired of creating new
accounts. Hence they tend to reuse their password(s). Providers started
outsourcing that sensitive topic of user management by integrating third
party authentication services, which not only makes that feature almost
effortless to implement, but also leaves the responsibility as well as
the accessibility to those service owners. Whereas users get the benefit
of just using one account for all their apps - a universal key so to
say, but only one exemplar. So the downside here is, in reality only a
handful of third parties
{[}\protect\hyperlink{ref-web_2009-success-of-facebook-connect}{143}{]}
provide those authentication services.\\
OpenID is designed with a very specific type of scenarios in mind,
namely the one just described - bringing decentralisation to the market
of authentication services - which differs from the ones addressed by
the \emph{PDaaS}; at least, when it comes to \emph{data consumer}
interactions. Even though, the \emph{PDaaS} has the ability to become
the digital representation of it's \emph{operator}. Hence it can and
also should be used to authenticate that individual against external
parties.

\emph{\textbf{Conclusions:}} Considering the amount of effort a
single-platform composition, namely \emph{desktop} or \emph{mobile},
would take to get fully operational with respect to the specification,
it is not only reasonable but also more secure to involve a server
platform with proper security measures, static IP and high availability.
Even if that server is a local machine connected to the operator's
private network. That said, it is sufficient to start with the
\emph{monolithic} approach and as suitable mobile applications emerge
that are supporting major administration features, notifications and
\emph{personal data storage}, it should be possible to migrate
effortlessly towards the \emph{distributed} approach that brings a
higher level of confidence, because all the sensitive personal data is
not on some computer machine somewhere on the internet, but right in the
hands of its owner. By the proposed architecture, all components (or
groups of components) are portable and therefore relocatable among the
suggested platforms; and with the introduced authentication methods for
operators multiple front ends for the same \emph{PDaaS} are thereby
supported and can be implemented with almost no effort, and that again
covers more use cases. As a supplement, an \emph{identity provider}
based on the OpenID standard would fit nicely into the existing
arrangement and does not interfere with the other components. However,
it is beyond the scope of this work to elaborate on this topic. For now
it is stated as a feasible and logical addition to the \emph{PDaaS}.

\section{Environment and Setup}\label{environment-and-setup}

As stated in the project's \protect\hyperlink{core-principles}{core
principles}, \emph{Open Development} is vital for the project to gain
trust. Interestingly, this has a significant impact on how a
\emph{PDaaS} might be deployed or installed. All its components can just
be grabbed and used as it suits everyone's needs; while respecting their
licenses. Furthermore, enforcing \emph{portability}
(\protect\hyperlink{sa02}{S.A.02}) leads to a more simplified and
independent development process that can be organized in a way so that
the primary division into components can be leveraged.

The range of environment systems for \emph{server} platforms is highly
diverse but the main shares
{[}\protect\hyperlink{ref-web_2017_wikipedia_os-market-share}{144}{]}
belong to either the UNIX or LINUX family, even though almost every
platform is POSIX-compliant.\footnote{Portable Operating System
  Interface; a collection of standards released by the IEEE Computer
  Society to preserve compatibility between operating systems} When it
comes to \emph{mobile} platforms the market is fare less divers. Native
applications are either developed in \emph{Java} (for Google's Android)
or in Swift (for Apple's iOS). Whereas the environment systems has
nearly no relevance for the \emph{desktop}, other then the screen size
and maybe which browser and version the environment system runs. But
that's a situation the user can change if necessary.

As a result, being able to use certain components on a \emph{server}
platform depends on what \emph{server} environment is available. And
vice versa, in order to decide what implementation of a component is
suitable, it's crucial to know in which environment that component has
to run in. Either way, not to forget all the dependencies a component
itself might have. Such constraints can be avoided by abstracting the
runtime of those components and either embed every required software
dependency or provide them in separate runtimes, if that's possible.
Depending on the used technologies, this concept is commonly known as
\emph{virtualization} or \emph{containerization}. It isolates software
by putting them into a so called \emph{container}. But since those
container-wrapped components still have to interact with each other,
they need to be supervised or at least managed. This is done by an
orchestration software, which not only allocates system resources but is
also capable of emulating a whole network infrastructure (e.g.~DNS,
TCP/IP, routing). Thereby, it is utilized to determine how certain
container (and its containing component) are allowed to communicate and
what resource are accessible from the inside (e.g.~filesystem). This
complete abstraction to the surrounding environment it effectively means
it's the only dependency the \emph{PDaaS} would have, regardless of how
its components are implemented. They just have to be
\emph{`containerizable'} - satisfy the
\emph{\protect\hyperlink{def--container}{container image specification}}
{[}\protect\hyperlink{ref-web_oci-spec_image}{116}{]}. This concept can
also be utilized for the
\emph{\protect\hyperlink{supervised-data-access}{supervised code
execution}} (\protect\hyperlink{sa01}{S.A.01}) mentioned before without
any restraints.

Migrating from a server-located \emph{personal data storage} to a
\emph{mobile} based version introduces another challenge. The subsequent
approach is a first and more general solution to that problem.

\emph{NOTICE: it is assumed that a running instance of a }PDaaS* is in
place, the \emph{operator} owns a modern mobile device and on this
device a \emph{PDaaS} mobile application is installed.*

\begin{enumerate}
\def\labelenumi{\arabic{enumi}.}
\item
  After starting the app, the operator needs to establish a connection
  between server and mobile application. Therefor the operator either
  has to scan a QR-Code with the help of that app. The QR-Code is
  presented to the operator within the management tool of the
  \emph{PDaaS} running in a browser. Or the operator inserts her
  credentials into a form presented by the mobile application.
\item
  After the connection has established, the operator can trigger a
  progress that duplicated all her personal data to the device that has
  just been associated with the \emph{PDaaS}.
\item
  At this point, one of two ways can be proceeded with, depending on
  whether a complete write log for the \emph{personal data}
  (\protect\hyperlink{data}{see discussion about backup strategies} does
  exist or not.

  \begin{enumerate}
  \def\labelenumii{\alph{enumii})}
  \tightlist
  \item
    If \emph{{[}LOG-EXISTS{]}}, query by query the whole log is obtained
    from the existing storage and is then again executed in
    chronologically order by the query language abstraction, starting
    with the oldest. The only difference here is that the target
    storage, on which that query is actually performed on, is located on
    that newly introduced platform
  \item
    If \emph{{[}LOG-NOT-EXISTS{]}}, the situation is more complicated,
    if the database systems are not based on exactly the same
    technology. Hence, additional migration software is required. If
    both database systems provide import and export mechanisms that
    support at least one interoperable data format, the migration
    software can leverage this features simply by exporting all the data
    and saving it to the filesystem. The software then transfers the
    dump to the target environment and triggers the import process. When
    this si not the case, the software not only needs to be aware of
    both database systems and their native query language, it also has
    to have a comprehensive understanding of how their data structuring
    concepts work, in order to reliably transform one into the another.
    So, to be more specific, at first the software has to analyse the
    structure of the source database. Based on this result it might need
    to perform some configuration on the target database, before
    actually obtaining the data from there. Received data then need to
    be transformed into queries that are supported by the target
    database system. Those transformed queries are transferred to the
    target environment, where those incoming queries get executed until
    all data is migrated.
  \end{enumerate}
\item
  After the duplication process has finished, the operator can decide
  which PDS the \emph{PDaaS} should use to serve \emph{access requests}
  and what should happen with the other storage(s).
\end{enumerate}

To conclude, a migration process like moving \emph{personal data} from
one platform instance to another can be much more simplified and robust,
if a complete query log would exist. It is also worth mentioning, that
the migration process described above is not restricted to exactly this
source or migration direction. As long as target and source are either a
\emph{server} or a \emph{mobile} platform, any variant is imaginable.

\emph{\textbf{Conclusions:}} Installing a \emph{PDaaS} should be
straightforward with the least possible effort being spend for
preparations. Package manager of all popular operating systems should
offer (semi-)automated installations. Additionally, components
themselves and the project as whole have to provide detailed
documentations for various ways of how those parts or the entire system
need to be installed. Alternatively, \emph{data subjects} might be
willing to entrust external third parties with hosting a \emph{PDaaS}
instance for them. In that case the distributed approach involving a
\emph{mobile} platform might come in handy, so that the actual data is
not stored somewhere beyond their reach. The \emph{PDaaS} as an open
source development encourages anybody who is interested or even wants to
contribute to checkout the source code of the various implementations,
get it to run and play around with it. But therefor at least the
components of the \emph{server} platform are required to have documented
on what other software they depend on, so that the target environment
can be prepared properly. Aside from hardware, on which the \emph{PDaaS}
needs to run, the only other requirement is owning an internet domain
that is registered on a public DNS\footnote{Domain Name System;
  decentralized open directory that associates readable (domain) names
  with IP addresses} server and has no subdomains configured yet.

If a component needs to get segregated from its host environment,
\emph{containerization} is the recommended technique, since it causes
less overhead compared to \emph{virtualization} and is generally a
lightweight approach. Though, additional abstraction might also
introduce new problems instead of solving them.

\section{User Interfaces}\label{user-interfaces}

Designing graphical user interfaces is beyond the scope of this work and
the \emph{PDaaS} specification as well. Nevertheless this section shell
be understood as a collection of proposed ideas addressing the questions
of what types of user interfaces the \emph{PDaaS} should provide and
which features they might need to support.

The most notable characteristic used to distinguish user interfaces from
each other are those interfaces that are visible and the ones that
arn't. For example a \emph{graphical user interface (GUI)}, composed of
visually separated areas with a certain semantic and assembled with
meaningful objects on which the user can physically act, for example by
touching them. The interface then reacts on those actions by changing
its appearance. In this way the user can recognize and comprehend her
actions. Whereas non-graphical user interfaces don't provide the user
with objects or surfaces to interact with. Instead, the primarily used
medium is text, regardless if it's human-readable or not. But
\emph{command line interfaces (CLI)}, available mainly in command line
environments or shells, still provide a certain level of interactivity.
A running program can pause in order to prompt the user with an input
request. If an input is made and submitted the program then proceeds.
The group of interfaces whose interactions can be fully automated, are
for example \emph{application programming interfaces (API)}. Depending
on the transport technologies, it's no unusual that \emph{API}
interactions are consisting of just one action causing one reaction.
Non-graphical interfaces enabling interactions on a lower level. Even
though they provide more functionality and can be more time efficient,
they are more rudemental and often less secure. While \emph{GUIs} are
normally meant for end users to interact with applications on a more
sophisticated level, \emph{CLIs} are used during development, for
automation, or for server environment administration; probably remotely,
because they are typically headless. Whereas \emph{APIs}, documented by
its provider, enable software developers to program automated requests
against those interfaces.

Table \ref{tbl:ui-features} provides a list of features and associates
the different types of user interfaces mentioned before, which indicates
if they should be supported by a certain type. It is notable that the
\emph{GUI} provides the \emph{operator} with a powerful tool Hence it
requires reliable protection mechanisms (see
\protect\hyperlink{authentication}{Authentication}). Whereas \emph{API}
capabilities are very limited, because it's the one interface that the
\emph{PDaaS} exposes to third parties.

\begin{longtable}[]{@{}lccc@{}}
\caption{Features that should be supported by the given user interfaces
\label{tbl:ui-features}}\tabularnewline
\toprule
Feature & GUI & CLI & API\tabularnewline
\midrule
\endfirsthead
\toprule
Feature & GUI & CLI & API\tabularnewline
\midrule
\endhead
manage permission profiles (\protect\hyperlink{pviu03}{P.VIU.03}) &
\textbf{X} & - & \textbf{X}\tabularnewline
view access history (\protect\hyperlink{pviu04}{P.VIU.04}) & \textbf{X}
& \textbf{X} & -\tabularnewline
register consumer & \textbf{X} & \textbf{X} & -\tabularnewline
add new front end & \textbf{X} & - & -\tabularnewline
authenticate operator & \textbf{X} & - & -\tabularnewline
migrate personal data & \textbf{X} & \textbf{X} & -\tabularnewline
review permission requests (\protect\hyperlink{pi04}{P.I.04}) &
\textbf{X} & - & -\tabularnewline
create \& maintain templates (\protect\hyperlink{pi05}{P.I.05}) &
\textbf{X} & - & -\tabularnewline
adjust precision of data (\protect\hyperlink{pi06}{P.I.06}) & \textbf{X}
& - & \textbf{X}\tabularnewline
introduce new data structs & \textbf{X} & - & \textbf{X}\tabularnewline
configure \emph{PDaaS} & \textbf{X} & - & -\tabularnewline
import personal data & \textbf{X} & - & \textbf{X}\tabularnewline
read/access personal data & \textbf{X} & \textbf{X} &
\textbf{X}\tabularnewline
manipulate personal data & \textbf{X} & \textbf{X} & -\tabularnewline
run supervised code execution & - & \textbf{X} &
\textbf{X}\tabularnewline
\bottomrule
\end{longtable}

The architectural design includes \emph{desktop} and \emph{mobile}
platforms. While prioritizing a web-based \emph{GUI}, the management
tool for the \emph{operator} also needs to be natively implemented for
common mobile systems (\protect\hyperlink{pviu02}{P.VIU.02}); in this
case Android and iOS. This again enables to provide real-time
notifications (\protect\hyperlink{pi03}{P.I.03},
\protect\hyperlink{pb02}{P.B.02}) on mobile platforms, whereas the same
feature is added to \emph{desktop} platforms by providing
WebSocket-based connections. Since screen sizes can vary - in particular
on \emph{mobile} platforms - the \emph{GUI} is required to be highly
responsive and has to adapt (\protect\hyperlink{pviu01}{P.VIU.01})
various screen sizes. Given the capabilities of the management tool, an
inaccurate or error-prone rendered \emph{GUI} can quickly cause
unintended incidents. Thus the main focus must be to ensure a very
robust and low-latency interface rendering.

Known challenges for the \emph{GUI} design are primarily to develop very
efficient but also fun to use interfaces for reviewing \emph{consumer
registrations} and \emph{permission requests}. Especially the latter can
become hard to solve, because how can graph-based and nested data
structures be displayed in such a way that makes reviewing and also
manipulating an easy task to do - even on a screen of a mobile phone?
One approach could be to utilize the \emph{accordion} pattern
{[}\protect\hyperlink{ref-web_2016_wikipedia_accordion-gui}{145}{]} for
edges and start nesting them in order to represent subsequent data
structure. The interaction then might look like tree-structured
navigation; moving alongside relations just by expanding and folding in
data points.

Since other parts of the system have to provide the mechanisms for
increasing or reducing the precision of data due to privacy protection,
the challenge here is to find the right design concepts for data subject
to facilitate those adjustments. Precision adjustments can be achieved
by either changing the sampling rate in a dataset containing a series of
data points, or by rounding values to a certain extend. Example are
cutting fractional digits of the latitude and longitude values in a
position information, or removing all position information obtained
between every quarter from a full day tracking period. Whether data
subjects can choose from an abstracted precision grading (e.g.
\emph{high}, \emph{mid}, and \emph{low}) or they set specific type or
unit related filter mechanisms, configurable defaults on a system-wide
level should be provided by \emph{GUIs} in any case. Following data
types are supposedly vulnerable to compromise privacy, thus proposed to
support precision adjustments: \emph{Date} (time), any kind of absolute
measurements, sets containing data series, and position information, as
mentioned before.

\emph{\textbf{Conclusions:}} The most important aspect, when interacting
with something or someone, is being provided with a feedback. An action
typically causes - and is therefore \emph{expected} - a reaction. The
result is an interaction, unless no reaction occurred.

The discussion above outlines the relevance of those interactions for
the \emph{PDaaS}. Thus, for users and other software to interact with
the \emph{PDaaS} interfaces is mandatory. Primary characteristics of
those interfaces are complete functionality, security precautions and
restrictions, as well as comprehensive documentations. And visual user
interfaces in addition, needs to provide reliable and adaptive
rendering, a consistent and encouraging interaction design. \emph{GUIs}
need to be provided for all \emph{desktop} and \emph{mobile} platforms,
primarily to provide an efficient user experience for the operator. The
operator is the only role with permissions to access a \emph{GUI}.
Components on the \emph{server} platform should provide \emph{CLIs}, at
least when no other technical option exist to interact with them. Also
accessing the database from command line could be appreciated at some
point. APIs are mostly meant for data consumers to interact with the
\emph{PDaaS}, and perhaps for automated data contribution (based on
\emph{operator} role; e.g.~browser plugin). \emph{Desktop} platforms
might use those \emph{APIs} as well. In any case, \emph{APIs} must be
separated according to the \emph{roles}.

These are all vital characteristics whose details need to be addressed
by the \emph{specification}. Whose implementation details though are not
the concern of this specification, as long as every stated requirement
is being acknowledged.

\chapter{\texorpdfstring{Specification
\emph{(Draft)}}{Specification (Draft)}}\label{specification-draft}

This chapter hold the first draft of what might become a specification.
As for now it has therefore no claim of completeness, continuity or
accuracy. The contents is based on and a result of all previous
discussions and developed solutions.

TODO: should might must n stuff in table (see dark mail spec) or just
reference: https://tools.ietf.org/html/rfc2119

\section{Overview}\label{overview}

\begin{itemize}
\tightlist
\item
  purpose
\item
  architectural overview
\item
  short description of the whole process
\end{itemize}

\section{Components}\label{components}

\subsection{Webserver}\label{webserver}

\subsection{User Interface}\label{user-interface}

\subsection{Storage/Persistence}\label{storagepersistence}

\subsection{Notification
Infrastructure}\label{notification-infrastructure}

\subsection{Data API}\label{data-api}

\section{Data}\label{data-1}

\subsection{Structure \& Types}\label{structure-types}

\begin{itemize}
\tightlist
\item
  henceforth only two things: primitive and struct
\item
  supported date types
\end{itemize}

\subsection{Read}\label{read}

\begin{itemize}
\tightlist
\item
  permission profiles

  \begin{itemize}
  \tightlist
  \item
    type
  \item
    how often
  \item
    what data
  \end{itemize}
\end{itemize}

\subsection{Write}\label{write}

(!) every data or configuration change has to be reversible

precision of data: demanding lower precision than the \emph{data
subject} has approved is always possible. The other ways around not.

\section{Protocols}\label{protocols}

\subparagraph{Consumer registration}\label{consumer-registration}

\begin{enumerate}
\def\labelenumi{\arabic{enumi})}
\setcounter{enumi}{-1}
\item
  {[}OPTIONAL{]} \emph{data subject} provides URI to \emph{data
  consumers}
\item
  \emph{data consumers} create \emph{permission request} that includes

  \begin{itemize}
  \tightlist
  \item
    X.509 based CSR\footnote{Certificate signing request}
  \item
    callback URI via HTTPS as feedback channel
  \item
    {[}OPTIONAL{]} information about what data points wanted to be
    accessed
  \end{itemize}
\item
  depending on 0), \emph{data consumer} provides \emph{operator} with
  priorly created \emph{permission request} either as QR-Code or via
  HTTPS by given URI
\item
  \emph{operator} reviews request and decides to either refuse or grant
  assess; the latter results in:

  \begin{enumerate}
  \def\labelenumii{\alph{enumii})}
  \tightlist
  \item
    creating new \emph{endpoint}

    \begin{itemize}
    \tightlist
    \item
      create new unique subdomain and a related asymmetric key pair
      signed by the system's root CA (self-signed)
    \item
      issue \emph{consumer} certificate based on it's CRS and sign it
      with the key pair related to this \emph{endpoint}
    \end{itemize}
  \item
    if information is provided, creating new \emph{permission profile}
    by either applying existing draft/template or configuring
    \emph{permission type} (incl. expiration date if required) and
    permitted data endpoints; associate to specific \emph{endpoint}
  \end{enumerate}
\item
  \emph{data consumers} gets informed about the decision via callback
  channel

  \begin{itemize}
  \tightlist
  \item
    on grant, response includes

    \begin{itemize}
    \tightlist
    \item
      \emph{consumer's} certified certificate
    \item
      certificate that's associated with the created endpoint
    \item
      information on what data points are allowed to be accessed;
    \end{itemize}
  \item
    on refusal: error code/message
  \end{itemize}
\item
  \emph{data consumer} handles the response appropriately

  \begin{itemize}
  \tightlist
  \item
    {[}OPTIONAL{]} pin the provided \emph{PDaaS} certificate
  \end{itemize}
\end{enumerate}

\subparagraph{Data Access}\label{data-access}

\begin{enumerate}
\def\labelenumi{\arabic{enumi})}
\setcounter{enumi}{-1}
\item
  after successfully authenticated, \emph{consumer} sends \emph{access
  request}
\item
  request contains at least the \emph{data query}; based on that query
  and the \emph{permission profiles}, access is tried to get verified

  \begin{enumerate}
  \def\labelenumii{\alph{enumii})}
  \tightlist
  \item
    on success, response gets computed
  \item
    on failure, error code/message is responded; process aborts

    \begin{itemize}
    \tightlist
    \item
      if the error was raised because no appropriate \emph{permission
      profile} was found, then the \emph{consumer} first needs to
      request permission for the \emph{data points} that were part of
      the query
    \end{itemize}
  \end{enumerate}
\item
  {[}OPTIONAL{]} depending on whether the \texttt{keepalive} flag was
  set \texttt{true}, the connection of this requests lasts until
  response computation has finished or timeout has reached, otherwise
  the response contains a URI unique to this current request including
  an estimation when response will be available under that URI;
  connection can still timeout, which is defined by the system
\item
  depending on the type of that \emph{access request},

  \begin{enumerate}
  \def\labelenumii{(\Alph{enumii})}
  \tightlist
  \item
    the data get queried and the result is added to the response
  \item
    based in further information provided by the request, the
    environment for the \emph{supervised code execution} is getting
    provisioned, the program from the \emph{consumer} will be ran
    against various tests

    \begin{enumerate}
    \def\labelenumiii{\alph{enumiii})}
    \tightlist
    \item
      in fail, error code/message get added to the response
    \item
      on pass, computed result gets added to the response
    \end{enumerate}
  \end{enumerate}
\item
  response is finalized and gets returned back to the \emph{consumer},
  either as a response to this request or provided under the unique URI
  as of 2)
\end{enumerate}

\subparagraph{Permission Validation}\label{permission-validation}

TODO: detailed description of the algorithm that checks \emph{permission
profiles} according to an \emph{access request}; including all different
possible cases (multiple profiles for one consumer etc)

\subparagraph{Add or Change Personal
Data}\label{add-or-change-personal-data}

\subsection{Data Management}\label{data-management}

\begin{itemize}
\tightlist
\item
  one third party access (consumer) relates to one access
  \emph{endpoint}, that also authenticates that third party by TLS based
  \emph{two-way auth}
\item
  zero or more \emph{permission profiles} are associated to one
  \emph{endpoint}
\end{itemize}

\section{APIs}\label{apis}

\textbf{Registration Request}

\begin{itemize}
\tightlist
\item
  contains certificate signing request
\item
  {[}OPTIONAL{]} contains \emph{permission request}
\end{itemize}

\begin{Shaded}
\begin{Highlighting}[]
\FunctionTok{\{}
    \DataTypeTok{"callbackUri"}\FunctionTok{:} \StringTok{"TODO"}\FunctionTok{,}
    \DataTypeTok{"csr"}\FunctionTok{:} \StringTok{"TODO"}\FunctionTok{,}
    \DataTypeTok{"dataPoints"}\FunctionTok{:} \OtherTok{[}
        \StringTok{"profile.lastname"}
    \OtherTok{]}
\FunctionTok{\}}
\end{Highlighting}
\end{Shaded}

\textbf{Permission Request}

\begin{itemize}
\tightlist
\item
  creates new \emph{permission profile}
\item
  \texttt{https://example-consumer.pdaas.tld/pr}
\end{itemize}

\begin{Shaded}
\begin{Highlighting}[]
\FunctionTok{\{}
    \DataTypeTok{"callbackUri"}\FunctionTok{:} \StringTok{""}\FunctionTok{,}
    \DataTypeTok{"dataPoints"}\FunctionTok{:} \OtherTok{[}
        \StringTok{"profile.lastname"}
    \OtherTok{]}
\FunctionTok{\}}
\end{Highlighting}
\end{Shaded}

\textbf{Access Request}

\begin{itemize}
\tightlist
\item
  obtains actual data
\item
  if \texttt{keepalive} is set \texttt{true}, the connections lasts
  until response computation has finished, otherwise the response
  contains a URI unique to this current request including an estimation
  when response will be available under that URI; connection can still
  timeout, which is defined by the system
\item
  \texttt{https://example-consumer.pdaas.tld/ar}
\end{itemize}

\begin{Shaded}
\begin{Highlighting}[]
\FunctionTok{\{}
    \DataTypeTok{"query"}\FunctionTok{:} \StringTok{"TODO"}
\FunctionTok{\}}
\end{Highlighting}
\end{Shaded}

\emph{Requirements:}

\begin{itemize}
\tightlist
\item
  query has to match exactly one corresponding \emph{permission profile}
\end{itemize}

TODO: basic structure of a \emph{permission profile}

How do the APIs involved with the protocols look like?

\section{Security}\label{security}

\begin{itemize}
\item
  the downside of having not just parts of the personal data in
  different places (which is currently the common way to store), is in
  case of security breach, it would increase the possible damage by an
  exponential rate Thereby all data is exposed at once, instead of not
  just the parts which a single service has stored
\item
  does it matter from what origin the data request was made? how to
  check that? is the requester's server domain in the http header?
  eventually there is no way to check that, so me might need to go with
  request logging and trying to detect abnormal behaviour/occurrence
  with a learning artificial intelligence
\item
  is the consumer able to call the access request URI repeatedly and any
  time? (meaning will this be stateless or stateful?)
\item
  initial consumer registration would be done on a common and valid
  https:443 CA-certified connection. after transferring their cert to
  them as a response, all subsequent calls need to go to their own
  endpoint, defined as subdomains like
  \texttt{consumer-name.owners-notification-server.tld}
\end{itemize}

\subsection{Environment}\label{environment}

\subsection{Transport}\label{transport}

\begin{itemize}
\tightlist
\item
  communication between internal components \emph{must} be done in https
  only, but which ciphers? eventually even http/2?
\end{itemize}

\subsection{Storage}\label{storage}

\begin{itemize}
\tightlist
\item
  documents based DB instead of Relational DBS, because of
  structure/model flexibility
\item
  graphql because of it's nature to abstract a storage engine, which
  comes in handy when the actual storage gets relocated (e.g.~from a
  server to a mobile device)
\end{itemize}

\subsection{Authentication}\label{authentication-1}

\begin{itemize}
\item
  how should consumer authenticate?
\item
  list min. configs for tls! e.g.~use ciphers supporting PFS
\end{itemize}

\section{Recommendations}\label{recommendations}

\subsection{Software Dependencies}\label{software-dependencies}

\subsection{Host Environment(s)}\label{host-environments}

\chapter{Conclusion}\label{conclusion}

\section{\texorpdfstring{Ethical \& Social Impact (TODO: or
``Relevance'')}{Ethical \& Social Impact (TODO: or Relevance)}}\label{ethical-social-impact-todo-or-relevance}

\begin{itemize}
\item
  Regarding involving an official party to verify data reliability: The
  actual question would be, is the \emph{data subject} certain, that she
  really wants to hand over those capabilities to official authorities?
  Depending on which \emph{data consumers}, what task their are
  entrusted with and what motivation the \emph{data subject} has has in
  mind to do so, the \emph{PDaaS} might become a powerful \emph{`digital
  reflection'} and starts to get seen as a real and reliable
  representation of herself. Then the decisions made by \emph{data
  consumers} might have a big impact for the \emph{data subject's} life.
  For example a housing loan won't be granted or a medical treatment has
  been refused.
\item
  give back the data subject to control the level of privacy she is
  willing to share
\item
  (formally placed in 230) At the end it all comes down to understanding
  the human being and why she behaves as she does. The challenge is not
  only to compute certain motives but rather concluding to the right
  ones. When analyzing computed results with the corresponding data
  models and trying to conclude, it is important to keep in mind, that
  correlation is by far no proof of causation.
\end{itemize}

\section{Business Models \&
Monetisation}\label{business-models-monetisation}

\begin{itemize}
\tightlist
\item
  possible resulting direct or indirect business models
\item
  data subject might want to sell her data, only under her conditions.
  therefor some kind of infrastructure and process is required (such as
  payment transfer, data anonymization, market place to offer data)
\end{itemize}

\section{Target group perspectives}\label{target-group-perspectives}

\begin{itemize}
\tightlist
\item
  User: would I use this stuff? The underpinning technical details and
  how it works is not my concern and non of my interests. I want this
  stuff work and being reliable. it its simple to use. and maybe even
  easy to setup (server n stuff), then the hell, I would!
\item
  Dev:

  \begin{itemize}
  \tightlist
  \item
    spec implementer
  \item
    integrater in consumer:
  \end{itemize}
\item
  Consumer:

  \begin{itemize}
  \tightlist
  \item
    what can she do with it: adjsut precision of datasets and values to
    increase privacy
  \item
    control and get an overview on where her data might flow (and for
    what purpose)
  \end{itemize}
\end{itemize}

TODO: make a reference or involve the research mentioned at the
beginning

\section{Challenges}\label{challenges}

\begin{itemize}
\item
  adoption rate of such technology
\item
  data reliability from the perspective of a \emph{data consumer} Since
  it is almost impossible to ensure complete reliability of all the data
  a \emph{PDaaS} has stored or might me offering, and because it is
  operated by exactly that individual, and that individual only, all
  data in question is relates to and is thereby owned by her, it, of
  cause, makes it not easy for \emph{data consumers} to trust
  \emph{PDaaS}s as resources for their business processes, but I am
  certain, that the demand for all different kinds of data exceeds the
  partial uncertainty of their reliability.
\item
  personal data leaking Preventing personal data from being leaked to
  the outside, is, especially because of the system's purpose, extremely
  hard to prevent, if not possible at all. Just by querying data from
  the storage or by physically transferring them from one location to
  another, it's already copied. It's the very nature of digital
  information technology/systems. So this cannot be defeated. It only
  can be impeded. Interestingly though, is the same approach the media
  industry for centuries is trying to make copyright infringements more
  difficult.
\item
  scenario where the mobile device, or in general the data storage get
  lost. first of all, not much of a problem, because either device
  backup or since the liberal relation, the system would continue to
  function, but limited, until a data storage gets part of the system
  again (TODO: touched on in the data section at the end)
\item
  during concept development, it appears to become necessary to define
  another role, for \emph{data contributors} (plugins/clients that are
  authorized by the \emph{operator} but only allowed to push data to the
  \emph{PDaaS}).
\item
  when \emph{structs} currently in use get changes all data have to
  migrate accordingly and fully automated
\end{itemize}

\section{Solutions}\label{solutions}

\begin{itemize}
\item
  even though \emph{OAuth} don't find it's way into this project,
  working through the standard inspired here and there a solution, for
  example using a URI as a feedback channel or TODO.
\item
  refer to the scenarios at the beginning by saying that with the
  \emph{PDaaS} one is able to implement all of them
\end{itemize}

\section{Attack Scenarios}\label{attack-scenarios}

\begin{itemize}
\item
  single point of failure (data-wise),

  \begin{itemize}
  \tightlist
  \item
    but considering what data users already put into their social
    networks (or: thE social network: fb), they/it has already become a
    de facto data silo and is thus a single point of failure. If that
    service breaks or get down, the data from all users might be lost or
    worse (stolen). The aspect of data decentralisation achieved by
    individual data stores can be valued as positive.
  \end{itemize}
\item
  what about token stealing when using jwt?
\item
  future work: add/activate an intrusion detection system
\end{itemize}

\section{Future Work}\label{future-work}

\begin{itemize}
\item
  maybe enable the tool to play the role of an own OpenID provider?
\item
  going one step further and train machine (predictor) by our self with
  our own data
  (https://www.technologyreview.com/s/514356/stephen-wolfram-on-personal-analytics/)
\item
  finalize first draft of the spec with all core aspect included and
  outlined
\item
  developing based on that a first prototype to find flaws in the spec.
  iterate/repeat
\item
  release 1.0 (spec and example implementation)
\item
  touch on parts that were left blank
\item
  first supporting platforms
\item
  full encryption of the \emph{data storage}
\end{itemize}

\section{Summary}\label{summary}

\begin{itemize}
\tightlist
\item
  main focus
\item
  unique features
\item
  technology stack \& standards
\item
  resources
\item
  the tool might be not a bulletproof vest, but
\end{itemize}

The work will be continued.

\chapter*{Source Code}\label{source-code}
\addcontentsline{toc}{chapter}{Source Code}

\pagenumbering{Roman} \setcounter{page}{6} \pagestyle{plain}

\textbf{\protect\hypertarget{code-01_sparql-query}{}{Code 01: Example
query in SPARQL}:}

\begin{Shaded}
\begin{Highlighting}[numbers=left,,]
\NormalTok{# }\KeywordTok{query} \DecValTok{1}\NormalTok{: obtain }\KeywordTok{the} \FunctionTok{first} \KeywordTok{and} \FunctionTok{last} \NormalTok{name fof }\KeywordTok{data} \NormalTok{subject}
\NormalTok{PREFIX person: <http://pdaas.tld/schemas/person>}

\KeywordTok{SELECT} \NormalTok{$firstname $lastname}
\KeywordTok{FROM} \NormalTok{<https://unique-consumer-endpoint.pdaas.tld/sparql/profile>}
\KeywordTok{WHERE} \NormalTok{\{}
    \NormalTok{$person person}\CharTok{:firstname} \NormalTok{$firstname .}
    \NormalTok{$person person}\CharTok{:lastname} \NormalTok{$lastname .}
\NormalTok{\}}


\NormalTok{# }\KeywordTok{query} \DecValTok{2}\NormalTok{: obtain }\KeywordTok{all} \NormalTok{bank accounts that are available }\KeywordTok{for} 
\NormalTok{# }\KeywordTok{online} \NormalTok{payment}
\NormalTok{PREFIX bank-account: <http://pdaas.tld/schemas/bank-account>}

\KeywordTok{SELECT} \NormalTok{$accountId $bankName $paymentMethod}
\KeywordTok{FROM} \NormalTok{<https://unique-consumer-endpoint.pdaas.tld/sparql/finance>}
\KeywordTok{WHERE} \NormalTok{\{}
    \NormalTok{$bank-account bank-account}\CharTok{:payment}\NormalTok{-method }\OtherTok{"online-service"} \NormalTok{.}
    \NormalTok{$bank-account bank-account}\CharTok{:payment}\NormalTok{-method $paymentMethod .}
    \NormalTok{$bank-account bank-account}\CharTok{:account}\NormalTok{-id $accountId . }
    \NormalTok{$bank-account bank-account}\CharTok{:bank}\NormalTok{-name $bankName .}
\NormalTok{\}}
\end{Highlighting}
\end{Shaded}

\newpage

\textbf{\protect\hypertarget{code-02_sparql-query-results}{}{Code 02:
Results of Code 01 in JSON}:}

\begin{Shaded}
\begin{Highlighting}[numbers=left,,]
\ErrorTok{//} \ErrorTok{result} \ErrorTok{1:}
\FunctionTok{\{}
    \DataTypeTok{"head"}\FunctionTok{:} \FunctionTok{\{}
        \DataTypeTok{"vars"}\FunctionTok{:} \OtherTok{[}
            \StringTok{"firstname"}\OtherTok{,}
            \StringTok{"lastname"}
        \OtherTok{]}
    \FunctionTok{\},}
    \DataTypeTok{"results"}\FunctionTok{:} \FunctionTok{\{}
        \DataTypeTok{"bindings"}\FunctionTok{:} \OtherTok{[}
            \FunctionTok{\{}
                \DataTypeTok{"firstname"}\FunctionTok{:} \FunctionTok{\{}
                    \DataTypeTok{"type"}\FunctionTok{:} \StringTok{"literal"}\FunctionTok{,}
                    \DataTypeTok{"value"}\FunctionTok{:} \StringTok{"Doe"}
                \FunctionTok{\},}
                \DataTypeTok{"lastname"}\FunctionTok{:} \FunctionTok{\{}
                    \DataTypeTok{"type"}\FunctionTok{:} \StringTok{"literal"}\FunctionTok{,}
                    \DataTypeTok{"value"}\FunctionTok{:} \StringTok{"Jane"}
                \FunctionTok{\}}
            \FunctionTok{\}}
        \OtherTok{]}
    \FunctionTok{\}}
\FunctionTok{\}}

\ErrorTok{//} \ErrorTok{result} \ErrorTok{2:}
\FunctionTok{\{}
    \DataTypeTok{"head"}\FunctionTok{:} \FunctionTok{\{}
        \DataTypeTok{"vars"}\FunctionTok{:} \OtherTok{[}
            \StringTok{"accountId"}\OtherTok{,}
            \StringTok{"bankName"}\OtherTok{,}
            \StringTok{"paymentMethod"}
        \OtherTok{]}
    \FunctionTok{\},}
    \DataTypeTok{"results"}\FunctionTok{:} \FunctionTok{\{}
        \DataTypeTok{"bindings"}\FunctionTok{:} \OtherTok{[}
            \FunctionTok{\{}
                \DataTypeTok{"accountId"}\FunctionTok{:} \FunctionTok{\{}
                    \DataTypeTok{"type"}\FunctionTok{:} \StringTok{"integer"}\FunctionTok{,}
                    \DataTypeTok{"value"}\FunctionTok{:} \DecValTok{0905553715}
                \FunctionTok{\},}
                \DataTypeTok{"bankName"}\FunctionTok{:} \FunctionTok{\{}
                    \DataTypeTok{"type"}\FunctionTok{:} \StringTok{"literal"}\FunctionTok{,}
                    \DataTypeTok{"value"}\FunctionTok{:} \StringTok{"A. W. Fritter Institute"}
                \FunctionTok{\},}
                \DataTypeTok{"paymentMethod"}\FunctionTok{:} \FunctionTok{\{}
                    \DataTypeTok{"type"}\FunctionTok{:} \StringTok{"literal"}\FunctionTok{,}
                    \DataTypeTok{"value"}\FunctionTok{:} \StringTok{"online-service"}
                \FunctionTok{\}}
            \FunctionTok{\}}
        \OtherTok{]}
    \FunctionTok{\}}
\FunctionTok{\}}
\end{Highlighting}
\end{Shaded}

\newpage

\textbf{\protect\hypertarget{code-03_graphql-query}{}{Code 03: Example
query in GraphQL}:}

\begin{Shaded}
\begin{Highlighting}[numbers=left,,]
\NormalTok{# URL}\OperatorTok{:} \NormalTok{https}\OperatorTok{:}\CommentTok{//unique-consumer-endpoint.pdaas.tld/graphql}

\NormalTok{query }\OperatorTok{\{}
    \NormalTok{profile }\OperatorTok{\{}
        \NormalTok{firstname}
        \NormalTok{lastname}
    \OperatorTok{\}}
    \AttributeTok{bankAccounts}\NormalTok{(}\DataTypeTok{paymentMethod}\OperatorTok{:} \StringTok{'online-service'}\NormalTok{) }\OperatorTok{\{}
        \NormalTok{accountId}
        \NormalTok{bankName}
        \NormalTok{paymentMethod}
    \OperatorTok{\}}
\OperatorTok{\}}
\end{Highlighting}
\end{Shaded}

~\\
\textbf{\protect\hypertarget{code-04_graphql-query-result}{}{Code 04:
Result of Code 03 in JSON}:}

\begin{Shaded}
\begin{Highlighting}[numbers=left,,]
\FunctionTok{\{}
    \DataTypeTok{"profile"}\FunctionTok{:} \FunctionTok{\{}
        \DataTypeTok{"firstname"}\FunctionTok{:} \StringTok{"Jane"}\FunctionTok{,} 
        \DataTypeTok{"lastname"}\FunctionTok{:} \StringTok{"Doe"}
    \FunctionTok{\},}
    \DataTypeTok{"bankAccounts"}\FunctionTok{:} \OtherTok{[}
        \FunctionTok{\{}
            \DataTypeTok{"accountId"}\FunctionTok{:} \DecValTok{0905553715}\FunctionTok{,}
            \DataTypeTok{"bankName"}\FunctionTok{:} \StringTok{"A. W. Fritter Institute"}\FunctionTok{,}
            \DataTypeTok{"paymentMethod"}\FunctionTok{:} \StringTok{"online-service"}
        \FunctionTok{\}}
    \OtherTok{]}
\FunctionTok{\}}
\end{Highlighting}
\end{Shaded}

\newpage

\emph{NOTICE: schema notation is based on the rules underpinning the
schema definition provided by the SimpleSchema project
{[}\protect\hyperlink{ref-web_2017_repo_node-simple-schema}{146}{]}.}

\textbf{\protect\hypertarget{code-05_struct_profile}{}{Code 05: Struct -
Profile (example)}}

\begin{Shaded}
\begin{Highlighting}[numbers=left,,]
\OperatorTok{\{}
    \DataTypeTok{firstname}\OperatorTok{:} \NormalTok{String}\OperatorTok{,}
    \DataTypeTok{lastname}\OperatorTok{:} \NormalTok{String}\OperatorTok{,}
    \DataTypeTok{pseudonym}\OperatorTok{:} \NormalTok{String}\OperatorTok{,}
    \DataTypeTok{birth}\OperatorTok{:} \NormalTok{Date}\OperatorTok{,}
    \DataTypeTok{gender}\OperatorTok{:} \NormalTok{String}\OperatorTok{,}
    \DataTypeTok{religion}\OperatorTok{:} \NormalTok{String}\OperatorTok{,}
    \DataTypeTok{motherTongue}\OperatorTok{:} \NormalTok{Language}
    \DataTypeTok{photo}\OperatorTok{:} \NormalTok{File}\OperatorTok{,}
    \DataTypeTok{residence}\OperatorTok{:} \NormalTok{Address}\OperatorTok{,}
    \DataTypeTok{employer}\OperatorTok{:} \NormalTok{Organisation}
\OperatorTok{\}}
\end{Highlighting}
\end{Shaded}

\textbf{\protect\hypertarget{code-06_struct_contact}{}{Code 06: Struct -
Contact (example)}}

\begin{Shaded}
\begin{Highlighting}[numbers=left,,]
\OperatorTok{\{}
    \DataTypeTok{label}\OperatorTok{:} \NormalTok{String}\OperatorTok{,}
    \DataTypeTok{type}\OperatorTok{:} \AttributeTok{String}\NormalTok{(}\StringTok{'phone'}\OperatorTok{|}\StringTok{'email'}\OperatorTok{|}\StringTok{'url'}\OperatorTok{|}\StringTok{'name-of-social-network'}\NormalTok{)}\OperatorTok{,}
    \DataTypeTok{prio}\OperatorTok{:} \AttributeTok{Integer}\NormalTok{(}\DecValTok{0-2}\NormalTok{)}\OperatorTok{,}
    \DataTypeTok{uid}\OperatorTok{:} \NormalTok{String}
\OperatorTok{\}}
\end{Highlighting}
\end{Shaded}

\textbf{\protect\hyperlink{code-07_struct_position}{Code 07: Struct -
Position (example)}}

\begin{Shaded}
\begin{Highlighting}[numbers=left,,]
\OperatorTok{\{}
    \DataTypeTok{lat}\OperatorTok{:} \NormalTok{Float}\OperatorTok{,}
    \DataTypeTok{lon}\OperatorTok{:} \NormalTok{Float}\OperatorTok{,}
    \DataTypeTok{radius}\OperatorTok{:} \OperatorTok{\{}
        \DataTypeTok{value}\OperatorTok{:} \NormalTok{Float}\OperatorTok{,}
        \DataTypeTok{unit}\OperatorTok{:} \NormalTok{Distance}
    \OperatorTok{\},}
    \DataTypeTok{description}\OperatorTok{:} \NormalTok{String}
    \DataTypeTok{ts}\OperatorTok{:} \NormalTok{Date}
\OperatorTok{\}}
\end{Highlighting}
\end{Shaded}

\chapter*{References}\label{references}
\addcontentsline{toc}{chapter}{References}

\hypertarget{refs}{}
\hypertarget{ref-web_2016_privacy-international-about-big-data}{}
{[}1{]} \emph{Big data privacy international}. URL
\url{https://www.privacyinternational.org/node/8}. - retrieved
2016-11-15

\hypertarget{ref-paper_2008_discrimination-aware-data-mining}{}
{[}2{]} \textsc{Pedreshi, Dino}\,; \textsc{Ruggieri, Salvatore}\,;
\textsc{Turini, Franco}: Discrimination-aware data mining. In:
\emph{Proceedings of the 14th ACM SIGKDD international conference on
Knowledge discovery and data mining}~: ACM, 2008, pp.~560--568

\hypertarget{ref-book_2015_ethical-it-innovation_ethical-uses-of-information-and-knowledge}{}
{[}3{]} \textsc{Spiekermann, Sarah}: \emph{Ethical IT Innovation: A
Value-Based System Design Approach}~: CRC Press; Taylor \& Francis
Group, LLC, 2015 --~scale ---~ISBN~978-1-4822-2635-5

\hypertarget{ref-paper_1996_bias-in-computer-systems}{}
{[}4{]} \textsc{Friedman, Batya}\,; \textsc{Nissenbaum, Helen}: Bias in
computer systems. In: \emph{ACM Transactions on Information Systems
(TOIS)} vol. 14 (1996), Nr.~3, pp.~330--347

\hypertarget{ref-wikipedia_2016_cognitive-bias}{}
{[}5{]} \emph{Cognitive bias}. URL
\url{https://en.wikipedia.org/w/index.php?title=Cognitive_bias\&oldid=742803386}.
- retrieved 2016-11-08. ---~Wikipedia. ---~Page Version ID: 742803386

\hypertarget{ref-web_2016_big-data-is-people}{}
{[}6{]} \textsc{Lemov, Rebecca}: \emph{Why big data is actually small,
personal and very human. Aeon essays}. URL
\url{https://aeon.co/essays/why-big-data-is-actually-small-personal-and-very-human}.
- retrieved 2016-11-17

\hypertarget{ref-video_2015_big-data-and-deep-learning_discrimination}{}
{[}7{]} \textsc{Dewes, Andreas}: \emph{C3TV - Say hi to your new boss:
How algorithms might soon control our lives.} URL
\url{https://media.ccc.de/v/32c3-7482-say_hi_to_your_new_boss_how_algorithms_might_soon_control_our_lives\#video\&t=1538}.
- retrieved 2016-11-03

\hypertarget{ref-study_2004_architecture-for-privacy-sensitive-ubiquitous-computing}{}
{[}8{]} \textsc{Hong, Jason I.}\,; \textsc{Landay, James A.}: An
architecture for privacy-sensitive ubiquitous computing. In:
\emph{Proceedings of the 2nd international conference on mobile systems,
applications, and services}~: ACM, 2004, pp.~177--189

\hypertarget{ref-web_2010_projectvrm_about}{}
{[}9{]} \emph{ProjectVRM - about. ProjectVRM}. URL
\url{https://blogs.harvard.edu/vrm/about/}. - retrieved 2016-11-09

\hypertarget{ref-paper_2013_the-personal-data-store-approach-to-personal-data-security_2013}{}
{[}10{]} \textsc{Tom Kirkham}\,; \textsc{Sandra Winfield}\,;
\textsc{Serge Ravet}\,; \textsc{Kellomaki, Sampo}: The personal data
store approach to personal data security. In: \emph{IEEE Security \&
Privacy} vol. 11. Los Alamitos, CA, USA, IEEE Computer Society (2013),
Nr.~5, pp.~12--19

\hypertarget{ref-whitepaper_2014_mydata-a-nordic-model-for-human-centered-personal-data-management-and-processing}{}
{[}11{]} \textsc{Poikola, Antti}\,; \textsc{Kuikkaniemi, Kai}\,;
\textsc{Honko, Harri}: MyData -- a nordic model for human-centered
personal data management and processing, Ministry of Transport;
Communications (2015), pp.~1--12 ---~ISBN~978-952-243-455-5

\hypertarget{ref-web_2016_meeco-how-it-works}{}
{[}12{]} \emph{Meeco how it works}. URL
\url{https://meeco.me/how-it-works.html}. - retrieved 2016-11-09

\hypertarget{ref-repo_2016_pdaas-spec}{}
{[}13{]} \emph{Open specification of the concept called personal data as
a service (pdaas). GitHub}. URL
\url{https://github.com/lucendio/pdaas_spec}. - retrieved 2016-11-11

\hypertarget{ref-report_2014_data-brokers}{}
{[}14{]} \textsc{USA, Federal Trade Commission}: \emph{Data brokers},
2014 --~scale

\hypertarget{ref-whitepaper_2012_the-value-of-our-digital-identity_definition}{}
{[}15{]} \textsc{Rose, John}\,; \textsc{Rehse, Olaf}\,; \textsc{Röber,
Björn}: The value of our digital identity. In: \emph{Boston Cons. Gr}
(2012)

\hypertarget{ref-regulation_2016_eu_general-data-protection-regulation_definition}{}
{[}16{]} Regulation (EU) 2016/679 --- General data protection
regulation, 2016 --~scale

\hypertarget{ref-web_2016_wikipedia_information-privacy-law_us}{}
{[}17{]} \textsc{Wikipedia}: \emph{Information privacy law}. URL
\url{https://en.wikipedia.org/wiki/Information_privacy_law\#United_States}.
- retrieved 2016-11-20. ---~Page Version ID: 749338152

\hypertarget{ref-web_2016_data-protection-laws-in-the-us}{}
{[}18{]} \textsc{Loeb), Ieuan Jolly (Loeb \&}: \emph{PLC - data
protection in the united states: Overview}. URL
\url{http://us.practicallaw.com/6-502-0467}. - retrieved 2016-11-20

\hypertarget{ref-web_2015_white-house-releases-consumer-privacy-bill-draft}{}
{[}19{]} \textsc{Wilhelm, Alex}: \emph{White house drops ``consumer
privacy bill of rights act'' draft. TechCrunch}. URL
\url{http://social.techcrunch.com/2015/02/27/white-house-drops-consumer-privacy-bill-of-rights-act-draft/}.
- retrieved 2016-11-20

\hypertarget{ref-bill-draft_2015_us_consumer-privacy-bill-of-rights-act_definition}{}
{[}20{]} Consumer privacy bill of rights act (cpbora) --- Administration
discussion draft: Consumer privacy bill of rights act of 2015, 2015
--~scale

\hypertarget{ref-rules_2016_fcc_to-protect-broadband-consumer-privacy_sensitive-types-of-data}{}
{[}21{]} FCC 16-148 --- Report and order, 2016 --~scale. ---~In the
Matter of Protecting the Privacy of Customers of Broadband and Other
Telecommunications Services

\hypertarget{ref-rules_2016_fcc_to-protect-broadband-consumer-privacy_personally-identifiable-information}{}
{[}22{]} FCC 16-39 --- Notice of proposed rulemaking, 2016 --~scale.
---~In the Matter of Protecting the Privacy of Customers of Broadband
and Other Telecommunications Services

\hypertarget{ref-web_2016_privacy-policies-are-mandatory-by-law}{}
{[}23{]} \emph{Privacy policies are mandatory by law}. URL
\url{https://termsfeed.com/blog/privacy-policy-mandatory-law/}. -
retrieved 2016-11-20. ---~Disclaimer: Legal information is not legal
advice

\hypertarget{ref-web_2016_facebooks-landing-page_policy-acknowledgement}{}
{[}24{]} \textsc{Facebook}: \emph{facebook - creating an account}. URL
\url{https://www.facebook.com/}. - retrieved 2016-11-20

\hypertarget{ref-web_2016_international-privacy-standards}{}
{[}25{]} \emph{International privacy standards}. URL
\url{https://www.eff.org/issues/international-privacy-standards}. -
retrieved 2016-11-20

\hypertarget{ref-web_2017_privacy-shield_faq}{}
{[}26{]} \textsc{Jan Philipp Albrecht, MdEP}: \emph{EU-US ``privacy
shield'' - background and frequently asked questions (faq)}. URL
\url{https://www.janalbrecht.eu/themen/datenschutz-digitalisierung-netzpolitik/eu-us-privacy-shield-2.html}.
- retrieved 2017-02-06

\hypertarget{ref-web_2017_privacy-shield_kritik}{}
{[}27{]} \textsc{Dachwitz, Ingo}: \emph{Nationale datenschutzbehörden
kritisieren privacy shield und kündigen umfassende prüfung an}. URL
\url{https://netzpolitik.org/2016/nationale-datenschutzbehoerden-kritisieren-privacy-shield-und-kuendigen-umfassende-pruefung-an/}.
- retrieved 2017-02-06

\hypertarget{ref-paper_2014_who-owns-yours-data}{}
{[}28{]} \textsc{Rosner, Gilad}: Who owns your data? In: ~: ACM Press,
2014 ---~ISBN~978-1-4503-3047-3, pp.~623--628

\hypertarget{ref-book_1987_private-ownership_definition}{}
{[}29{]} \textsc{Grunebaum, J.O.}: \emph{Private ownership},
\emph{Problems of philosophy}~: Routledge \& Kegan Paul, 1987 --~scale
---~ISBN~9780710207067

\hypertarget{ref-regulation_2016_eu_general-data-protection-regulation_ownership}{}
{[}30{]} Regulation (EU) 2016/679 --- General data protection
regulation, 2016 --~scale

\hypertarget{ref-rules_2016_fcc_to-protect-broadband-consumer-privacy_ownership}{}
{[}31{]} FCC 16-148 --- Report and order, 2016 --~scale. ---~In the
Matter of Protecting the Privacy of Customers of Broadband and Other
Telecommunications Services

\hypertarget{ref-web_2016_facebook_terms-of-service}{}
{[}32{]} \textsc{Facebook}: \emph{Facebooks's terms of service.
Statement of rights and responsibilities}. URL
\url{https://www.facebook.com/legal/terms}. - retrieved 2016-12-01

\hypertarget{ref-web_2016_twitter_terms-of-service}{}
{[}33{]} \textsc{Twitter}: \emph{Twitters's terms of service. Twitter
terms of service}. URL \url{https://twitter.com/tos\#intlTerms}. -
retrieved 2016-12-01

\hypertarget{ref-web_2016_google_terms-of-service}{}
{[}34{]} \textsc{Google}: \emph{Google's terms of service. Google terms
of service}. URL
\url{https://www.google.com/intl/en/policies/terms/regional.html}. -
retrieved 2016-12-01

\hypertarget{ref-web_2016_apple-icloud_terms-of-service}{}
{[}35{]} \textsc{Apple}: \emph{Apple's iClound terms and conditions. V.
content and your conduct}. URL
\url{https://www.apple.com/legal/internet-services/icloud/en/terms.html}.
- retrieved 2016-12-01

\hypertarget{ref-web_2013_why-metadata-matters}{}
{[}36{]} \emph{Why metadata matters}. URL
\url{https://www.eff.org/deeplinks/2013/06/why-metadata-matters}. -
retrieved 2016-11-24

\hypertarget{ref-web_2016_why-you-need-metadata-for-big-data-to-success}{}
{[}37{]} \textsc{Stevens, John P.}: \emph{Why you need metadata for big
data success}. URL
\url{http://www.datasciencecentral.com/profiles/blogs/why-you-need-metadata-for-big-data-success}.
- retrieved 2016-11-24

\hypertarget{ref-web_2016_oxford_definition_big-data}{}
{[}38{]} \emph{Big data n.} URL
\url{http://www.oed.com/view/Entry/18833\#eid301162177}. - retrieved
2016-11-11

\hypertarget{ref-web_2016_wikipedia_definition_big-data}{}
{[}39{]} \textsc{Wikipedia}: \emph{Big data}. URL
\url{https://en.wikipedia.org/w/index.php?title=Big_data\&oldid=748964100}.
- retrieved 2016-11-11. ---~Page Version ID: 748964100

\hypertarget{ref-chapter_2007_the-knowledge-discovery-process}{}
{[}40{]} \textsc{Cios, Krzysztof J.}\,; \textsc{Swiniarski, Roman W.}\,;
\textsc{Pedrycz, Witold}\,; \textsc{Kurgan, Lukasz A.}: The knowledge
discovery process. In: \emph{Data mining}~: Springer, 2007, pp.~9--24

\hypertarget{ref-paper_2009_a-data-mining-knowledge-discovery-process-model}{}
{[}41{]} \textsc{Marbán, Óscar}\,; \textsc{Mariscal, Gonzalo}\,;
\textsc{Segovia, Javier}: A data mining \& knowledge discovery process
model. In: \emph{Data mining and knowledge discovery in real life
applications}. Madrid, Span~: InTech, 2009

\hypertarget{ref-web_2016_facebook-utilizes-98-data-points}{}
{[}42{]} \textsc{Dewey, Caitlin}\,; \textsc{Dewey, Caitlin}: 98 personal
data points that facebook uses to target ads to you. In: \emph{The
Washington Post} (2016)

\hypertarget{ref-web_2016_big-data-types-of-data-used-in-analytics}{}
{[}43{]} \textsc{Taie, Mohammed Zuhair Al}: \emph{Big data: Types of
data used in analytics. Agroknow blog}. URL
\url{http://blog.agroknow.com/?p=4690}. - retrieved 2017-02-08

\hypertarget{ref-book-chapter_1999_Principles-of-knowledge-discovery-in-databases_introduction-to-data-mining}{}
{[}44{]} \textsc{Zaïane, Osmar R}: \emph{Principles of knowledge
discovery in databases}, 1999 --~scale

\hypertarget{ref-web_2013_big-data-collection-collides-with-privacy-concerns}{}
{[}45{]} \emph{Big data collection collides with privacy concerns,
analysts say. PCWorld}. URL
\url{http://www.pcworld.com/article/2027789/big-data-collection-collides-with-privacy-concerns-analysts-say.html}.
- retrieved 2016-11-15

\hypertarget{ref-web_2016_answers-io}{}
{[}46{]} \emph{Answers.io. Answers}. URL
\url{https://answers.io/answers}. - retrieved 2016-11-14

\hypertarget{ref-web_2016_big-data-enthusiasts-should-not-ignore}{}
{[}47{]} \textsc{Burgelman, Author: Luc}\,; \textsc{Burgelman, NGDATA
Luc}\,; \textsc{NGDATA}: \emph{Attention, big data enthusiasts: Here's
what you shouldn't ignore. WIRED}. URL
\url{https://www.wired.com/insights/2013/02/attention-big-data-enthusiasts-heres-what-you-shouldnt-ignore/}.
- retrieved 2016-11-15. ---~Partner Content

\hypertarget{ref-report_2001_3d-data-management-controlling-data-volume-velocity-and-variety}{}
{[}48{]} \textsc{Laney, Douglas}: \emph{3D data management: Controlling
data volume, velocity, and variety}~: META Group, 2001 --~scale

\hypertarget{ref-paper_2015_big-data-for-development-a-review-of-promises-and-challenges:more-data}{}
{[}49{]} \textsc{Hilbert, Martin}: Big data for development: A review of
promises and challenges. In: \emph{Development Policy Review} vol. 34
(2015), Nr.~1, pp.~135--174

\hypertarget{ref-web_2016_the-state-of-big-data}{}
{[}50{]} \textsc{Davis Kho, Nancy}: \emph{The state of big data}. URL
\url{http://www.econtentmag.com/Articles/Editorial/Feature/The-State-of-Big-Data-108666.htm}.
- retrieved 2016-11-18

\hypertarget{ref-web_2016_apple_customer-letter}{}
{[}51{]} \textsc{CEO), Tim Cook (Apple's}: \emph{A message to our
customers. Customer letter}. URL
\url{http://www.apple.com/customer-letter/}. - retrieved 2016-11-18

\hypertarget{ref-web_2016_what-is-differential-privacy}{}
{[}52{]} \textsc{Green, Matthew}: \emph{What is differential privacy? A
few thoughts on cryptographic engineering}. URL
\url{https://blog.cryptographyengineering.com/2016/06/15/what-is-differential-privacy/}.
- retrieved 2016-11-18

\hypertarget{ref-web_2016_eff_whatsapp-rolls-out-emd-to-end-encryption}{}
{[}53{]} \textsc{Budington, Bill}: \emph{WhatsApp rolls out end-to-end
encryption to its over one billion users}. URL
\url{https://www.eff.org/deeplinks/2016/04/whatsapp-rolls-out-end-end-encryption-its-1bn-users}.
- retrieved 2016-11-18

\hypertarget{ref-web_2016_research-experiment_ai-rembrandt}{}
{[}54{]} \emph{The next rembrandt: Blurring the lines between art,
technology and emotion. Microsoft news centre europe}. URL
\url{https://news.microsoft.com/europe/features/the-next-rembrandt-blurring-the-lines-between-art-technology-and-emotion-2/}.
- retrieved 2017-02-08

\hypertarget{ref-web_2007_introducing-google-traffic}{}
{[}55{]} \emph{Stuck in traffic? Insights from googlers into our
products, technology, and the google culture}. URL
\url{https://googleblog.blogspot.com/2007/02/stuck-in-traffic.html}. -
retrieved 2016-11-18

\hypertarget{ref-web_2016_wikipedia_google-traffic}{}
{[}56{]} \textsc{Wikipedia}: \emph{Google traffic}. URL
\url{https://en.wikipedia.org/w/index.php?title=Google_Traffic\&oldid=746200591}.
- retrieved 2016-11-18. ---~Page Version ID: 746200591

\hypertarget{ref-graphic_2016_global-mobile-os-market-share}{}
{[}57{]} \emph{Global mobile OS market share}. URL
\url{https://www.statista.com/statistics/266136/global-market-share-held-by-smartphone-operating-systems/}.
- retrieved 2016-11-18

\hypertarget{ref-estimating-the-locations-of-emergency-events-from-twitter-streams_2014}{}
{[}58{]} \textsc{Ao, Ji}\,; \textsc{Zhang, Peng}\,; \textsc{Cao, Yanan}:
Estimating the Locations of Emergency Events from Twitter Streams. In:
\emph{Procedia Computer Science} vol. 31 (2014), pp.~731--739

\hypertarget{ref-paper_2015_improving-power-grid-monitoring-data-quality-an-efficient-machine-learning-framework-for-missing-data-prediction}{}
{[}59{]} \textsc{Shi, Weiwei}\,; \textsc{Zhu, Yongxin}\,; \textsc{Zhang,
Jinkui}\,; \textsc{Tao, Xiang}\,; \textsc{Sheng, Gehao}\,; \textsc{Lian,
Yong}\,; \textsc{Wang, Guoxing}\,; \textsc{Chen, Yufeng}: Improving
power grid monitoring data quality: An efficient machine learning
framework for missing data prediction. In: ~: IEEE, 2015
---~ISBN~978-1-4799-8937-9, pp.~417--422

\hypertarget{ref-the-practice-of-predictive-analytics-in-healthcare_2013}{}
{[}60{]} \textsc{Palem, Gopalakrishna}: The Practice of Predictive
Analytics in Healthcare. In: \emph{ResearchGate} (2013)

\hypertarget{ref-data-collection-for-climate-changes_2014}{}
{[}61{]} \textsc{Burger, Nicholas}\,; \textsc{Ghosh-Dastidar, Bonnie}\,;
\textsc{Grant, Audra}\,; \textsc{Joseph, George}\,; \textsc{Ruder,
Teague}\,; \textsc{Tchakeva, Olesya}\,; \textsc{Wodon, Quentin}: Data
Collection for the Study on Climate Change and Migration in the MENA
Region (2014)

\hypertarget{ref-graphic_2015_applications-of-big-data-in-10-industry-verticals}{}
{[}62{]} \textsc{Gaitho, Maryanne}: \emph{Applications of big data in 10
industry verticals}. URL
\url{https://cfs22.simplicdn.net/ice9/free_resources_article_thumb/Applications_of_big_data_infographic.png}.
- retrieved 2016-11-19

\hypertarget{ref-graphic_2012_personal-data-ecosystem}{}
{[}63{]} \textsc{USA, Federal Trade Commission}: \emph{Personal data
ecosystem}. URL
\url{https://www.ftc.gov/sites/default/files/documents/public_events/exploring-privacy-roundtable-series/personaldataecosystem.pdf}.
- retrieved 2016-11-17. ---~Protecting Consumer Privacy in an Era of
Rapid Change - Recommendations for Business and Policymakers - FTC
Report

\hypertarget{ref-video_2016_corporate-surveillance-digital-tracking-big-data-privacy}{}
{[}64{]} \textsc{Christl, Wolfie}: \emph{Corporate surveillance, digital
tracking, big data \& privacy}, 2016 --~scale

\hypertarget{ref-paper_1965_moors-law}{}
{[}65{]} \textsc{Moore, Gordon E.}: Cramming more components onto
integrated circuits. In: \emph{Electronics} vol. 38 (1965), p.~4

\hypertarget{ref-podcast_2015_cre-neuronale-netze}{}
{[}66{]} \textsc{Pritlove, Tim}\,; \textsc{Schöneberg, Ulf}:
\emph{Neuronale netze}, 2015 --~scale

\hypertarget{ref-web_2016_industries-intention-to-invest-in-big-data}{}
{[}67{]} \textsc{Columbus, Louis}: \emph{51\% of enterprises intend to
invest more in big data}. URL
\url{http://www.forbes.com/sites/louiscolumbus/2016/05/22/51-of-enterprises-intend-to-invest-more-in-big-data/}.
- retrieved 2016-12-07

\hypertarget{ref-web_2016_projectvrm_development-work}{}
{[}68{]} \emph{ProjectVRM - cDevelopment work. ProjectVRM}. URL
\url{https://cyber.harvard.edu/projectvrm/VRM_Development_Work}. -
retrieved 2016-12-09

\hypertarget{ref-web_2016_projectvrm_principles}{}
{[}69{]} \emph{ProjectVRM - principles. ProjectVRM}. URL
\url{https://cyber.harvard.edu/projectvrm/Main_Page\#VRM_Principles}. -
retrieved 2016-12-09

\hypertarget{ref-web_2011_tas3-project}{}
{[}70{]} \emph{TAS3 - project overview}. URL
\url{http://vds1628.sivit.org/tas3/}. - retrieved 2017-02-10

\hypertarget{ref-graphic_2011_architecture_components-of-organization-domain}{}
{[}71{]} \textsc{The TAS3 Consortium}: \emph{TAS3 architecture - figure
2.2: Major components of organization domain.}, 2011 --~scale. ---~v
2.24

\hypertarget{ref-web_kantara-initiative}{}
{[}72{]} \emph{Kantara initiative -- join. innovate. trust.} URL
\url{https://kantarainitiative.org/}. - retrieved 2016-12-14

\hypertarget{ref-paper_2014_personal-data-store-approach}{}
{[}73{]} \textsc{Kirkham, Tom}\,; \textsc{Winfield, Sandra}\,;
\textsc{Ravet, Serge}\,; \textsc{Kellomaki, Sampo}: The personal data
store approach to personal data security. In: \emph{IEEE Security \&
Privacy} vol. 11 (2013), Nr.~5, pp.~12--19

\hypertarget{ref-paper_2012_openpds_on-trusted-use-of-large-scale-personal-data}{}
{[}74{]} \textsc{Montjoye, Yves-Alexandre de}\,; \textsc{Wang, Samuel
S.}\,; \textsc{Pentland, Alex}\,; \textsc{Anh, Dinh Tien Tuan}\,;
\textsc{Datta, Anwitaman}\,; \textsc{others}: On the trusted use of
large-scale personal data. In: \emph{IEEE Data Eng. Bull.} vol. 35
(2012), Nr.~4, pp.~5--8

\hypertarget{ref-web_mit_openpds-safeanswers-project-page}{}
{[}75{]} \emph{openPDS/SafeAnswers - the privacy-preserving personal
data store}. URL \url{http://openpds.media.mit.edu/}. - retrieved
2016-12-14

\hypertarget{ref-paper_2014_openpds_protecting-privacy-of-meta-data-through-safeanswers}{}
{[}76{]} \textsc{Montjoye, Yves-Alexandre de}\,; \textsc{Shmueli,
Erez}\,; \textsc{Wang, Samuel S.}\,; \textsc{Pentland, Alex Sandy}:
openPDS: Protecting the privacy of metadata through SafeAnswers. In:
\textsc{Preis, T.} (ed.) \emph{PLoS ONE} vol. 9 (2014), Nr.~7, p.~e98790

\hypertarget{ref-web_microsoft_healthvault}{}
{[}77{]} \emph{Microsoft HealthVault. Overview}. URL
\url{https://www.healthvault.com/de/en/overview}. - retrieved 2016-12-14

\hypertarget{ref-web_meeco_how-it-works}{}
{[}78{]} \emph{How it works meeco}. URL
\url{https://meeco.me/how-it-works.html}. - retrieved 2016-12-14

\hypertarget{ref-slides_2015_meeco-case-study}{}
{[}79{]} \textsc{Page, Mike}: Online adver\textgreater{}sing -- booming
or broken?, 2015 --~scale

\hypertarget{ref-web_industrial-data-space}{}
{[}80{]} \emph{The principles. Industrial data space e.V.} URL
\url{http://www.industrialdataspace.org/en/the-principles/}. - retrieved
2016-12-14

\hypertarget{ref-whitepaper_2016_industrial-data-space}{}
{[}81{]} \textsc{Prof. Dr.-Ing. Otto, Boris}\,; \textsc{Prof. Dr. Auer,
Sören}\,; \textsc{Cirullies, Jan}\,; \textsc{Prof. Dr. Jürjens, Jan}\,;
\textsc{Menz, Nadja}\,; \textsc{Schon, Jochen}\,; \textsc{Dr. Wenzel,
Sven}: Industrial data space - digital sovereignity over data.

\hypertarget{ref-web_spec_http1}{}
{[}82{]} \textsc{Leach, Paul J.}\,; \textsc{Berners-Lee, Tim}\,;
\textsc{Mogul, Jeffrey C.}\,; \textsc{Masinter, Larry}\,;
\textsc{Fielding, Roy T.}\,; \textsc{Gettys, James}: \emph{Hypertext
transfer protocol -- HTTP/1.1}. URL
\url{https://tools.ietf.org/html/rfc2616}. - retrieved 2016-12-17

\hypertarget{ref-web_spec_http2}{}
{[}83{]} \textsc{Belshe, Mike}\,; \textsc{Thomson, Martin}\,;
\textsc{Peon, Roberto}: \emph{Hypertext transfer protocol version 2
(HTTP/2)}. URL \url{https://tools.ietf.org/html/rfc7540}. - retrieved
2016-12-17

\hypertarget{ref-web_spec_websockets}{}
{[}84{]} \textsc{Fette, Ian}\,; \textsc{Melnikov, A.}: \emph{The
WebSocket protocol}. URL \url{https://tools.ietf.org/html/rfc6455}. -
retrieved 2016-12-17

\hypertarget{ref-web_spec_json}{}
{[}85{]} \textsc{Crockford, Douglas}: The JSON data interchange format.

\hypertarget{ref-web_rfc_json}{}
{[}86{]} \textsc{Bray, T.}: \emph{The JavaScript object notation (JSON)
data interchange format}. URL \url{https://tools.ietf.org/html/rfc7159}.
- retrieved 2016-12-17

\hypertarget{ref-web_2012_problem-with-oauth-for-authentication}{}
{[}87{]} \textsc{Bradley, John}: \emph{The problem with OAuth for
authentication.} URL
\url{http://www.thread-safe.com/2012/01/problem-with-oauth-for-authentication.html}.
- retrieved 2016-12-17

\hypertarget{ref-web_spec_oauth-1a}{}
{[}88{]} \emph{OAuth core 1.0a}. URL \url{https://oauth.net/core/1.0a/}.
- retrieved 2016-12-18

\hypertarget{ref-web_spec_oauth-2}{}
{[}89{]} \textsc{Hardt, Dick}: \emph{The OAuth 2.0 authorization
framework}. URL \url{https://tools.ietf.org/html/rfc6749}. - retrieved
2016-12-18

\hypertarget{ref-web_2016_oauth-2}{}
{[}90{]} \textsc{WG, IETF OAuth}: \emph{OAuth 2.0}. URL
\url{https://oauth.net/2/}. - retrieved 2016-12-16

\hypertarget{ref-web_spec_openid-spec-index}{}
{[}91{]} \emph{Specifications \& developer information OpenID}. URL
\url{https://openid.net/developers/specs/}. - retrieved 2017-02-10

\hypertarget{ref-web_spec_openid-connect-1}{}
{[}92{]} \emph{OpenID connect core 1.0 incorporating errata set 1}. URL
\url{https://openid.net/specs/openid-connect-core-1_0.html}. - retrieved
2016-12-17

\hypertarget{ref-web_2017_wikipedia_openid-vs-pseudo-oauth}{}
{[}93{]} \emph{OpenID - openid vs. pseudo-authentication using oauth}.
URL
\url{https://en.wikipedia.org/w/index.php?title=OpenID\&oldid=763047614\#OpenID_vs._pseudo-authentication_using_OAuth}.
- retrieved 2017-02-10. ---~Page Version ID: 763047614

\hypertarget{ref-web_spec_json-web-token}{}
{[}94{]} \textsc{Bradley, John}\,; \textsc{Sakimura, Nat}\,;
\textsc{Jones, Michael}: \emph{JSON web token (JWT)}. URL
\url{https://tools.ietf.org/html/rfc7519}. - retrieved 2016-12-17

\hypertarget{ref-web_spec_json-web-encryption}{}
{[}95{]} \textsc{Hildebrand, Joe}\,; \textsc{Jones, Michael}: \emph{JSON
web encryption (JWE)}. URL \url{https://tools.ietf.org/html/rfc7516}. -
retrieved 2016-12-17

\hypertarget{ref-web_spec_json-web-signature}{}
{[}96{]} \textsc{Bradley, John}\,; \textsc{Sakimura, Nat}\,;
\textsc{Jones, Michael}: \emph{JSON web signature (JWS)}. URL
\url{https://tools.ietf.org/html/rfc7515}. - retrieved 2016-12-17

\hypertarget{ref-paper_1976_d-h-key-exchange}{}
{[}97{]} \textsc{Diffie, Whitfield}\,; \textsc{Hellman, Martin}: New
directions in cryptography. In: \emph{IEEE transactions on Information
Theory} vol. 22 (1976), Nr.~6, pp.~644--654

\hypertarget{ref-book_2014_chapter-9-1-public-key-crypto}{}
{[}98{]} \textsc{Stallings, William}: 9.1 principles of public-key
cryptosystems. In: \emph{Cryptography and network security: Principles
and practice}. Seventh edition. ed. Boston~: Pearson, 2014
---~ISBN~978-0-13-335469-0, pp.~256--264

\hypertarget{ref-web_spec_tls}{}
{[}99{]} \textsc{Dierks, Tim}\,; \textsc{Rescorla, E.}: \emph{The
transport layer security (TLS) protocol version 1.2}. URL
\url{https://tools.ietf.org/html/rfc5246}. - retrieved 2016-12-17

\hypertarget{ref-book_2014_chapter-14-5-pki}{}
{[}100{]} \textsc{Stallings, William}: 10.5 pseudorandom number
generation based on an asymmetric cipher. In: \emph{Cryptography and
network security: Principles and practice}. Seventh edition. ed.
Boston~: Pearson, 2014 ---~ISBN~978-0-13-335469-0, pp.~443--445

\hypertarget{ref-web_spec_x509}{}
{[}101{]} \textsc{Cooper, Dave}\,; \textsc{Santesson, S.}\,;
\textsc{Farrell, S.}\,; \textsc{Boeyen, S.}\,; \textsc{Housley, W.,
\textnormal{R. andPolk}}: \emph{Internet x.509 public key infrastructure
certificate and certificate revocation list (CRL) profile}. URL
\url{https://tools.ietf.org/html/rfc5280}. - retrieved 2017-01-11

\hypertarget{ref-web_spec_rest}{}
{[}102{]} \textsc{Fielding, Thomas}: Representational state transfer
(REST). In: \emph{Architectural styles and the design of network-based
software architectures}~: University of California, Irvine, 2000,
pp.~76--106

\hypertarget{ref-web_spec_http-methods}{}
{[}103{]} \textsc{Leach, Paul J.}\,; \textsc{Berners-Lee, Tim}\,;
\textsc{Mogul, Jeffrey C.}\,; \textsc{Masinter, Larry}\,;
\textsc{Fielding, Roy T.}\,; \textsc{Gettys, James}: \emph{HTTP
methods}. URL \url{https://tools.ietf.org/html/rfc2616\#section-9}. -
retrieved 2016-12-18

\hypertarget{ref-web_spec_graphql}{}
{[}104{]} \emph{GraphQL}. URL \url{https://facebook.github.io/graphql/}.
- retrieved 2016-12-17

\hypertarget{ref-web_w3c-tr_rdf}{}
{[}105{]} \textsc{Beckett, Dave}\,; \textsc{McBride, Brian}:
\emph{RDF/XML syntax specification (revised)}. URL
\url{https://www.w3.org/TR/REC-rdf-syntax/}. - retrieved 2016-12-19

\hypertarget{ref-web_w3c-tr_owl}{}
{[}106{]} \textsc{W3C OWL Working Group}: \emph{OWL 2 web ontology
language document overview (second edition)}. URL
\url{https://www.w3.org/TR/owl2-overview/}. - retrieved 2016-12-19

\hypertarget{ref-web_w3c-tr_sparql}{}
{[}107{]} \textsc{Harris, Steve}\,; \textsc{Seaborne, Andy}\,;
\textsc{Prud'hommeaux, Eric}: \emph{SPARQL 1.1 query language}. URL
\url{https://www.w3.org/TR/sparql11-query/}. - retrieved 2016-12-19

\hypertarget{ref-web_w3c-draft_webid}{}
{[}108{]} \emph{WebID specifications}. URL
\url{https://www.w3.org/2005/Incubator/webid/spec/}. - retrieved
2016-12-19

\hypertarget{ref-web_spec_solid}{}
{[}109{]} \emph{Solid specification}. URL
\url{https://github.com/solid/solid-spec}. - retrieved 2016-12-17

\hypertarget{ref-web_2016_wiki_webaccesscontrol}{}
{[}110{]} \emph{WebAccessControl - w3c wiki}. URL
\url{https://www.w3.org/wiki/WebAccessControl}. - retrieved 2016-12-19

\hypertarget{ref-web_2016_demo_databox}{}
{[}111{]} \emph{Databox.me}. URL \url{https://databox.me/}. - retrieved
2016-12-19

\hypertarget{ref-web_2015_cgroup-doc}{}
{[}112{]} \textsc{Heo, Tejun}: \emph{Control group (v2) documentation}.
URL \url{https://www.kernel.org/doc/Documentation/cgroup-v2.txt}. -
retrieved 2016-12-20

\hypertarget{ref-web_2016_kernel-namespace}{}
{[}113{]} \emph{Overview of linux namespaces}. URL
\url{http://man7.org/linux/man-pages/man7/namespaces.7.html}. -
retrieved 2016-12-20

\hypertarget{ref-web_2016_open-container-initiative}{}
{[}114{]} \emph{Open container initiative}. URL
\url{https://www.opencontainers.org/}. - retrieved 2016-12-20

\hypertarget{ref-web_oci-spec_runtime}{}
{[}115{]} \emph{Container runtime specification (v1.0.0-rc3)}. URL
\url{https://github.com/opencontainers/runtime-spec/tree/v1.0.0-rc3}. -
retrieved 2016-12-20

\hypertarget{ref-web_oci-spec_image}{}
{[}116{]} \emph{Container image specification (v1.0.0-rc3)}. URL
\url{https://github.com/opencontainers/image-spec/tree/v1.0.0-rc3}. -
retrieved 2016-12-20

\hypertarget{ref-web_2013_npa-sicherheitsdefizit}{}
{[}117{]} \emph{Basisleser weiterhin kritische schwachstelle des
elektronischen / neuen personalausweises. Netzpolitik.org}. URL
\url{https://netzpolitik.org/2013/basisleser-weiterhin-kritische-schwachstelle-des-elektronischen-neuen-personalausweises/}.
- retrieved 2017-01-05

\hypertarget{ref-web_2014_test-qes-support-in-npa}{}
{[}118{]} \textsc{Stiemerling, Oliver}: \emph{Qualifizierte
elektronische signatur mit dem neuen personalausweis -- oder: QES mit
nPA, ein selbstversuch. CR-online.de blog}. URL
\url{http://www.cr-online.de/blog/2014/08/26/qualifizierte-elektronische-signatur-mit-dem-neuen-personalausweis-oder-qes-mit-npa-ein-selbstversuch/}.
- retrieved 2017-01-05

\hypertarget{ref-web_2017_about-de-mail}{}
{[}119{]} \textsc{Bundesregierung für Informationstechnik, Der
Bundesbeauftragte der}: \emph{IT-beauftragter der bundesregierung
de-mail}. URL
\url{http://www.cio.bund.de/Web/DE/Innovative-Vorhaben/De-Mail/de_mail_node.html}.
- retrieved 2017-01-06

\hypertarget{ref-statement_2013_de-mail}{}
{[}120{]} \textsc{Neumann, Linus}: Stellungnahme zum elektronischen
rechtsverkehr.

\hypertarget{ref-book_2015_ethical-it-innovation}{}
{[}121{]} \textsc{Spiekermann, Sarah}: \emph{Ethical IT Innovation: A
Value-Based System Design Approach}~: CRC Press; Taylor \& Francis
Group, LLC, 2015 --~scale ---~ISBN~978-1-4822-2635-5

\hypertarget{ref-paper_2004_distributed-mapreduce}{}
{[}122{]} \textsc{Dean, Eeffrey}\,; \textsc{Ghemawat, Sanjay}:
MapRednce: Simplified data processing on large clusters (2004)

\hypertarget{ref-web_spec_tls-12_client-auth}{}
{[}123{]} \textsc{Dierks, Tim}: \emph{The transport layer security (TLS)
protocol version 1.2}. URL
\url{https://tools.ietf.org/html/rfc5246\#section-7.4.6}. - retrieved
2017-01-09

\hypertarget{ref-web_2017_wikipedia_mutual-auth}{}
{[}124{]} \emph{Mutual authentication}. URL
\url{https://en.wikipedia.org/w/index.php?title=Mutual_authentication\&oldid=737409981}.
- retrieved 2017-01-10. ---~Page Version ID: 737409981

\hypertarget{ref-book_2013_networking-101_tls-session-resumption}{}
{[}125{]} \emph{Networking 101: Transport layer security (TLS) - high
performance browser networking (o'Reilly). High performance browser
networking}. URL
\url{https://hpbn.co/transport-layer-security-tls/\#tls-session-resumption}.
- retrieved 2017-01-12

\hypertarget{ref-web_spec_tls-session-ticket-resumption}{}
{[}126{]} \textsc{Joseph Salowey, P. Eronen, \textnormal{H. Zhou}}:
\emph{Transport layer security (TLS) session resumption without
server-side state}. URL \url{https://tools.ietf.org/html/rfc5077}. -
retrieved 2017-01-12

\hypertarget{ref-web_bsi-spec_eid}{}
{[}127{]} \emph{BSI - technische richtlinien des BSI - BSI TR-03130
eID-server}. URL
\url{https://www.bsi.bund.de/DE/Publikationen/TechnischeRichtlinien/tr03130/tr-03130.html}.
- retrieved 2017-01-06

\hypertarget{ref-web_2017_npa-eid-server}{}
{[}128{]} \emph{Personalausweisportal - eID-server}. URL
\url{https://personalausweisportal.de/DE/Wirtschaft/Technik/eID-Server/eID-Server_node.html}.
- retrieved 2017-01-06

\hypertarget{ref-book_2014_chapter-10-5-asym-random-number-gen}{}
{[}129{]} \textsc{Stallings, William}: 9.1 public-key infrastructure.
In: \emph{Cryptography and network security: Principles and practice}.
Seventh edition. ed. Boston~: Pearson, 2014 ---~ISBN~978-0-13-335469-0,
p.~307

\hypertarget{ref-web_spec_http-error-codes}{}
{[}130{]} \textsc{Leach, Paul J.}\,; \textsc{Berners-Lee, Tim}\,;
\textsc{Mogul, Jeffrey C.}\,; \textsc{Masinter, Larry}\,;
\textsc{Fielding, Roy T.}\,; \textsc{Gettys, James}: \emph{Hypertext
transfer protocol -- HTTP/1.1}. URL
\url{https://tools.ietf.org/html/rfc2616\#section-10}. - retrieved
2017-01-20

\hypertarget{ref-web_spec_oauth-1a_client-reg}{}
{[}131{]} \emph{OAuth core 1.0a}. URL
\url{https://oauth.net/core/1.0a/\#rfc.section.4.2}. - retrieved
2016-11-01

\hypertarget{ref-web_spec_oauth-2_client-reg}{}
{[}132{]} \textsc{Hardt, Dick}: \emph{The OAuth 2.0 authorization
framework}. URL \url{https://tools.ietf.org/html/rfc6749\#section-2}. -
retrieved 2016-11-01

\hypertarget{ref-web_spec_oauth-1a_access-verification}{}
{[}133{]} \emph{OAuth core 1.0a}. URL
\url{https://oauth.net/core/1.0a/\#rfc.section.7}. - retrieved
2016-11-01

\hypertarget{ref-web_spec_oauth-2_access-verification}{}
{[}134{]} \textsc{Hardt, Dick}: \emph{The OAuth 2.0 authorization
framework}. URL \url{https://tools.ietf.org/html/rfc6749\#section-7}. -
retrieved 2016-11-01

\hypertarget{ref-web_w3c-tr_rdf-schemas}{}
{[}135{]} \textsc{Brickley, Dan}\,; \textsc{Guha, R.V.}: \emph{RDF
schema 1.1}. URL \url{https://www.w3.org/TR/rdf-schema/}. - retrieved
2017-02-12

\hypertarget{ref-web_spec_data-schemas_ehr}{}
{[}136{]} \textsc{Foundation}: \emph{OpenEHR - EHR information model}.
URL \url{http://www.openehr.org/releases/RM/latest/docs/ehr/ehr.html}. -
retrieved 2017-01-28

\hypertarget{ref-web_spec_data-schemas_poi}{}
{[}137{]} \textsc{W3C}: \emph{Points of interest core}. URL
\url{https://www.w3.org/TR/poi-core/}. - retrieved 2017-01-28

\hypertarget{ref-web_spec_data-schemas_bank-transfer}{}
{[}138{]} \textsc{Authority, ISO 20022 Registration}: \emph{ISO 20022 -
universal financial industry message scheme}. URL
\url{https://www.iso20022.org/}. - retrieved 2017-01-28

\hypertarget{ref-web_spec_xml_types}{}
{[}139{]} \textsc{W3C}: \emph{XML schema part 2: Datatypes second
edition}. URL
\url{https://www.w3.org/TR/xmlschema-2/\#built-in-primitive-datatypes}.
- retrieved 2017-01-29

\hypertarget{ref-web_spec_graphql_types}{}
{[}140{]} \textsc{Facebook, Inc.}: \emph{GraphQL Specification}. URL
\url{https://facebook.github.io/graphql/\#sec-Input-Values}. - retrieved
2017-01-29

\hypertarget{ref-web_2016_wikipedia_separation-of-concerns}{}
{[}141{]} \emph{Separation of concerns}. URL
\url{https://en.wikipedia.org/w/index.php?title=Separation_of_concerns\&oldid=747272729}.
- retrieved 2017-01-24. ---~Page Version ID: 747272729

\hypertarget{ref-web_spec_acme}{}
{[}142{]} \textsc{Kasten, James}\,; \textsc{Barnes, Richard}\,;
\textsc{Hoffman-Andrews, Jacob}: \emph{Automatic certificate management
environment (ACME)}. URL
\url{https://tools.ietf.org/html/draft-ietf-acme-acme-04}. - retrieved
2017-01-11

\hypertarget{ref-web_2009-success-of-facebook-connect}{}
{[}143{]} \textsc{Carlson, Nicholas}: \emph{Facebook connect is a huge
success -- by the numbers}. URL
\url{http://www.businessinsider.com/six-months-in-facebook-connect-is-a-huge-success-2009-7}.
- retrieved 2016-12-16

\hypertarget{ref-web_2017_wikipedia_os-market-share}{}
{[}144{]} \emph{Usage share of operating systems}. URL
\url{https://en.wikipedia.org/w/index.php?title=Usage_share_of_operating_systems\&oldid=764383522\#Public_servers_on_the_Internet}.
- retrieved 2017-02-12. ---~Page Version ID: 764383522

\hypertarget{ref-web_2016_wikipedia_accordion-gui}{}
{[}145{]} \emph{Accordion (GUI)}. URL
\url{https://en.wikipedia.org/w/index.php?title=Accordion_(GUI)\&oldid=758084292}.
- retrieved 2017-02-01. ---~Page Version ID: 758084292

\hypertarget{ref-web_2017_repo_node-simple-schema}{}
{[}146{]} \emph{Aldeed/node-simple-schema. GitHub}. URL
\url{https://github.com/aldeed/node-simple-schema}. - retrieved
2017-01-29

\end{document}
